\chapter{Part I: Well Posedness of the Multilayered FSI Systems}
\section{Canonical Multilayered FSI PDE Model}
    \begin{figure}
        \begin{center}
            \includegraphics[width=0.5\textwidth]{Screenshot 2025-04-27 193507.png}
            \caption{Polygonal domain FSI model}
        \end{center}
    \end{figure}

    Let the fluid geometry $\Om_f\subseteq \R^3$ be a Lipschitz, bounded domain. The structure domain $\Om_s\subseteq \R^3$ will be ``completely immersed'' in $\Om_f$; with $\Om_s$ being a convex polyhedral domain. In the figure $\G_f$ is the part of the boundary of $\partial \Om_f$ which does not come into contact with $\Om_s$; $\G_s=\partial \Om_s$ is the boundary interface between $\Om_f$ and $\Om_s$ wherein the coupling between the two distinct fluid and elastic dynamics occurs. (And so $\partial \Om_f=\G_s\cup\G_f$.) It follows that 
    \begin{align}
        \G_s=\cup_{j=1}^K \overline{\G_j}\label{bmeq:1}
    \end{align}
    where $\G_i\cap\G_j=\emptyset$, for $i\neq j$. It is further assumed that each $\G_j$ is an open polygonal domain.\\
    Moreover, $n_j$ will denote the unit normal vector which is exterior to $\partial \G_j,\: 1\leq j\leq K$. With respect to this geometry, the $\R^3$ wave-$\R^2$ wave-$\R^3$ heat interaction PDE model is given as follow: For $1\leq j\leq K$,
    \begin{align}
        &\begin{cases}
            u_t-\Delta u=0\text{ in } (0,T)\times \Om_f\\
            u|_{\G_f}=0\text{ on } (0,T)\times\G_f\label{bmeq:2};
        \end{cases}\\
        &\begin{cases}
            \frac{\partial^2}{\partial t^2}h_j-\Delta h_j+h_j=\frac{\partial w}{\partial \nu}|_{\G_j}-\frac{\partial u}{\partial \nu}|_{\G_j}\text{ on } (0,T)\times \G_j\\
            h_j|_{\partial\G_j\cap\partial\G_l}=h_l|_{\partial\G_j\cap\partial\G_l}\text{ on } (0,T)\times (\partial\G_j\cap\partial\G_l),\text{ for all } 1\leq l\leq K\\
            \text{ such that } \partial\G_j\cap\partial\G_l\neq\emptyset\\
            \frac{\partial h_j}{\partial n_j}\bigg |_{\partial\G_j\cap\partial\G_l}=-\frac{\partial h_l}{\partial n_l}\bigg |_{\partial\G_j\cap\partial\G_l}\text{ on }(0,T)\times(\partial\G_j\cap\partial\G_l)\text{, for all }1\leq l\leq K\\
            \text{ such that }\partial\G_j\cap\partial\G_l\neq \emptyset\label{bmeq:3}
        \end{cases}\\
        &\begin{cases}
            w_{tt}-\Delta w=0\text{ on } (0,T)\times\Om_s\\
            w_t|_{\G_j}=\frac{\partial}{\partial t}h_j=u|_{\G_j}\text{ on } (0,T)\times\G_j\text{, for }j=1,\dots,K\label{bmeq:4}
        \end{cases}\\
        &[u(0),h_1(0),\frac{\partial}{\partial t}h_1(0),\dots,h_K(0),\frac{\partial}{\partial t}h_K(0),w(0),w_t(0)]=[u_0,h_{01},h_{11},\dots,h_{0K},h_{1K},w_0,w_1]\label{bmeq:5}
    \end{align}
    Equation $(\ref{bmeq:3})_1$ is the dynamic coupling condition and represents a balance of forces on $\G_j$. The left-hand side comes from the inertia and elastic energy of the thin structure, while the right-hand side accounts for the contact forces coming from the 3-D structure and the fluid, respectively. The last term of the left-hand side is added to ensure the uniqueness of the solution and physically means that the structure is anchored and therefore the displacement does not have a translational component. The coupling conditions $(\ref{bmeq:3})_2$ and $(\ref{bmeq:3})_3$ represent continuity of the displacement and contact force along the interface between sides $\G_i$ and $\G_l$, respectively. Equation $(\ref{bmeq:4})_2$ is a kinematic coupling condition and accounts for continuity of the velocity across the interface $\G_j$. It corresponds to the no-slip boundary condition in fluid mechanics. Note that the boundary condition in $(\ref{bmeq:4})$ implies that for $t>0$,
    \begin{align*}
        w(t)|_{\G_j}-h_j(t)=w(0)|_{\G_j}-h_j(0)\text{, for } j=1,\dots,K.
    \end{align*}
    Accordingly, the associated space of initial data $\BH$ incorporates a compatibility condition. Namely,
    \begin{align}
        \BH=&\{[u_0,h_{01},h_{11},\dots,h_{0K},h_{1K},w_0,w_1]\in \LP{2}{\Om_f}\times\HM{1}{\G_1}\times\LP{2}{\G_1}\times\dots\nonumber\\
        &\times\HM{1}{\G_K}\times\LP{2}{\G_K}\times\HM{1}{\Om_s}\times\LP{2}{\Om_s}\text{, such that for each } 1\leq j\leq K:\nonumber\\
        &(i) w_0|_{\G_j}=h_{0j};\nonumber\\
        &(ii)h_{0j}|_{\partial\G_j\cap\partial\G_l}=h_{0l}|_{\partial\G_j\cap\partial\G_l}\text{ on } \partial\G_j\cap\partial\G_l\text{, for all } 1\leq l\leq K\text{ such that }\partial\G_j\cap\partial\G_l\neq \emptyset\}\label{bmeq:6}
    \end{align}
    Because of the given boundary interface compatibility condition, $\BH$ is a Hilbert space with the inner product
    \begin{align}
        \IP{\Phi_0,\widetilde{\Phi}_0}{\BH}=\IP{u_0,\widetilde{u}_0}{\Om_f}&+\sum_{j=1}^K \IP{\nabla h_{0j},\nabla \widetilde{h}_{0j}}{\G_j}+\sum_{j=1}^K\IP{h_{0j},\widetilde{h}_{0j}}{\G_j}\nonumber\\
        &+\sum_{j=1}^K \IP{h_{1j},\widetilde{h}_{1j}}{\G_j}+\IP{\nabla w_0,\nabla\widetilde{w}_0}{\Om_s}+\IP{w_1,\widetilde{w}_1}{\Om_s}\label{bmeq:7}
    \end{align}
    where
    \begin{align}
        \Phi_0=[u_0,h_{01},h_{11},\dots,h_{0K},h_{1K},w_0,w_1]\in\BH;\widetilde{\Phi}_0=[\widetilde{u}_0,\widetilde{h}_{01},\widetilde{h}_{11},\dots,\widetilde{h}_{0K},\widetilde{h}_{1K},\widetilde{w}_0,\widetilde{w}_1]\in\BH.\label{bmeq:8}
    \end{align}

\section{Preliminaries}
    With respect to the above setting, the PDE system given above can be recast as an ODE in Hilbert space $\BH$. That is, if $\Phi(t)=[u,h_1,\frac{\partial}{\partial t}h_1,\dots,h_K,\frac{\partial}{\partial t}h_K,w,w_t]\in C([0,T];\BH)$ solves the problem for $\Phi_0\in\BH$, then there is a modeling operator $\A:D(\A)\subseteq \BH\to\BH$ such that $\Phi(\cdot)$ satisfies
    \begin{align}
        \frac{d}{dt}\Phi(t)=A\Phi_0; \quad \Phi(0)=\Phi_0\label{bmeq:9}.
    \end{align}
    In fact, this operator $\A:D(\A)\subseteq\BH\to\BH$ is defined as follows
    \begin{align}
        \A=\begin{bmatrix}
            \Delta&0&0&\dots&0&0&0&0\\
            0&0&I&\dots&0&0&0&0\\
            -\frac{\partial}{\partial \nu}|_{\G_1}&(\Delta-I)&0&\dots&0&0&\frac{\partial}{\partial \nu}|_{\G_1}&0\\
            \vdots&\vdots&\vdots&\ddots&\vdots&\vdots&\vdots&\vdots\\
            0&0&0&\dots&0&I&0&0\\
            -\frac{\partial}{\partial \nu}|_{\G_K}&0&0&\dots&(\Delta-I)&0&\frac{\partial}{\partial \nu}|_{\G_K}&0\\
            0&0&0&\dots&0&0&0&I\\
            0&0&0&\dots&0&0&\Delta&0
        \end{bmatrix};\label{bmeq:10}
    \end{align}
    \begin{align}
        D(&\A)=\{[u_0,h_{01},h_{11},\dots,h_{0K},h_{1K},w_0,w_1]\in\BH:\nonumber\\
        (&\textbf{A.i})\:u_0\in\HM{1}{\Om_f},\:h_{1j}\in\HM{1}{\G_j}\text{ for }1\leq j\leq K,\:w_1\in\HM{1}{\Om_s};\nonumber\\
        (&\textbf{A.ii})(a)\:\Delta u_0\in\LP{2}{\Om_f},\:\Delta w_0\in\LP{2}{\Om_s},\:(b)\:\Delta h_{0j}-\frac{\partial u_0}{\partial \nu}\bigg|_{\G_j}+\frac{\partial w_0}{\partial\nu}\bigg|_{\G_j}\in\LP{2}{\G_j}\nonumber\\
        &\text{ for } 1\leq j\leq K; \text{(c) }\frac{\partial h_{0j}}{\partial n_j}\bigg |_{\partial\G_j}\in \HM{-1/2}{\partial\G_j},\text{ for }1\leq j\leq K;\nonumber\\
        (&\textbf{A.iii})\:u_0|_{\G_f}=0,u_0|_{\G_j}=h_{1j}=w_1|_{\G_j},\text{ for }1\leq j\leq K;\nonumber\\
        (&\textbf{A.iv}) \text{ For }1\leq j\leq K:\nonumber\\
        &(a)\:h_{1j}|_{\partial \G_j\cap\partial\G_l}=h_{1l}|_{\partial\G_j\cap\partial\G_l}\text{ on }\partial\G_j\cap\partial\G_l,\text{ for all }1\leq l\leq K\text{ such that }\partial\G_j\cap\partial\G_l\neq \emptyset;\nonumber\\
        &(b)\:\frac{\partial h_{0j}}{\partial n_j}\bigg |_{\partial\G_j\cap\partial\G_l}=-\frac{\partial h_{0l}}{\partial n_l}\bigg |_{\partial\G_j\cap\partial\G_l}\text{ on }\partial\G_j\cap\partial\G_l,\text{ for all }1\leq l\leq K\text{ such that }\partial\G_j\cap\partial\G_l\neq\emptyset\}.\label{bmeq:11}
    \end{align}

\section{Main Result: Existence-Uniqueness of Solution}
    \begin{thrm}
        The operator $\A:D(\A)\subseteq\BH\to\BH$ defined above generates a $C_0$-semigroup of contractions. Consequently, the solution $\Phi(t)=[u,h_1,\frac{\partial}{\partial t}h_1,\dots,h_K,\frac{\partial}{\partial t}h_K,w,w_t]$ of the PDE model is given by
        \begin{align*}
            \Phi(t)=e^{\A t}\Phi_0\in C([0,T];\BH),
        \end{align*}
        where $\Phi_0=[u_0,h_{01},h_{11},\dots,h_{0K},h_{1k},w_0,w_1]\in\BH$\label{bmthrm:1}
    \end{thrm}

    This section is devoted to prove the Hadamard wellposedness of the coupled system given above. The proof hinges on the application of the Lumer-Phillips Theorem $(see Theorem \ref{LP})$ which assures the existence of a $C_0$-semigroup of contractions $\{e^{\A t}\}_{t\geq 0}$ once it is established that $\A$ is maximally dissipative.\\
    \subsection{Step 1 (Dissipativity of A)} 
    \noindent Given data $\Phi_0$ to be in $D(\A)$,
    \begin{align}
        \IP{\A\Phi_0,\Phi_0}{\BH}&=\IP{\Delta u_0,u_0}{\Om_f}+\sum_{j=1}^K \IP{\nabla h_{1j},\nabla h_{0j}}{\G_j}\nonumber\\
        &+\sum_{j=1}^K\IP{ \frac{\partial u_0}{\partial \nu},h_{1j}}{\G_j}+\sum_{j=1}^K \IP{\frac{\partial w_0}{\partial \nu},h_{1j}}{\G_j}\nonumber\\
        &+\sum_{j=1}^K\IP{h_{0j},h_{1j}}{\G_j}+\sum_{j=1}^K \IP{(\Delta-I)h_{0j},h_{1j}}{\G_j}\nonumber\\
        &+\IP{\nabla w_1,\nabla\widetilde{w}_0}{\Om_s}+\IP{\Delta w_0,w_1}{\Om_s}\nonumber\\
        =&-\IP{\nabla u_0,\nabla u_0}{\Om_f}+\IP{\frac{\partial}{\partial\nu}u_0,u_0 }{\G_s}\nonumber\\
        &+\sum_{j=1}^K \IP{\nabla h_{1j},\nabla h_{0j}}{\G_j}+\sum_{j=1}^K \IP{h_{1j},h_{0j}}{\G_j}\nonumber\\
        &-\sum_{j=1}^K \overline{\IP{\nabla h_{1j},\nabla h_{0j}}{\G_j}}-\sum_{j=1}^K \overline{\IP{h_{1j},h_{0j}}{\G_j}}+\sum_{j=1}^K \IP{\frac{\partial h_{0j}}{\partial n_j},h_{1j}}{\partial \G_j}\nonumber\\
        &+\sum_{j=1}^K \IP{\frac{\partial u_0}{\partial\nu},h_{1j}}{\G_j}-\sum_{j=1}^K \IP{\frac{\partial w_0}{\partial\nu},h_{1j}}{\G_j}\nonumber\\
        &+\IP{\nabla w_1,\nabla w_0}{\Om_s}-\overline{\IP{\nabla w_1,\nabla w_0}{\Om_s}}-\IP{\frac{\partial w_0}{\partial \nu},w_1}{\G_s}.\label{bmeq:12}
    \end{align}
    (In the last expression, the authors are implicitly using the fact that the unit normal vector $\nu$ is $\textit{interior}$ with respect to $\G_s$.) Note now via domain criterion $(\textbf{A.iv})$, that for fixed index $j,\:1\leq j\leq K,$
    \begin{align*}
        \IP{\frac{\partial h_{0j}}{\partial n_j},h_{1j}}{\partial \G_j}=\sum_{\substack{1\leq l\leq K\\ \partial\G_j\cap\partial\G_l\neq\emptyset}}-\IP{\frac{\partial h_{0l}}{\partial n_l},h_{1l}}{\partial\G_j\cap\partial\G_l}
    \end{align*}
    Such relation gives then the inference
    \begin{align}
        \sum_{j=1}^K\IP{\frac{\partial h_{0j}}{\partial n_j},h_{1j}}{\partial\G_j}=0\label{bmeq:13}
    \end{align}
    Applying this relation and domain criterion $(\textbf{A.iii})$ to the inner product,
    \begin{align}
        \IP{\A\Phi_0,\Phi_0}{\BH}&= -\norm{\nabla u_0}{\Om_f}^2+2i\sum_{j=1}^K \Imag \IP{\nabla h_{1j},\nabla h_{0j}}{\G_j}\nonumber\\
        &+2i\sum_{j=1}^K \Imag \IP{h_{1j},h_{0j}}{\G_j}+2i\Imag \IP{\nabla w_1,\nabla w_0}{\Om_s}\label{bmeq:14},
    \end{align}
    which gives
    \begin{align*}
        \Real \IP{\A\Phi_0,\Phi_0}{\BH}\leq 0.
    \end{align*}

    \subsection{Step 2 (The Maximality of A)} 
    Given parameter $\lambda>0$, suppose $\Phi=[u_0,h_{01},h_{11},\dots,h_{0K},h_{1K},w_0,w_t]\in D(\A)$ is a solution of the equation
    \begin{align}
        (\lambda I-A)\Phi=\Phi^*,\label{bmeq:15}
    \end{align}
    where $\Phi^*=[u_0^*,h_{01}^*,h_{11}^*,\dots,h_{0K}^*,h_{1K}^*,w_0^*,w_t^*]\in\BH$. Then in PDE terms, the abstract equation above becomes
    \begin{align}
        \begin{cases}
            \lambda u_0-\Delta u_0=u_0^*\text{ in }\Om_f\\
            u_0|_{\G_f}=0\text{ on } \G_f\label{bmeq:16};
        \end{cases}
    \end{align}
    and for $1\leq j\leq K$,
    \begin{align}
        \begin{cases}
            \lambda h_{0j}-h_{1j}=h_{0j}^*\text{ in }\G_j\\
            \lambda h_{1j}-\Delta h_{0j}+h_{0j}-\frac{\partial w_0}{\partial\nu}+\frac{\partial u_0}{\partial\nu}=h_{1j}^*\text{ in } \G_j\\
            u_0|_{\G_j}=h_{1j}=w_1|_{\G_j}\text{ in }\G_j\\
            h_{0j}|_{\partial\G_j\cap\partial\G_l}=h_{0l}|_{\partial\G_j\cap\partial\G_l}\text{ on }\partial\G_j\cap\partial\G_l,\text{ for all }1\leq l\leq K\text{ such that }\partial\G_j\cap\partial\G_l\neq \emptyset\\
            \frac{\partial h_{0j}}{\partial n_j}\bigg |_{\partial\G_j\cap\partial\G_l}=-\frac{\partial h_{0l}}{\partial n_l}\bigg |_{\partial\G_j\cap\partial\G_l}\text{ on }\partial\G_j\cap\partial\G_l,\text{ for all }1\leq l\leq K\text{ such that }\partial\G_j\cap\partial\G_l\neq \emptyset;\label{bmeq:17}
        \end{cases}
    \end{align}
    and also
    \begin{align}
        \begin{cases}
            \lambda w_0-w_1=w_0^*\text{ in }\Om_s\\
            \lambda w_1-\Delta w_0=w_1^*\text{ in }\Om_s.\label{bmeq:18}
        \end{cases}
    \end{align}
    With respect to this static PDE system, by multiplying the heat equation above by a test function $\varphi\in H_{\G_f}^1(\Om_f)$, where
    \begin{align*}
        H_{\G_f}^1(\Om_f)=\set{\zeta \in\HM{1}{\Om_f}}{\zeta |_{\G_f}=0}.
    \end{align*}
    Upon integrating and invoking Green's Theorem, the solution component $u_0$ satisfies the variational relation,
    \begin{align}
        \lambda\IP{u_0,\varphi}{\Om_f}+\IP{\nabla u_0,\nabla\varphi}{\Om_f}-\IP{\frac{\partial u_0}{\partial\nu},\varphi}{\G_s}=\IP{u_0^*,\varphi}{\Om_f}\text{ for } \varphi\in H_{\G_f}^1(\Om_f).\label{bmeq:19}
    \end{align}
    In addition, define the Hilbert space $\mathcal{V}$ by
    \begin{align}
        \mathcal{V}=&\{[\psi_1,\dots,\psi_K]\in\HM{1}{\G_1}\times\dots\times\HM{1}{\G_k}|\text{ For all } 1\leq j\leq K,\nonumber\\
        &\psi_j|_{\partial\G_j\cap\partial\G_l}=\psi_l|_{\partial\G_j\cap\partial\G_l},\text{ for all }1\leq l\leq K\text{ such that }\partial\G_j\cap\partial\G_l\neq\emptyset\}.\label{bmeq:20}
    \end{align}
    Therewith, by multiplying both sides of the $h_{0j}$-wave equation in $(\ref{bmeq:17})$ by component $\psi_j$ of $\psi\in\mathcal{V}$, for $1\leq j\leq K$. Upon integration, for $\psi\in\mathcal{V}$,
    \begin{align*}
        &\begin{bmatrix}
            \lambda\IP{h_{11},\psi_1}{\G_1}-\IP{\Delta h_{01},\psi_1}{\G_1}+\IP{h_{01},\psi_1}{\G_1}-\IP{\frac{\partial}{\partial\nu}w_0,\psi_1}{\G_1}+\IP{\frac{\partial}{\partial\nu}u_0,\psi_1}{\G_1}\\
            \vdots\\
            \lambda\IP{h_{1K},\psi_K}{\G_K}-\IP{\Delta h_{0K},\psi_K}{\G_K}+\IP{h_{0K},\psi_K}{\G_K}-\IP{\frac{\partial}{\partial\nu}w_0,\psi_K}{\G_K}+\IP{\frac{\partial}{\partial\nu}u_0,\psi_K}{\G_K}
        \end{bmatrix}\\
        =&\begin{bmatrix}
        \IP{h_{11}^*,\psi_1}{\G_1}\\
        \vdots\\
        \IP{h_{1K}^*,\psi_K}{\G_K}
        \end{bmatrix}.
    \end{align*}
    For each vector component, it is subsequently integrated by parts while invoking the resolvent relations in $(\ref{bmeq:17})$ (and using the domain criterion (\textbf{A.iv.b})). Summing up the components of the resulting vectors, the solution components $[h_{11},\dots,h_{1K}]\in\mathcal{V}$ of $(\ref{bmeq:15})$ satisfy
    \begin{align}
        \sum_{j=1}^K\left [\lambda \IP{h_{1j},\psi_j}{\G_j}+\frac{1}{\lambda}\IP{\nabla h_{1j},\nabla\psi_j}{\G_j}+\frac{1}{\lambda}\IP{h_{1j},\psi_j}{\G_j}+\IP{\frac{\partial}{\partial \nu} u_0-\frac{\partial}{\partial \nu} w_0,\psi_j}{\G_j}\right ]\nonumber\\
        =\sum_{j=1}^K\left [\IP{h_{1j}^*,\psi_j}{\G_j}-\frac{1}{\lambda}\IP{\nabla h_{0j}^*,\nabla\psi_j}{\G_j}+\frac{1}{\lambda}\IP{h_{0j}^*,\psi_j}{\G_j}\right ],\text{ for }\psi\in\mathcal{V}.\label{bmeq:21}
    \end{align}
    Moreover, multiplying the both sides of the wave equation in $(\ref{bmeq:18})$ by $\xi\in\HM{1}{\Om_s}$, and integrating by parts - while using the resolvent relations in $(\ref{bmeq:18})$ - the authors note that the solution component $w_1$ of $(\ref{bmeq:15})$ satisfies
    \begin{align}
        \lambda \IP{w_1,\xi}{\Om_s}+\frac{1}{\lambda}\IP{\nabla w_1,\nabla\xi}{\Om_s}+\IP{\frac{\partial}{\partial\nu}w_0,\xi}{\G_s}=\IP{w_1^*,\xi}{\Om_s}-\frac{1}{\lambda}\IP{\nabla w_0^*,\nabla\xi}{\Om_s},\text{ for }\xi\in\HM{1}{\Om_s}\label{bmeq:22}
    \end{align}
    Set now
    \begin{align}
        \BW\equiv \set{[\varphi,\psi_1,\dots,\psi_K,\xi]\in H_{\G_f}^1(\Om_f)\times\mathcal{V}\times\HM{1}{\Om_s}}{\varphi|_{\G_j}=\psi_j=\xi|_{\G_j},\text{ for }1\leq j\leq K};\nonumber\\
        \norm{[\varphi,\psi_1,\dots,\psi_K,\xi]}{\BW}^2=\norm{\nabla \varphi}{\Om_f}^2+\sum_{j=1}^K\left [\norm{\nabla \psi_j}{\G_j}^2+\norm{\psi_j}{\G_j}^2 \right ]+\norm{\nabla\xi}{\Om_f}^2.\label{bmeq:23}
    \end{align}
    With respect to this Hilbert space, upon adding (\ref{bmeq:19}), (\ref{bmeq:21}), and (\ref{bmeq:22}) if $\Phi=[u_0,h_{01},h_{11},\dots,h_{0K},$\\ $h_{1K},w_0,w_1]\in D(\A)$ solves (\ref{bmeq:15}), then necessarily its solution components $[u_0,h_{11},\dots,h_{1K},w_1]\in \BW$ satisfy for $[\varphi,\psi,\xi]\in\BH$,
    \begin{align}
        \lambda \IP{u_0,\varphi}{\Om_f}+\IP{\nabla u_0,\nabla\varphi}{\Om_f}+\lambda\IP{w_1,\xi}{\Om_s}+\frac{1}{\lambda}\IP{\nabla w_1,\nabla\xi}{\Om_s}\nonumber\\
        +\sum_{j=1}^K[\lambda\IP{h_1,\psi_j}{\G_j}+\frac{1}{\lambda}\IP{\nabla h_{1j},\nabla \psi_j}{\G_j}+\frac{1}{\lambda}\IP{h_{1j},\psi_j}{\G_j}]=\textbf{F}_\lambda\left (\begin{bmatrix}\varphi\\ \psi\\ \xi\end{bmatrix}\right );\label{bmeq:24}
    \end{align}
    where
    \begin{align}
        \textbf{F}_\lambda\left (\begin{bmatrix}\varphi\\ \psi\\ \xi\end{bmatrix}\right )&=\IP{u_0^*,\varphi}{\Om_f}+\sum_{j=1}^K[\IP{h_{1j}^*,\psi_j}{\G_j}-\frac{1}{\lambda}\IP{\nabla h_{0j}^*,\nabla \psi_j}{\G_j}\nonumber\\
        &-\frac{1}{\lambda}\IP{h_{0j}^*, \psi_j}{\G_j}]+\IP{w_1^*,\xi}{\Om_s}-\frac{1}{\lambda}\IP{\nabla w_0^*,\nabla\xi}{\Om_s}\label{bmeq:25}
    \end{align}
    In sum, in order to recover the solution $\Phi=[u_0,h_{01},h_{11},\dots,h_{0K},h_{1K},w_0,w_1]\in D(\A)$ to $(\ref{bmeq:15})$, one can straightaway apply the Lax-Milgram Theorem (see Theorem $\ref{Lax}$) to the operator $\textbf{B}\in\Lin (\BW,\BW^*)$, given by
    \begin{align*}
        \IP{\textbf{B}
        \begin{bmatrix}
        \varphi\\\psi_1\\\vdots\\\psi_K\\\xi
        \end{bmatrix},\begin{bmatrix}
        \widetilde{\varphi}\\\widetilde{\psi}_1\\\vdots\\\widetilde{\psi}_K\\ \widetilde{\xi}
        \end{bmatrix}
        }{\BW^*\times\BW}&=\lambda\IP{\varphi,\widetilde{\varphi}}{\Om_f}+\IP{\nabla \varphi,\nabla \widetilde{\varphi}}{\Om_f}+\lambda\IP{\xi,\widetilde{\xi}}{\Om_s}+\IP{\nabla \xi,\nabla\widetilde{\xi}}{\Om_s}\\
        &=\sum_{j=1}^K\left [\lambda \IP{\psi_j,\widetilde{\psi}_j}{\G_j}+\frac{1}{\lambda}\IP{\nabla\psi_j,\nabla\widetilde{\psi}_j}{\G_j}+\frac{1}{\lambda}\IP{\psi_j,\widetilde{\psi}_j}{\G_j}\right ]
    \end{align*}
    It is clear that $\textbf{B}\in\Lin(\BW,\BW^*)$ is $\BW$-elliptic; so by the Lax-Milgram Theorem (see Theorem $\ref{Lax}$), the equation $(\ref{bmeq:24})$ has a unique solution
    \begin{align}
        [u_0,h_{11},\dots,h_{1K},w_1]\in\BW.\label{bmeq:26}
    \end{align}
    Subsequently, set
    \begin{align}
        \begin{cases}
            h_{0j}=\frac{h_{1j}+h_{0j}^*}{\lambda},\text{ for }1\leq j\leq K,\\
            w_0=\frac{w_1+w_0^*}{\lambda}.\label{bmeq:27}
        \end{cases}
    \end{align}
    In particular, since the data $[u_0^*,h_{01}^*,h_{11}^*,\dots,h_{0K}^*,h_{1K}^*,w_0^*,w_1^*]\in\BH$, then the relations in $(\ref{bmeq:27})$ gives that
    \begin{align}
        w_0|_{\G_j}=h_{0j},\:1\leq j\leq K.\label{bmeq:28}
    \end{align}
    Further the authors show that the dependent variable $\Phi=[u_0,h_{01},h_{11},\dots,h_{0K},h_{1K},w_0,w_1]$, given by the solution of $(\ref{bmeq:24})$ and $(\ref{bmeq:27})$, is an element of $D(\A)$: If $[\varphi,0,\dots,0,0]\in \BW$ in $(\ref{bmeq:24})$, where $\varphi\in\D(\Om_f)$, then
    \begin{align*}
        \lambda\IP{u_0,\varphi}{\Om_f}-\IP{\Delta u_0,\varphi}{\Om_f}=\IP{u_0^*,\varphi}{\Om_f}\text{ for all }\varphi\in\D(\Om_f)
    \end{align*}
    whence
    \begin{align}
        \lambda u_0-\Delta u_0=u_0^*\text{ in }\LP{2}{\Om_f}.\label{bmeq:29}
    \end{align}
    Subsequently, the fact that $\{\Delta u_0,u_0\}\in\LP{2}{\Om_f}\times\HM{1}{\Om_F}$ gives
    \begin{align}
        \frac{\partial u_0}{\partial\nu}\bigg |_{\G_s}\in\HM{-1/2}{\G_s}\label{bmeq:30}
    \end{align}
    In turn, using the relations in $(\ref{bmeq:27})$, if $[0,0,\dots,0,\xi]\in\BW$, where $\xi\in\D(\Om_s)$, then upon integrating by parts,
    \begin{align*}
        \lambda\IP{w_1,\xi}{\Om_s}-\IP{\Delta w_0,\xi}{\Om_s}=\IP{w_1^*,\xi}{\Om_s}\text{ for all }\xi\in\D(\Om_s),
    \end{align*}
    and so
    \begin{align}
        \lambda w_1-\Delta w_0=w_1^*\text{ in }\LP{2}{\Om_s}\label{bmeq:31}
    \end{align}
    which gives that $\{\Delta w_0,w_0\}\in\LP{2}{\Om_s}\times\HM{1}{\Om_s}$. A subsequent integration by parts yields that
    \begin{align}
        \frac{\partial w_0}{\partial\nu}\bigg |_{\G_s}\in\HM{-1/2}{\G_s}.\label{bmeq:32}
    \end{align}
    Moreover, let $\gamma_s^+\in\Lin(\HM{1/2}{\G_s},\HM{1}{\Om_s})$ be the right continuous inverse for the Sobolev trace map $\gamma_s\in\Lin(\HM{1}{\Om_s},\HM{1/2}{\G_s});$ viz.,
    \begin{align*}
        \gamma_s(f)=f|_{\G_s}\text{ for } f\in C^\infty (\overline{\Om}_s).
    \end{align*}
    Likewise, let $\gamma_f^+\in\Lin(\HM{1/2}{\G_s},H_{\G_f}^1(\Om_f))$ denote the right inverse for the Sobolev trace map $\gamma_f\in\Lin(H_{\G_f}^1(\Om_f),\HM{1/2}{\G_s})$. Also, for given $\psi_j\in H_0^1(\G_j),\:1\leq j\leq K$, let
    \begin{align}
        (\psi_j)_{\text{ext}}(x)\equiv\begin{cases}
        \psi_j,\:x\in\G_j\\
        0,\: x\in\G_s\backslash\G_j
        \end{cases}.\label{bmeq:33}
    \end{align}
    Then $(\psi_j)_{\text{ext}}\in\HM{1/2}{\G_s}$ for all $1\leq j\leq K$. Specifying the test functions in $(\ref{bmeq:24})$, $[\varphi,\psi_1,\dots,\psi_K,\xi]\in\BW$: namely, $\psi_j\in H_0^1(\G_j),\:1\leq j\leq K$, and
    \begin{align}
        \varphi\equiv\gamma_f^+\left [\sum_{j=1}^K(\psi_j)_{\text{ext}}\right ],\qquad \xi\equiv\gamma_s^+\left [\sum_{j=1}^K(\psi_j)_{\text{ext}}\right ].\label{bmeq:34}
    \end{align}
    Therewith, from $(\ref{bmeq:24})$
    \begin{align*}
        &\lambda\IP{u_0,\varphi}{\Om_f}+\IP{\nabla u_0,\nabla\varphi}{\Om_f}\\
        &+\sum_{j=1}^K\left [\lambda \IP{h_{1j},\psi_j}{\G_j}+\frac{1}{\lambda}\IP{\nabla h_{1j},\nabla \psi_j}{\G_j}+\frac{1}{\lambda}\IP{h_{1j},\psi_j}{\G_j}\right ]\\
        &+\lambda\IP{w_1,\xi}{\Om_s}+\frac{1}{\lambda}\IP{\nabla w_1,\nabla \xi}{\Om_s}\\
        =&\IP{u_0^*,\varphi}{\Om_f}+\sum_{j=1}^K\left [\IP{h_{1j}^*,\psi_j}{\G_j}-\frac{1}{\lambda}\IP{\nabla h_{0j}^*,\nabla \psi_j}{\G_j}-\frac{1}{\lambda}\IP{h_{0j}^*, \psi_j}{\G_j}\right ]\\
        &+\IP{w_1^*,\xi}{\Om_s}-\frac{1}{\lambda}\IP{\nabla w_0^*,\nabla\xi}{\Om_s}
    \end{align*}
    Upon integrating by parts, and invoking the relations in $(\ref{bmeq:27})$, as well as $(\ref{bmeq:29})$-$(\ref{bmeq:32})$,
    \begin{align}
        \IP{ \frac{\partial u_0}{\partial\nu},\varphi}{\G_s}+\sum_{j=1}^K\left [\lambda \IP{h_{1j},\psi_j}{\G_j}-\frac{1}{\lambda}\IP{\Delta h_{0j},\psi_j}{\G_j}+\frac{1}{\lambda}\IP{h_{0j},\psi_j}{\G_j}\right ]-\IP{ \frac{\partial w_0}{\partial\nu},\xi}{\G_s}=\sum_{j=1}^K\IP{h_{1j}^*,\psi_j}{\G_j}.\label{bmeq:35}
    \end{align}
    Since each test function component $\psi_j\in H_0^1(\G_j)$ is arbitrary, from this relation and $(\ref{bmeq:33})$-$(\ref{bmeq:34})$ that each $h_{0j}$ solves
    \begin{align}
        \lambda h_{1j}-\Delta h_{0j}+h_{0j}-\frac{\partial w_0}{\partial \nu}+\frac{\partial u_0}{\partial \nu}=h_{1j}^*\text{ in }\G_j,\:1\leq j\leq K.\label{bmeq:36}
    \end{align}
    In addition, from $(\ref{bmeq:36})$, $(\ref{bmeq:26})$, $(\ref{bmeq:30})$, and $(\ref{bmeq:32})$ that $\{\Delta h_{0j},h_{0j}\}\in [\HM{1}{\G_j}]'\times\HM{1}{\G_j},\text{ for } 1\leq j\leq K.$ Consequently, an integration by parts gives that
    \begin{align}
        \frac{\partial h_{0j}}{\partial n_j}\in\HM{-1/2}{\partial\G_j},\text{ for }1\leq j\leq K.\label{bmeq:37}
    \end{align}
    Finally: Let given indices $j^*,l^*,1\leq j^*,l^*\leq K$, satisfy $\partial\G_{j^*}\cap\partial\G_{l^*}\neq \emptyset$. Let $g$ be a given element in $H_0^{1/2+\varepsilon}(\partial\G_{j^*}\cap\partial\G_{l^*})$. Then one has that $\widetilde{g}_{j^*}\in\HM{1/2+\varepsilon}{\partial\G_{j^*}}$ and $\widetilde{g}_{l^*}\in\HM{1/2+\varepsilon}{\partial\G_{l^*}}$, where 
    \begin{align*}
        \widetilde{g}_{j^*}(x)\equiv\begin{cases}
        g(x),\:x\in \partial\G_{j^*}\cap\partial\G_{l^*}\\
        0,\:x\in\partial\G_{j^*}\backslash(\partial\G_{j^*}\cap\partial\G_{l^*});
        \end{cases}\qquad
        \widetilde{g}_{l^*}(x)\equiv\begin{cases}
        g(x),\:x\in \partial\G_{j^*}\cap\partial\G_{l^*}\\
        0,\:x\in\partial\G_{l^*}\backslash(\partial\G_{j^*}\cap\partial\G_{l^*});
        \end{cases}
    \end{align*}
    (see e.g., Theorem 3.33, p. 95 of \cite{mclean}). Subsequently, by the (limited) surjectivity of the Sobolev trace map on Lipschitz domains - see e.g., Theorem 3.38, p. 102 of \cite{mclean} there exists $\psi_{j^*}\in \HM{1+\varepsilon}{\G_{j^*}}$ and $\psi_{l^*}\in \HM{1+\varepsilon}{\G_{l^*}}$ such that
    \begin{align}
        \psi_{j^*}|_{\partial\G_{j^*}}=\widetilde{g}_{j^*}\text{ and }\psi_{l^*}|_{\partial\G_{l^*}}=\widetilde{g}_{l^*}.\label{bmeq:38}
    \end{align}
    In turn, by the Sobolev Embedding Theorem, define on $\overline{\G}_s$ the function
    \begin{align}
        \Upsilon (x)\equiv\begin{cases}
            \psi_{j^*}(x),\text{ for }x\in \overline{\G}_{j^*}\\
            \psi_{l^*}(x),\text{ for }x\in \overline{\G}_{l^*}\\
            0,\text{ for }x\in \overline{\G}_s\backslash(\overline{\G}_{j^*}\cup\overline{\G}_{l^*}),\label{bmeq:39}
        \end{cases}
    \end{align}
    then $\Upsilon(x)\in C(\overline{\G}_s)$. Since also $\psi_{j^*}\in\HM{1}{\G_{j^*}}$ and $\psi_{l^*}\in\HM{1}{\G_{l^*}}$, the authors eventually deduce via an integration by parts that $\Upsilon\in\HM{1}{\G_s}$ (See e.g., the proof of Theorem 2, p. 36 of \cite{ciarlet1}). With this $H^1$-function in hand, and with aforesaid continuous right inverses $\gamma_s^+\in\Lin(\HM{1/2}{\G_s},\HM{1}{\Om_s})$ and $\gamma_f^+\in\Lin(\HM{1/2}{\G_s},H_{\G_f}^1(\Om_f))$, specify the vector
    \begin{align}
        [\varphi,\psi,\xi]\equiv \left [\gamma_f^+(\Upsilon),0,\dots,\psi_{j^*},0,\dots,\psi_{l^*},\dots,0,\gamma_s^+(\Upsilon)\right ]\in\BW,\label{bmeq:40}
    \end{align}
    where again, space $\BW$ is given in $(\ref{bmeq:23})$. With this vector in hand, consider the thin wave equation in $(\ref{bmeq:36})$: With respect to the two fixed indices $1\leq j^*,l^*\leq K$, it follows via $(\ref{bmeq:36})$
    \begin{align*}
        &\lambda \IP{h_{1j^*},\psi_{j^*}}{\G_{j^*}}-\IP{\Delta h_{0j^*},\psi_{j^*}}{\G_{j^*}}+\IP{h_{0j^*},\psi_{j^*}}{\G_{j^*}}\\
        &-\IP{\frac{\partial w_0}{\partial\nu}-\frac{\partial u_0}{\partial v},\psi_{j^*}}{\G_{j^*}}+\lambda \IP{h_{1l^*},\psi_{l^*}}{\G_{l^*}}-\IP{\Delta h_{0l^*},\psi_{l^*}}{\G_{l^*}}\\
        &+\IP{h_{0l^*},\psi_{l^*}}{\G_{l^*}}-\IP{\frac{\partial w_0}{\partial\nu}-\frac{\partial u_0}{\partial v},\psi_{l^*}}{\G_{l^*}}=\IP{h_{1j^*}^*,\psi_{j^*}}{\G_{j^*}}+\IP{h_{1l^*}^*,\psi_{l^*}}{\G_{l^*}}.
    \end{align*}
    A subsequent integration by parts, with $(\ref{bmeq:40})$ in mind, subsequently yields
    \begin{align*}
        &\lambda \IP{h_{1j^*},\psi_{j^*}}{\G_{j^*}}+\IP{\nabla h_{0j^*},\nabla\psi_{j^*}}{\G_{j^*}}-\IP{ \frac{\partial h_{0j^*}}{\partial n_{j^*}},g}{\partial\G_{j^*}\cap\partial\G_{l^*}}+\IP{h_{0j^*},\psi_{j^*}}{\G_{j^*}}\\
        &+\lambda \IP{h_{1l^*},\psi_{l^*}}{\G_{l^*}}+\IP{\nabla h_{0l^*},\nabla\psi_{l^*}}{\G_{l^*}}-\IP{ \frac{\partial h_{0l^*}}{\partial n_{l^*}},g}{\partial\G_{j^*}\cap\partial\G_{l^*}}+\IP{h_{0l^*},\psi_{l^*}}{\G_{l^*}}\\
        &+\IP{\nabla w_0,\nabla\xi}{\Om_s}+\IP{\Delta w_0,\xi}{\Om_s}+\IP{\nabla u_0,\nabla\varphi}{\Om_f}+\IP{\Delta u_0,\varphi}{\Om_f}\\
        &=\IP{h_{1j^*}^*,\psi_{j^*}}{\G_{j^*}}+\IP{h_{1l^*}^*,\psi_{l^*}}{\G_{l^*}}
    \end{align*}
    Invoking $(\ref{bmeq:29})$ and $(\ref{bmeq:31})$, it follows
    \begin{align*}
        &-\IP{ \frac{\partial h_{0j}^*}{\partial n_{j^*}},g}{\partial\G_{j^*}\cap\partial\G_{l^*}}-\IP{ \frac{\partial h_{0l}^*}{\partial n_{l^*}},g}{\partial\G_{j^*}\cap\partial\G_{l^*}}+\IP{h_{0j^*},\psi_{j^*}}{\G_{j^*}}+\IP{h_{0l^*},\psi_{l^*}}{\G_{l^*}}\\
        &+\lambda \IP{h_{1j^*},\psi_{j^*}}{\G_{j^*}}+\IP{\nabla h_{0j^*},\nabla \psi_{j^*}}{\G_{j^*}}+\lambda \IP{h_{1l^*},\psi_{l^*}}{\G_{l^*}}+\IP{\nabla h_{0l^*},\nabla \psi_{l^*}}{\G_{l^*}}\\
        &+\IP{\nabla w_0,\nabla \xi}{\Om_s}+\lambda \IP{w_1,\xi}{\Om_s}-\IP{w_1^*,\xi}{\Om_s}+\IP{\nabla u_0,\nabla \varphi}{\Om_f}+\lambda \IP{u_0,\varphi}{\Om_f}-\IP{u_0^*,\varphi}{\Om_f}\\
        &=\IP{h_{1j^*}^*,\psi_{j^*}}{\G_{j^*}}+\IP{h_{1l^*}^*,\psi_{l^*}}{\G_{l^*}}
    \end{align*}
    Invoking the relations in $(\ref{bmeq:27})$ and the variational equation $(\ref{bmeq:24})$, which is satisfied by $[u_0,h_{11},\dots,h_{1K},w_1]$ (where again vector $[\varphi,\psi,\xi]$ is given by $(\ref{bmeq:40})$), it follows
    \begin{align*}
        \IP{ \frac{\partial h_{0j^*}}{\partial n_{j^*}},g}{\partial\G_{j^*}\cap\partial\G_{l^*}}=-\IP{ \frac{\partial h_{0l^*}}{\partial n_{l^*}},g}{\partial\G_{j^*}\cap\partial\G_{l^*}},\text{ for all } g\in H_0^{1/2+\varepsilon}(\partial\G_{j^*}\cap\partial\G_{l^*}).
    \end{align*}
    Since $H_0^{1/2+\varepsilon}(\partial\G_{j^*}\cap\partial\G_{l^*})$ is dense in $\HM{1/2+\varepsilon}{\partial\G_{j^*}\cap\partial\G_{l^*}}$, the authors then deduce that
    \begin{align}
        \frac{\partial h_{0j^*}}{\partial n_{j^*}}=-\frac{\partial h_{0l^*}}{\partial n_{l^*}},\text{ for }\partial\G_{j^*}\cap\partial\G_{l^*}\neq \emptyset.\label{bmeq:41}
    \end{align}
    Collecting $(\ref{bmeq:26})$-$(\ref{bmeq:32})$ and $(\ref{bmeq:36})$, $(\ref{bmeq:37})$, and $(\ref{bmeq:41})$, the authors obtained the variable
    \begin{align*}
        [u_0,h_{01},h_{11},\dots,h_{0K},h_{1K},w_0,w_1]\in D(\A),
    \end{align*}
    and solves the resolvent equation $(\ref{bmeq:15})$. This concludes the proof of Theorem $\ref{bmthrm:1}$, upon application of the Lumer-Phillips Theorem (see Theorem $\ref{LP}$).

\section{Stokes Wave Lam\'e Multilayered FSI PDE Model}
    \begin{figure}
        \begin{center}
            \includegraphics[width=0.5\textwidth]{Screenshot 2025-05-01 075848.png}
            \caption{Smooth domain FSI model}
        \end{center}
    \end{figure}
    Now we start to introduce the second improved model. One of the main differences being the pressure term being added to first PDE in the system. It follows,
    \begin{align}
        &\begin{cases}
            u_t-\diver (\nabla u+\nabla^T u)+\nabla p=0 &\text{in } (0,T)\times \Om_f\\
            \diver(u)=0 &\text{in } (0,T)\times \Om_f\\
            u|_{\G_f}=0 &\text{on } (0,T)\times\G_f;
        \end{cases}\label{sweq:1}\\
        &\begin{cases}
        h_{tt}-\Delta_{\G_s}h=[\nu\cdot\sigma(w)]|_{\G_s}-[\nu\cdot(\nabla u+\nabla ^T u)]|_{\G_s}+p\nu\quad\text{on }(0,T)\times\G_s,
        \end{cases}\label{sweq:2}\\
        &\begin{cases}
            w_{tt}-\diver \sigma(w)+w=0&\text{ on } (0,T)\times\Om_s\\
            w_t|_{\G_s}=h_t=u|_{\G_s}&\text{ on } (0,T)\times\G_s
        \end{cases}\label{sweq:3}\\
        &[u(0),h(0),h_t(0),w(0),w_t(0)]=[u_0,h_0,h_1,w_0,w_1]\in \BH\label{sweq:4}
    \end{align}
    Here, $\Delta_{\G_s}(\cdot)$ is the Laplace Beltrami operator, and the stress tensor $\sigma(\cdot)$ constitutes the Lam\'e system of elasticity on the ``thick'' layer. Namely, for function $v$ in $\Om_s$,
    \begin{align*}
        \sigma(v)=2\mu\epsilon(v)+\lambda[I_3\cdot\epsilon(v)]I_3,
    \end{align*}
    where strain tensor $\epsilon(\cdot)$ is given by 
    \begin{align*}
        \epsilon_{ij}=\frac{1}{2}\bigg (\frac{\partial v_j}{\partial x_i}+\frac{\partial v_i}{\partial x_j}\bigg ),\quad 1\leq i,\:j\leq 3
    \end{align*}
    Also, $\BH$ is the finite energy space defined in $(\ref{sweq:5})$ below.\\
    \underline{\textbf{Remark 1}} For the sake of numerical computation, the structure geometry $\Om_s\subseteq \R^3$, can also be taken to be a convex polyhedral domain with polygonal boundary faces $\G_j$, $1\leq j\leq K$, where $\G_i\cap\G_j\neq \emptyset$ for $i\neq j$, and
    \begin{align*}
        \G_s=\cup_{j=1}^K \overline{\G_j}.
    \end{align*}
    In this case the thin wave equation can be modeled for $j=1,\dots,K$ as
    \begin{align*}
        \begin{cases}
            \frac{\partial^2}{\partial t^2}h_j-\Delta h_j=[\nu\cdot\sigma(w)]|_{\G_j}-[\nu\dot(\nabla u+\nabla ^T u)]|_{\G_j}+pv\text{ on } (0,T)\times \G_j\\
            h_j|_{\partial\G_j\cap\partial\G_l}=h_l|_{\partial\G_j\cap\partial\G_l}\text{ on } (0,T)\times (\partial\G_j\cap\partial\G_l),\text{ for all } 1\leq l\leq K\\
            \text{ such that } \partial\G_j\cap\partial\G_l\neq\emptyset\\
            \frac{\partial h_j}{\partial n_j}\bigg |_{\partial\G_j\cap\partial\G_l}=-\frac{\partial h_l}{\partial n_l}\bigg |_{\partial\G_j\cap\partial\G_l}\text{ on }(0,T)\times(\partial\G_j\cap\partial\G_l)\text{, for all }1\leq l\leq K\\
            \text{ such that }\partial\G_j\cap\partial\G_l\neq \emptyset
        \end{cases}
    \end{align*}
    where the Laplace Beltrami Operator $\Delta_{\G_s}(\cdot)$ in $(\ref{sweq:2})$ is replaced with the standard Laplace operator with the imposition of additional continuity and boundary conditions in order to satisfy the surface differentiation \cite{AGM, RD}.\\
    With respect to the PDE system given in $(\ref{sweq:1})-(\ref{sweq:4})$, the finite energy Hilbert space $\BH$ is given as
    \begin{align}
        \nonumber \BH= \{[u_0,h_0,h_1,w_0,w_1]\in\BFLP{2}{\Om_f}\times\BFHM{1}{\G_s}\times\BFLP{2}{\G_s}\times\BFHM{1}{\Om_s}\times\BFLP{2}{\Om_s}|\diver (u_0)=0,\\
        u_0\cdot\nu|_{\G_f}=0,\text{ and }w_0|_{\G_s}=h_0 \}\label{sweq:5}
    \end{align}
    with the inner product
    \begin{align}
        \IP{\Phi_0,\widetilde{\Phi}_0}{\BH}=\IP{u_0,\widetilde{u}_0}{\Om_f}+\IP{\nabla _{\G_s}(h_0),\nabla _{\G_s}(\widetilde{h}_0)}{\G_s}+\IP{h_1,\widetilde{h}_1}{\G_s}+\IP{\sigma(w_0),\epsilon(\widetilde{w}_0)}{\Om_s}\nonumber\\
        +\IP{w_0,\widetilde{w}_0}{\Om_s}+\IP{w_1,\widetilde{w}_1}{\Om_s}\label{sweq:6}
    \end{align}
    where
    \begin{align}
        \Phi_0=[u_0,h_0,h_1,w_0,w_1]\in\BH;\widetilde{\Phi}_0=[\widetilde{u}_0,\widetilde{h}_0,\widetilde{h}_1,\widetilde{w}_0,\widetilde{w}_1]\in\BH.\label{sweq:7}
    \end{align}

\section{Preliminaries}
    The PDE system given in $(\ref{sweq:1})-(\ref{sweq:4})$ may be associated with an abstract ODE in Hilbert space $\BH$; namely,
    \begin{align}
        \begin{cases}
            \frac{d}{dt}\Phi (t)=\A\Phi(t)\\
            \Phi(0)=\Phi_0\label{sweq:13}
        \end{cases}
    \end{align}
    where $\Phi(t)=[u(t),h(t),h_t(t),w(t),w_t(t)],\text{ and }\Phi_0=[u_0,h_0,h_1,w_0,w_1]$. Here, the operator $\A:D(\A)\subseteq \BH\to\BH$ is defined by
    \begin{align}
        \A=&\begin{bmatrix}
            \diver(\nabla (\cdot)+\nabla^T(\cdot))&0&0&0&0\\
            0&0&I&0&0\\
            -[\nu\cdot(\nabla(\cdot)+\nabla^T(\cdot))]|_{\G_s}&\Delta_{\G_s}(\cdot)&0&\nu\cdot\sigma(\cdot)|_{\G_s}&0\\
            0&0&0&0&I\\
            0&0&0&\diver\:\sigma(\cdot)-I&0
        \end{bmatrix};\nonumber\\
        &\begin{bmatrix}
            -\nabla \mathcal{P}_1(\cdot)&-\nabla \mathcal{P}_2(\cdot)&0&-\nabla \mathcal{P}_3(\cdot)&0\\
            0&0&0&0&0\\
            \mathcal{P}_1(\cdot)\nu&\mathcal{P}_2(\cdot)\nu&0&\mathcal{P}_3(\cdot)\nu&0\\
            0&0&0&0&0\\
            0&0&0&0&0\\
        \end{bmatrix}\label{sweq:14}
    \end{align}
    Here, the ``pressure'' operators $\mathcal{P}_i$ are as defined below. The domain $D(\A)$ of the generator $\A$ is characterized as follows $[u_0,h_0,h_1,w_0,w_1]\in D(\A) \Leftrightarrow$\\
    (\textbf{A.i}) $u_0\in\BFHM{1}{\Om_f},\:h_{1}\in\BFHM{1}{\G_s}\:w_1\in\BFHM{1}{\Om_s};$\\
    (\textbf{A.ii}) There exists an associated $\LP{2}{\Om_f}$-function $p_0=p_0(u_0,h_0,w_0)$ such that
    \begin{align*}
        [\diver(\nabla u_0+\nabla^T u_0)-\nabla p_0]\in\LP{2}{\Om_f}
    \end{align*}
    Consequently, $p_0$ is harmonic and so one has the boundary traces\\
    \textbf{(a)} $\left [p_0|_{\G_f},\frac{\partial p_0}{\partial\nu}\big |_{\G_f}\right ]\in\HM{-1/2}{\G_f}\times\HM{-3/2}{\G_f};$\\
    \textbf{(b)} $(\nabla u_0+\nabla^T u_0)\cdot\nu\in\HM{-3/2}{\G_f}$,\\
    (\textbf{A.iii}) $\diver\sigma(w_0)\in\LP{2}{\Om_s}$; consequently, $\nu\cdot\sigma\in\HM{-1/2}{\G_s}$,\\
    (\textbf{A.iv}) $\Delta_{\G_s}(h_0)+[\nu\cdot\sigma(w_0)]_{\G_s}-[(\nabla u_0+\nabla^T u_0)\cdot\nu]|_{\G_s}+[p_0\nu]|_{\G_s}\in\LP{2}{\G_s}$,\\
    (\textbf{A.v}) $u_0|_{\G_f}=0,\:u_0|_{\G_s}=h_1=w_1|_{\G_s}$\\
    Moreover, the PDE system satisfies the following energy relation:
    \begin{align*}
        \mathcal{E}(T)-\int_0^T\int_{\Om_f}\norm{\nabla u+\nabla^T u}{}^2=\mathcal{E}(0),
    \end{align*}
    where
    \begin{align*}
        \mathcal{E}(T)=\norm{u}{}^2+\norm{\nabla_{\G_s}h}{}^2+\norm{h_t}{}^2+\norm{w}{}^2+\IP{ \sigma(w),\epsilon(w)}{\Om_s}+\norm{w_t}{}^2.
    \end{align*}\\
    It is important to note that the elimination of the pressure variable is very crucial in order to formulate the PDE system as an ODE problem. For this, the author basically applies the divergence operator to the Stokes equation and use the fact that $u$ is solenoidal. This gives that the (pointwise) pressure variable $p(t)$ is harmonic; i.e.,
    \begin{align}
        \Delta p(t)=0\text{ in }\Om_f.\label{sweq:8}
    \end{align}
    Subsequently, by multiplying by $\nu|_{\G_s}$ and use the matching velocity condition to obtain the following boundary condition for the pressure variable $p$:
    \begin{align}
        p+\frac{\partial p}{\partial \nu}=\diver(\nabla(u)+\nabla^T(u))\cdot\nu|_{\G_s}+[(\nabla u+\nabla^T u)\cdot\nu-\Delta_{\G_s}(h)-\nu\cdot\sigma(w)|_{\G_s}]\cdot\nu|_{\G_s}\label{sweq:9}
    \end{align}
    Also, since $u$ is divergence free, by taking the inner product of both sides the $u$ equation, with an extension of the normal vector, and subsequently take the trace of this relation on $\G_f$,
    \begin{align*}
        \frac{\partial p}{\partial\nu}=[\diver(\nabla u+\nabla^T u)]\cdot\nu\text{ on }\G_f.
    \end{align*}
    Accordingly, the pressure variable $p(t)$, as the solution to the last two equations, can formally be written pointwise in time as 
    \begin{align*}
        p(t)=\mathcal{P}_1(u(t))+\mathcal{P}_2(h(t))+\mathcal{P}_3(w(t))
    \end{align*}
    where the harmonic functions $\mathcal{P}_1(u(t)),\:\mathcal{P}_2(h(t)),\text{ and }\mathcal{P}_3(w(t))$ solve the following elliptic BVPs:
    \begin{align}
        \begin{cases}
            \Delta\mathcal{P}_1(u)=0 &\text{in }\Om_f,\\
            \mathcal{P}_1(u)=\diver(\nabla(u)+\nabla^T(u))\cdot\nu|_{\G_s}+([(\nabla u+\nabla^T u)]\cdot \nu)\cdot \nu|_{\G_s} &\text{on }\G_s,\\
            \frac{\partial\mathcal{P}_1(u)}{\partial\nu}=\diver(\nabla(u)+\nabla^T(u))\cdot\nu|_{\G_f} &\text{on }\G_f,
        \end{cases}\label{sweq:10}
    \end{align}
    \begin{align}
       \begin{cases}
            \Delta\mathcal{P}_2(h)=0&\text{in }\Om_f,\\
            \mathcal{P}_2(h)=-\Delta_{\G_s}(h)\cdot\nu|_{\G_s}&\text{on }\G_s,\\
            \frac{\partial\mathcal{P}_2(h)}{\partial\nu}=0&\text{on }\G_f,
       \end{cases}\label{sweq:11}
    \end{align}
    and
    \begin{align}
       \begin{cases}
            \Delta\mathcal{P}_3(w)=0&\text{in }\Om_f,\\
            \mathcal{P}_3(w)=-[\nu\cdot\sigma(w)|_{\G_s}]\cdot\nu|_{\G_s}&\text{on }\G_s,\\
            \frac{\partial\mathcal{P}_3(w)}{\partial\nu}=0&\text{on }\G_f.
       \end{cases}\label{sweq:12} 
    \end{align}
    The construction of these $\mathcal{P}_i$ functions, defined as the solutions to above harmonic equations, allows for the elimination of the pressure term in the original system. As such, the pressure-free system can indeed be associated with the abstract ODE in Hilbert space $\BH$, and the associated pressure function $p_0$ in $\textbf{(A.ii)}$ can be identified explicitly, via 
    \begin{align}
        p_0=\mathcal{P}_1(u_0)+\mathcal{P}_2(h_0)+\mathcal{P}_3(w_0).\label{sweq:15}
    \end{align}
\section{Main Result: Existence-Uniqueness of Solution}
    The main result of this section is to show that the system $(\ref{sweq:1})-(\ref{sweq:4})$ or equivalently the abstract ODE system $(\ref{sweq:13})$ may be associated with a $C_0$-semigroup $\{e^{\A t}\}_{t\geq 0}$, where $\A:D(\A)\subseteq\BH\to\BH$ is the matrix operator defined in $(\ref{sweq:14})$. To this end, the author constructs a mixed variational formulation which is necessarily predicated on the ``thick'' and ``thin'' structural PDE components. This is quite different than the inf-sup formulations which have been derived for uncoupled Stokes flow; see e.g., \cite{BF}. There is no choice in the matter: the proper mixed variational formulation must be driven here by the structural PDE components, although the associated bilinear form, $a(\cdot,\cdot)$, written in Theorem $\ref{BB}$ below - necessarily takes into account the presence in $(\ref{sweq:1})-(\ref{sweq:4})$ of Stokes flow. Ultimately, the author will arrive at an inf-sup system of the classical form $(\ref{sweq:16})$, for which they will apply the theorem.\\
    \indent This inf-sup result will be invoked below to recover the ``thick'' and ``thin'' structural variables $[h_0,h_1,w_0,w_1]$ of the solution of the abstract resolvent Eq $(\ref{sweq:18})$ below, which is formally a frequency domain version of the time dependent system $(\ref{sweq:1})-(\ref{sweq:4})$. Subsequently, the author will reconstruct (from $[h_1,w_1]$) the fluid and pressure variables $\{u_0,p_0\}$ of the solution to $(\ref{sweq:18})$ and moreover show that this fluid-structure interaction solution is in $D(\A)$, where $\A$ is the matrix generator defined in $(\ref{sweq:14})$. In this ``post-processing'' work, the following Lemma will be required, the proof of which closely follows from that of Proposition 2 of \cite{AD}.
    \begin{lemma}[Elliptic Regularity]
        Given a vector-valued function $\mu\in[\HM{1}{\Om_f}]^d\:\cap$ Null(div), suppose there exists a scalar-valued function $\rho\in\LP{2}{\Om_f}$ which satisfies
        \begin{align*}
            -\nabla\cdot(\nabla\mu+\nabla\mu^T)+\nabla\rho\in\text{Null(div)},
        \end{align*}
        where the subspace Null(div) is defined as
        \begin{align*}
            \text{Null(div)}=\set{f\in [\LP{2}{\Om_f}]^d}{\diver (f)=0\text{ in }\Om_f}.
        \end{align*}
        Then, $\rho$ is harmonic $(\Delta\rho=0\text{ in }\Om_f)$, and one has the following additional boundary regularity for the pair $(\mu,\rho)$:
        \begin{align*}
            \rho|_{\partial\Om_f}\in\HM{-1/2}{\partial\Om_f},\:\frac{\partial\rho}{\partial\nu}\bigg |_{\partial\Om_f}\in\HM{-3/2}{\partial\Om_f};\\
            (\nabla\mu+\nabla\mu^T)\cdot\nu|_{\partial\Om_f}\in[\HM{-1/2}{\partial\Om_f}]^d\\
            [\nabla\cdot(\nabla\mu+\nabla\mu^T)]\cdot\nu|_{\partial\Om_f}\in\HM{-3/2}{\partial\Om_f}
        \end{align*}\label{swlemma::1}
    \end{lemma}
    We now give cover the main result of the this section:
    \begin{thrm}
        With reference to the problem $(\ref{sweq:1})-(\ref{sweq:4})$, the operator $\A:D(\A)\subseteq\BH\to\BH$, defined in $(\ref{sweq:14})$, generates a $C_0$-semigroup of contractions on $\BH$. Consequently, the solution $\Phi(t)=[u(t),h(t),h_t(t),w(t),w_t(t)]$ of $(\ref{sweq:1})-(\ref{sweq:4})$, or equivalently $(\ref{sweq:13})$, is given by
        \begin{align*}
            \Phi(t)=e^{\A t}\Phi_0\in C([0,T];\BH),
        \end{align*}
        where $\Phi_0=[u_0,h_0,h_1,w_0,w_1]\in\BH$.\label{swthrm:2}
    \end{thrm}
\subsection{Dissipativity}
    \noindent Given the matrix generator $\A,\:\Phi_0=[u_0,h_0,h_1,w_0,w_1]\in D(\A)$, and $p_0$ as in $(\ref{sweq:15})$,
    \begin{align*}
        \IP{\A\Phi,\Phi}{\BH}=&\IP{\diver (\nabla (u_0)+\nabla ^T (u_0)),u_0}{\Om_f}+\IP{ \nabla_{\G_s}(h_1),\nabla_{\G_s}(h_0)}{\G_s}\\
        &+\IP{ -(\nabla u_0+\nabla^T u_0)\cdot\nu|_{\G_s},h_1}{\G_s}+\IP{ \Delta_{\G_s}(h_0),h_1}{\G_s}+\IP{ \sigma(w_0)\cdot\nu|_{\G_s},h_1}{\G_s}\\
        &+\IP{\sigma(w_1),\epsilon(w_0)}{\Om_s}+\IP{ w_1,w_0}{\Om_s}+\IP{\diver\:\sigma(w_0),w_1}{\Om_s}-\IP{ w_0,w_1}{\Om_s}\\
        &-\IP{ [\nabla \mathcal{P}_1(u_0)+\nabla\mathcal{P}_2(h_0)+\nabla\mathcal{P}_3(w_0)],u_0}{\Om_f}\\
        &+\IP{ [\mathcal{P}_1(u_0)\cdot\nu+\mathcal{P}_2(h_0)\cdot\nu+\mathcal{P}_3(w_0)\cdot\nu],h_1}{\G_s}.
    \end{align*}
    Applying Green's Theorem, using the fact that $u_0$ is solenoidal, and $u_0=0$ on $\G_f$,
    \begin{align*}
        \IP{ \A\Phi,\Phi}{\BH}=&\IP{ (\nabla(u_0)+\nabla^T(u_0))\cdot\nu,u_0}{\G_s}-\frac{1}{2}\bigg |\bigg |\nabla (u_0)+\nabla^T (u_0)\bigg |\bigg |^2\\
        &+\IP{ \nabla_{\G_s}(h_1),\nabla_{\G_s}(h_0)}{\G_s}-\IP{ (\nabla (u_0)+\nabla^T (u_0))\nu,h_1}{\G_s}\\
        &-\IP{ \nabla_{\G_s}(h_0),\nabla_{\G_s}(h_1)}{\G_s}+\IP{\sigma(w_0)\cdot\nu|_{\G_s},h_1}{\G_s}\\
        &+\IP{\sigma(w_1),\epsilon(w_0)}{\Om_s}+\IP{ w_1,w_0}{\Om_s}-\IP{\sigma(w_0)\cdot\nu|_{\G_s},w_1}{\G_s}\\
        &-\IP{\sigma(w_0),\epsilon(w_1)}{\Om_s}-\IP{ w_0,w_1}{\Om_s}.
    \end{align*}
    Now, using the matching velocity condition given in $(\textbf{A.v})$ it follows that
    \begin{align*}
        \IP{ \A\Phi,\Phi}\BH=&-\frac{1}{2}\big |\big |\nabla (u_0)+\nabla^T (u_0)\big |\big |^2+2i\Imag\big\{\IP{ \nabla_{\G_s}(h_1),\nabla_{\G_s}(h_0)}{\G_s}\\
        &+\IP{\sigma(w_1),\epsilon(w_0)}{\Om_s}\\
        &+\IP{ w_1,w_0}{\Om_s}+\IP{\sigma(w_1),\epsilon(w_0)}{\Om_s}+\IP{ w_1,w_0}{\Om_s}\big\}
    \end{align*}
    and hence
    \begin{align*}
        \Real\IP{ \A\Phi,\Phi}\BH=-\frac{1}{2}\big |\big |\nabla (u_0)+\nabla^T (u_0)\big |\big |^2\leq 0,
    \end{align*}
    which gives the dissipativity of the operator $\A$.

\subsection{Maximality}
    This step is the challenging part of the proof. In order to prove the maximality condition, it is shown that the operator $(\lambda I-\A)$ is surjective for $\lambda >0$. That is, it is established that the range condition
    \begin{align}
        \text{Range}(\lambda I-\A)=\BH,\label{sweq:17}
    \end{align}
    where parameter $\lambda >0$. Let $\Phi ^*=[u_0^*,h_0^*,h_1^*,w_0^*,w_1^*]\in\BH$, and consider the problem of finding $\Phi=[u_0,h_0,h_1,w_0,w_1]\in D(\A)$ which solves
    \begin{align}
        (\lambda I-\A)\Phi=\Phi^*,\:\lambda>0\label{sweq:18}
    \end{align}
    In PDE terms, this  resolvent equation will generate the following relations, where again $p_0$ is given via $(\ref{sweq:15})$:
    \begin{align}
        &\begin{cases}
            \lambda u_0-\diver (\nabla u_0+\nabla^T u_0)+\nabla p_0=u_0^*\text{ in }\Om_f\\
            \diver (u_0)=0\text{ in }\Om_f\\
            u_0|_{\G_f}=0\text{ on }\G_f;
        \end{cases}\label{sweq:19}\\
        &\begin{cases}
            \lambda h_0-h_1=h_0^*\text{ in }\G_s\\
            \lambda h_1+[\nu\cdot(\nabla u_0+\nabla^T u_0)]|_{\G_s}-\Delta_{\G_s}(h_0)-[\nu\cdot\sigma (w_0)]|_{\G_s}-p_0\nu=h_1^*\text{ in }\G_s
        \end{cases}\label{sweq:20}\\
        &\begin{cases}
            \lambda w_0-w_1=w_0^*\text{ in }\Om_s\\
            \lambda w_1-\diver \sigma (w_0)+w_0=w_1^*\text{ in }\Om_s\\
            w_1|_{\G_s}=h_1=u_0|_{\G_s}\text{ on }\G_s
        \end{cases}\label{sweq:21}
    \end{align}
    Because of the composite structure of the coupled dynamics, and the matching fluid and structure velocities given in $(\ref{sweq:23})_3$, the static problem $(\ref{sweq:19})-(\ref{sweq:21})$ cannot be solved via mixed variational approaches which are given for uncoupled fluid flows (see pg. 15 and pg. 107 of \cite{temam}). Hence, the proof is based on solving an appropriate and nonstandard mixed variational problem formulated for the static PDE system $(\ref{sweq:19})-(\ref{sweq:21})$. It is important to note that the mixed variational formulation is generated and solved with respect to the ``thin'' and ``thick'' structure variables $h_1$ and $w_1$. For this, the author mainly appeals to the Babuska-Brezzi approach (see Theorem $\ref{BB}$). Once these variables are solved, then the recovery of the other structure solution variables $h_0$ and $w_0$ will be via the relations given in $(\ref{sweq:20})_1$ and $(\ref{sweq:21})_1$ for given data $h_0^*\in\BFHM{1}{\G_s}$ and $w_0^*\in\BFHM{1}{\Om_s}$, i.e.,
    \begin{align}
        h_0=\frac{1}{\lambda}h_1+\frac{1}{\lambda}h_0^*\label{sweq:22}\\
        w_0=\frac{1}{\lambda}w_1+\frac{1}{\lambda}w_0^*\label{sweq:23}
    \end{align}
    Because of the matching velocities boundary conditions in $(\ref{sweq:21})$,
    \begin{align*}
        \int_{\G_s}w_1|_{\G_s}d\G_s=\int_{\G_s}h_1d\G_s=0.
    \end{align*}
    Before starting to generate a weak formulation for the ``thin'' and ``thick'' structure variables $h_1$ and $w_1$, the author first proceeds with the fluid component $u_0$: For given $g\in\BFHM{1/2}{\G_s}$, the unique pair $[u_1(g),p_1(g)]\in\BFHM{1}{\Om_f}\times\hat{L}^2(\Om_f)$ solves the following static Stokes equation:
    \begin{align}
        \begin{cases}
            \lambda u_1-\diver (\nabla u_1+\nabla^T u_1)+\nabla p_1=0\text{ in }\Om_f\\
            \diver (u_1)=\frac{\int_{\G_s}(g\cdot\nu)d\G_s}{\text{meas}(\Om_f)}\text{ in }\Om_f\\
            u_1|_{\G_s}=g\text{ on }\G_s\\
            u_1|_{\G_f}=0\text{ on }\G_f,
        \end{cases}\label{sweq:24}
    \end{align}
    See (\cite{temam}, pg 22, Theorem 2.4). Here,
    \begin{align*}
        \hat{L}^2(\Om_f)=\set{f\in\LP{2}{\Om_f}}{\int_{\Om_f}fd\Om_f=0}
    \end{align*}
    In particular, the compatibility condition holds for the solvability of the above problem, and so $p_1$ is unique up to a constant (see, e.g., \cite{temam}, pg 31, Theorem 2.4). In a similar way, for a given force term $u_0^*\in\BFLP{2}{\Om_f}$, the unique pair $[u_2(u_0^*),p_2(u_0^*)]\in\BFHM{1}{\Om_f}\times\hat{L}^2(\Om_f)$ solves the following problem:
    \begin{align}
        \begin{cases}
            \lambda u_2-\diver (\nabla u_2+\nabla^T u_2)+\nabla p_2=u_0^*\text{ in }\Om_f\\
            \diver(u_2)=0\text{ in }\Om_f\\
            u_2|_{\G_f}\text{ on }\G_f
        \end{cases}\label{sweq:25}
    \end{align}
    With the solution maps of $(\ref{sweq:24})-(\ref{sweq:25})$, the unique $\{u_0,p_0\}$ of $(\ref{sweq:19})$ may then be expressed as
    \begin{align}
        u_0=u_1(w_1|_{\G_s})+u_2(u_0^*);\qquad p_0=p_1(w_1|_{\G_s})+p_2(u_0^*)+c_0,\label{sweq:26}
    \end{align}
    where $c_0$ is the (presently) unknown constant component of the pressure $p_0$ of $(\ref{sweq:19})$.\\
    Define the space
    \begin{align*}
        \textbf{S}=\set{(\varphi,\psi)\in\BFHM{1}{\G_s}\times\BFHM{1}{\Om_s}}{\varphi=\psi|_{\G_s}}.
    \end{align*}
    In order to generate a mixed variational formulation for the static ``thin'' and ``thick'' solution variables in $(\ref{sweq:20})-(\ref{sweq:21})$, by respectively multiplying $(\ref{sweq:20})_2$ and $(\ref{sweq:21})_2$ by functions $\phi\in\BFHM{1}{\G_s}$, and $\psi\in\BFHM{1}{\Om_s}$ from the space $\textbf{S}$, use Green's Theorem, and add the subsequent relations, (taking into account $(\ref{sweq:22})-(\ref{sweq:23})$) to get:
    \begin{align}
        \lambda&\IP{ h_1,\varphi}{\G_s}+\frac{1}{\lambda}\IP{ \nabla_{\G_s}h_1,\nabla_{\G_s}\varphi}{\G_s}+\lambda\IP{ w_1,\psi}{\Om_s}\nonumber\\
        &+\frac{1}{\lambda}\IP{\sigma(w_1),\epsilon(\psi)}{\Om_s}+\frac{1}{\lambda}\IP{ w_1,\psi}{\Om_s}-c_0\IP{\nu,\varphi}{\G_s}\nonumber\\
        &+\IP{ \nu\cdot(\nabla u_0+\nabla^T u_0)|_{\G_s},\varphi}{\G_s}-\IP{ (p_1+p_2)\nu,\varphi}{\G_s}\nonumber\\
        =&-\frac{1}{\lambda}\IP{ \nabla_{\G_s}h_0^*,\nabla_{\G_s}\varphi}{\G_s}-\frac{1}{\lambda}\IP{ \sigma (w_0^*),\epsilon (\psi)}{\Om_s}\nonumber\\
        &+\IP{ h_1^*,\varphi}{\G_s}+\IP{ w_1^*,\psi}{\Om_s}-\frac{1}{\lambda}\IP{ w_0^*,\psi}{\Om_s}.\label{sweq:27}
    \end{align}
    In order to estimate the sum in $(\ref{sweq:27})$, recall that for any $\varphi\in\BFHM{1/2}{\G_s}$, there is a unique pair $[\widetilde{u}(\varphi),\widetilde{p}(\varphi)]\in\BFHM{1}{\Om_f}\times\hat{L}^2(\Om_f)$ which solves the BVP
    \begin{align}
        \begin{cases}
            \lambda\widetilde{u}-\diver (\nabla\widetilde{u}+\nabla^T\widetilde{u})+\nabla\widetilde{p}=0\text{ in }\Om_f\\
            \diver{\widetilde{u}}=\frac{\int_{\G_s}(\varphi\cdot\nu)d\G_s}{\text{meas}(\Om_f)}\text{ in }\Om_f\\
            \widetilde{u}|_{\G_s}=\varphi\text{ on }\G_s\\
            \widetilde{u}|_{\G_f}=0\text{ on }\G_f.\label{sweq:29}
        \end{cases}
    \end{align}
    Appealing to problem $(\ref{sweq:29})$, for the sum in $(\ref{sweq:27})$,
    \begin{align}
        \langle\nu&\cdot(\nabla u_0+\nabla^T u_0)|_{\G_s},\varphi\rangle_{\G_s}-\IP{(p_1+p_2)\nu,\varphi}{\G_s}\nonumber\\
        =&\IP{ \nabla u_0+\nabla^T u_0,\nabla\widetilde{u}(\varphi)+\nabla^T\widetilde{u}(\varphi)}{\Om_f}+\IP{ \diver (\nabla u_0+\nabla^T u_0),\widetilde{u}(\varphi)}{\Om_f}\nonumber\\
        &-\IP{\nabla(p_1+p_2),\widetilde{u}(\varphi)}{\Om_f}-\IP{ p_1+p_2,\diver\widetilde{u}(\varphi)}{\Om_f}\\
        =&\IP{ \nabla u_0+\nabla^T u_0,\nabla \widetilde{u}(\varphi)+\nabla^T\widetilde{u}(\varphi)}{\Om_f}+\lambda\IP{ u_0,\widetilde{u}(\varphi)}{\Om_f}\nonumber\\
        &-\IP{ u_0^*,\widetilde{u}(\varphi)}{\Om_f}-\IP{ p_1+p_2,\diver\widetilde{u}(\varphi)}{\Om_f},\label{sweq:30}
    \end{align}
    where the author has also used $(\ref{sweq:19})_1$. Then from the last relation and the fluid representation in $(\ref{sweq:26})$,
    \begin{align}
        \langle \nu\cdot(&\nabla u_0+\nabla^T u_0)|_{\G_s},\varphi\rangle_{\G_s}-\IP{ (p_1+p_2)\nu,\varphi}{\G_s}\nonumber\\
        =&\IP{ \nabla u_1(w_1|_{\G_s})+\nabla^Tu_1(w_1|_{\G_s}),\nabla\widetilde{u}(\varphi)+\nabla^T\widetilde{u}(\varphi)}{\Om_f}\nonumber\\
        &+\IP{\nabla u_2(u_0^*)+\nabla^T u_2(u_0^*),\nabla\widetilde{u}(\varphi)+\nabla^T\widetilde{u}(\varphi)}{\Om_f}\nonumber\\
        &+\lambda\IP{ u_1(w_1|_{\G_s}),\widetilde{u}(\varphi)}{\Om_f}+\lambda\IP{ u_2(u_0^*),\widetilde{u}(\varphi)}{\Om_f}\nonumber\\
        &\boxed{-\IP{ p_1(w_1|_{\G_s}),\diver \widetilde{u}(\varphi)}{\Om_f}-\IP{ p_2(u_0^*),\diver\widetilde{u}(\varphi)}{\Om_f}}\nonumber\\
        &-\IP{ u_0^*,\widetilde{u}(\varphi)}{\Om_f}.\label{sweq:31}
    \end{align}
    Since $p_1(w_1|_{\G_s})$ and $p_2(u_0^*)$ are each in $\hat{L}^2(\Om_f)$, then the boxed term of $(\ref{sweq:31})$ disappears. Combining $(\ref{sweq:27})-(\ref{sweq:31})$,
    \begin{align}
        \lambda\langle  h_1&,\varphi\rangle_{\G_s}+\frac{1}{\lambda}\IP{\nabla_{\G_s}h_1,\nabla_{\G_s}\varphi}{\G_s}+\lambda\IP{ w_1,\psi}{\Om_s}+\frac{1}{\lambda}\IP{ \sigma(w_1),\epsilon(\psi)}{\Om_s}\nonumber\\
        &+\frac{1}{\lambda}\IP{ w_1,\psi}{\Om_s}-c_0\IP{ \nu,\varphi}{\G_s}+\lambda\IP{ u_1(w_1|_{\G_s}),\widetilde{u}(\varphi)}{\Om_f}\nonumber\\
        &+\IP{\nabla u_1(w_1|_{\G_s})+\nabla^T u_1(w_1|_{\G_s}),\nabla\widetilde{u}(\varphi)+\nabla^T\widetilde{u}(\varphi)}{\Om_f}\nonumber\\
        =&-\IP{ \nabla u_2(u_0^*)+\nabla^T u_2(u_0^*),\nabla\widetilde{u}(\varphi)+\nabla^T \widetilde{u}(\varphi)}{\Om_f}\nonumber\\
        &-\lambda\IP{ u_2(u_0^*),\widetilde{u}(\varphi)}{\Om_f}+\IP{ u_0^*,\widetilde{u}(\varphi)}{\Om_f}\nonumber\\
        &-\frac{1}{\lambda}\IP{ \nabla_{\G_s}h_0^*,\nabla_{\G_s}\varphi}{\G_s}-\frac{1}{\lambda}\IP{\sigma(w_0^*),\epsilon(\psi)}{\Om_s}\nonumber\\
        &+\IP{ h_1^*,\varphi}{\G_s}+\IP{ w_1^*,\psi}{\Om_s}-\frac{1}{\lambda}\IP{ w_0^*,\psi}{\Om_s}.\label{sweq:32}
    \end{align}
    The last relation now gives the following mixed variational formulation in terms of the ``thin'' and ``thick'' structure variables $h_1$ and $w_1$: Namely,
    \begin{align}
        \textbf{a}([h_1,w_1],[\varphi,\psi])+&\textbf{b}([\varphi,\psi],c_0)=\textbf{F}([\varphi,\psi]),\text{ for all }[\varphi,\psi]\in\textbf{S}\nonumber\\
        &\textbf{b}([h_1,w_1],r)=0,\text{ for all }r\in\R.\label{sweq:33}
    \end{align}
    Here, the bilinear forms $\textbf{a}(\cdot,\cdot):\textbf{S}\times\textbf{S}\to\R$ and $\textbf{b}(\cdot,\cdot):\textbf{S}\times\R\to\R$ are respectively given as
    \begin{align*}
        \textbf{a}([\phi,\xi],[\widetilde{\phi},\widetilde{\xi}])=&\lambda\IP{\phi,\widetilde{\phi}}{\G_s}+\frac{1}{\lambda}\IP{\nabla_{\G_s}\phi,\nabla_{\G_s}\widetilde{\phi}}{\G_s}\\
        &+\lambda\IP{\xi,\widetilde{\xi}}{\Om_s}+\frac{1}{\lambda}\IP{\sigma(\xi),\epsilon(\widetilde{\xi})}{\Om_s}+\frac{1}{\lambda}\IP{\xi,\widetilde{\xi}}{\Om_s}\\
        &+\IP{\nabla u_1(\xi|_{\G_s})+\nabla^T u_1(\xi|_{\G_s}),\nabla\widetilde{u}(\widetilde{\phi})+\nabla^T\widetilde{u}(\widetilde{\phi})}{\Om_f}\\
        &+\lambda\IP{ u_1(\xi|_{\G_s}),\widetilde{u}(\widetilde{\phi})}{\Om_f},
    \end{align*}
    and
    \begin{align*}
        \textbf{b}([\widetilde{\phi},\widetilde{\xi}],r)=-r\IP{ v,\widetilde{\phi}}{\G_s},
    \end{align*}
    and the functional $\textbf{F}(\cdot)$ is defined as
    \begin{align*}
        \textbf{F}([\widetilde{\phi},\widetilde{\xi}])=&-\IP{\nabla u_2(u_0^*)+\nabla^T u_2(u_0^*),\nabla\widetilde{u}(\widetilde{\phi})+\nabla^T \widetilde{u}(\widetilde{\phi})}{\Om_f}\\
        &-\frac{1}{\lambda}\IP{\nabla_{\G_s}h_0^*,\nabla_{\G_s}\widetilde{\phi}}{\G_s}-\frac{1}{\lambda}\IP{\sigma(w_0^*),\epsilon(\widetilde{\xi})}{\Om_s}\\
        &-\lambda\IP{ u_2(u_0^*),\widetilde{u}(\widetilde{\phi})}{\Om_f}+\IP{ u_0^*,\widetilde{u}(\widetilde{\phi})}{\Om_f}\\
        &+\IP{ h_1^*,\widetilde{\phi}}{\G_s}+\IP{ w_1^*,\widetilde{\xi}}{\Om_s}-\frac{1}{\lambda}\IP{ w_0^*,\widetilde{\xi}}{\Om_s}.
    \end{align*}
    The remainder of the proof hinges on properly applying Theorem $\ref{BB}$. It is clear that the bilinear forms $\textbf{a}(\cdot,\cdot)$ and $\textbf{b}(\cdot,\cdot)$ are continuous and moreover $\textbf{a}(\cdot,\cdot)$ is $\text{S}$-elliptic. In order to conclude that the variational problem $(\ref{sweq:33})$ has a unique solution, the author needed to show that the bilinear form $\textbf{b}(\cdot,\cdot)$ satisfies the ``inf-sup'' condition given in Theorem $\ref{BB}$. For this, consider the following problem:\\
    Given $r\in\R$, let $z\in\BFHM{1}{\G_s}$ satisfy
    \begin{align*}
        \Delta_{\G_s}z=\text{sgn}(r)\nu\text{ on }\G_s
    \end{align*}
    It is easily seen that $||\nabla_{\G_s}z||_{\G_s}\leq C||\nu||_{\G_s}$. Now, taking into account that $\gamma:\HM{1}{\Om_s}\to\HM{1/2}{\G_s}$ is a surjective map, and so it has a continuous right inverse $\gamma^+(z)$,
    \begin{align*}
        \sup_{[\eta,\varsigma]}\frac{\textbf{b}([\eta,\varsigma],r)}{||[\eta,\varsigma]||_\textbf{S}}&\geq \frac{\textbf{b}([z,\gamma^+(z)],r)}{||z||_{\BFHM{1}{\G_s}}}\\
        &=\frac{-r\int_{\G_s}\nu\cdot zd\G_s}{||z||_{\BFHM{1}{\G_s}}}\\
        &=-r\text{ sgn}(r)\frac{\int_{\G_s}\Delta_{\G_s}z\cdot z d\G_s}{||z||_{\BFHM{1}{\G_s}}}\\
        &=|r|\frac{\int_{\G_s}|\nabla_{\G_s}z|^2d\G_s}{||z||_{\BFHM{1}{\G_s}}}\\
        &=|r|\:||z||_{\BFHM{1}{\G_s}}
    \end{align*}
    which yields that the inf-sup condition holds wth the constant $\beta=||z||_{\BFHM{1}{\G_s}}$. Consequently, the existence and uniqueness of the solution $[h_1,w_1]\in\textbf{S}$ and $c_0\in\R$ to the mixed variational problem $(\ref{sweq:33})$ follows from Theorem $\ref{BB}$.\\
    Now, the unique attainment of the solution components $[h_1,w_1]\in\textbf{S}$ allows the subsequent recovery of the solution variables $h_0$ and $w_0$ via the resolvent relations $(\ref{sweq:22})-(\ref{sweq:23})$, with
    \begin{align*}
        h_0=w_0|_{\G_s},
    \end{align*}
    since the data $\Phi^*=[u_0^*,h_0^*,h_1^*,w_0^*,w_1^*]\in\BH$. Moreover, having the ``thick'' structure component
    \begin{align*}
        w_1\in\BFHM{1}{\Om_s}\text{ with }w_1|_{\G_s}=h_1\in\BFHM{1}{\G_s},
    \end{align*}
    and $u_0^*\in\BFLP{2}{\Om_f}$, the fluid solution component $u_0$ and the pressure term $p_0$ can be given via the expressions in $(\ref{sweq:26})$. That is, the unique pair $\{u_0,p_0\}\in\BFHM{1}{\Om_f}\times\LP{2}{\Om_f}$ solves the Stokes system
    \begin{align}
        \begin{cases}
            \lambda u_0-\diver(\nabla u_0+\nabla^T u_0)+\nabla p_0=u_0^*\text{ in }\Om_f\\
            \diver(u_0)=0\text{ in }\Om_f\\
            u_0|_{\G_f}=0\text{ on }\G_f\\
            u_0|_{\G_s}=w_1\text{ on }\G_s.
        \end{cases}\label{sweq:34}
    \end{align}
    To conclude the proof of maximality, it is shown that the derived solution $[u_0,h_0,h_1,w_0,w_1]\in D(\A)$ and satisfies the resolvent Eqs. $(\ref{sweq:19})-(\ref{sweq:21})$. First, by $(\ref{sweq:26})$, it is clear that $\{u_0,p_0\}\in\BFHM{1}{\Om_f}\times\LP{2}{\Om_f}$ where $u_0=u_1+u_2;\:p_0=p_1+p_2+c_0$ satisfies $(\ref{sweq:19})$. Consequently, by Lemma $\ref{swlemma::1}$,
    \begin{align*}
        p_0|_{\G_f}\in\HM{-1/2}{\G_f},\quad\frac{\partial p_0}{\partial\nu}\bigg |_{\G_f}\in\HM{-3/2}{\G_f},
    \end{align*}
    and
    \begin{align*}
        (\nabla u_0+\nabla u_0^T)\cdot\nu|_{\G_f}\in[\HM{-1/2}{\G_f}]^d;\:[\nabla\cdot(\nabla u_0+\nabla^T u_0)]\nu|_{\G_f}\in\HM{-3/2}{\G_f}.
    \end{align*}
    Moreover, if in $(\ref{sweq:33})$, $\phi=0$, and $\psi\in[\mathcal{D}(\Om_s)]^3$, then
    \begin{align*}
        \lambda\IP{ w_1,\psi}{\Om_s}+\frac{1}{\lambda}\IP{\sigma(w_1),\epsilon(\psi)}{\Om_s}+\frac{1}{\lambda}\IP{ w_1,\psi}{\Om_s}\\
        =-\frac{1}{\lambda}\IP{\sigma(w_0^*),\epsilon(\psi)}{\Om_s}+\IP{ w_1^*,\psi}{\Om_s}-\frac{1}{\lambda}\IP{ w_0^*,\psi}{\Om_s}.
    \end{align*}
    Thus, via the resolvent relation $w_0=\frac{1}{\lambda}[w_1+w_0^*]$,
    \begin{align}
        \lambda w_1-\diver \sigma(w_0)+w_0=w_1^*\text{ in }\BFLP{2}{\Om_s},\label{sweq:35}
    \end{align}
    which gives $(\ref{sweq:21})$. Since also $w_0\in\BFHM{1}{\Om_s}$, an energy method yields in turn,
    \begin{align}
        \nu\cdot\sigma(w_0)\in\HM{-1/2}{\G_s}\label{sweq:36}
    \end{align}
    via $(\ref{sweq:35})$. This gives, for all $[\varphi,\psi]\in\textbf{S}$, (after using the resolvent relation $h_0=\frac{1}{\lambda}(h_1+h_0^*)$):
    \begin{align*}
        -\lambda\IP{ w_1,\psi}{\Om_s}+\IP{\diver\sigma(w_0),\psi}{\Om_s}-\IP{ w_0,\psi}{\Om_s}=-\IP{ w_1^*,\psi}{\Om_s},
    \end{align*}
    and
    \begin{align*}
        \lambda\langle h_1,&*\varphi\rangle_{\G_s}+\IP{\nabla_{\G_s}h_0,\nabla_{\G_s}\varphi}{\G_s}+\lambda\IP{w_1,\psi}{\Om_s}+\IP{\sigma(w_0),\epsilon(\psi)}{\Om_s}\\
        &+\IP{w_0,\psi}{\Om_s}-c_0\IP{\nu,\varphi}{\G_s}-\IP{(p_1+p_2)\nu,\varphi}{\G_s}+\IP{(p_1+p_2)\nu,\varphi}{\G_s}\\
        &+\IP{\nabla u_0+\nabla^T u_0,\nabla\widetilde{u}(\varphi)+\nabla^T\widetilde{u}(\varphi)}{\Om_f}+\lambda\IP{u_0,\widetilde{u}(\varphi)}{\Om_f}\\
        =&\IP{h_1^*,\varphi}{\G_s}+\IP{w_1^*,\psi}{\Om_s}+\IP{u_0^*,\widetilde{u}(\varphi)}{\Om_f}.
    \end{align*}
    Integrating by parts (having in hand the boundary trace in $(\ref{sweq:36})$) and adding the two relations now gives
    \begin{align*}
        \lambda\langle h_1,&\varphi\rangle_{\G_s}+\IP{\nabla_{\G_s}h_0,\nabla_{\G_s}\varphi}{\G_s}-\IP{\nu\cdot\sigma(w_0),\varphi}{\G_s}\\
        &-\IP{p\nu,\varphi}{\G_s}+\IP{\nabla p\nu,\varphi}{\Om_f}+\lambda\IP{u_0,\widetilde{u}(\varphi)}{\Om_f}\\
        &-\IP{\diver[\nabla u_0+\nabla^T u_0],\widetilde{u}(\varphi)}{\Om_f}+\IP{\nu\cdot[\nabla u_0+\nabla^T u_0],\varphi}{\Om_s}\\
        =&\IP{h_1^*,\varphi}{\G_s}+\IP{u_0^*,\widetilde{u}(\varphi)}{\Om_f}\text{ for all }[\varphi,\psi]\in\textbf{S}.
    \end{align*}
    This finally gives the inference that
    \begin{align*}
        \lambda\langle h_1,&\varphi\rangle_{\G_s}+\IP{\nabla_{\G_s}h_0,\nabla_{\G_s}\varphi}{\G_s}-\IP{\nu\cdot\sigma(w_0),\varphi}{\G_s}\\
        &-\IP{p\nu,\varphi}{}+\IP{\nu\cdot[\nabla u_0+\nabla^T u_0],\varphi}{\Om_s}\\
        =&\IP{h_1^*,\varphi}{\G_s},\text{ for all }\varphi\in[\mathcal{D}(\Om_s)]^3
    \end{align*}
    which gives the ``thin'' structural equation in $(\ref{sweq:20})$, and the proof of Theorem $\ref{swthrm:2}$ now finishes.