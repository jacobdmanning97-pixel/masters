\chapter{Conclusions and Discussion of Future Work}
As we have discussed, multilayered FSI PDEs arise in the context of blood transportation and cellular dynamics. To this end, the models that have been discussed in this project gives a better understanding to what is actually happening in the real world. The canonical model provides a great foundation both numerically and analytically for future work, which was improved upon in the following Stokes flow model. Along with the additionally complexity added in the second model, there were new challenges such as the pressure term. This called for elimination using elliptic boundary value problems and restrictions for the strong decay to avoid intersection with the imaginary axis. Something that was not covered in this project was that in \cite{RD}, the authors showed that the canonical model displayed rational decay.\\
\indent It has also been theorized that in all of the aforementioned models, the models do not exhibit exponential decay. For example, in \cite{micuzuazua}, exponential decay has only been shown in simpler models in lower dimensions or for the canonical model in 2D using eigenfunctions to generate solutions explicitly. Thus most effort goes towards establishing rational decay, see \cite{AvalosLasieckaTriggiani16, RD, RauchZhangZuazua, ZhangZuazuaARMA07, micuzuazua} which is not as strong as exponential decay.  Rational decay has not been shown for Stokes flow model and thus is an open problem being actively worked on.\\
\indent Finally, as both of these models are simplifications of the  Navier-Stokes model, future work would be forced to tackle the full physical model. One of the major issues that is introduced with the full Navier-Stokes model is that the problem suddenly becomes nonlinear. This would cause major issues as the methodology used in the proofs that we covered is based on linear $C_0$-semigroups. Thus changing to the Navier-Stokes model would require a different approach with potentially unforeseen difficulties.\\