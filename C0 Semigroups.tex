
\begin{define}
        Let $X$ be a Banach space and $\{S(t)\}_{t\geq 0}$ be a family of bounded linear operators on $X$. It is said to be a $C_0$ semigroup if the following are true:
        \begin{itemize}
            \item $S(0)=I$, the identity of $X$\\
            \item $S(t+s)=S(t)S(s)$, for all $t,s\geq 0$\\
            \item For every $u\in X$
                \begin{align*}
                    S(t)u\to u \quad \text{as } t\downarrow 0
                \end{align*}
        \end{itemize}
    \end{define}

\begin{thrm}
    Let $\{S(t)\}_{t\geq 0}$ be a $C_0$-semigroup on $X$. Then there exists $M\geq 1$ and $\omega$ such that
    \begin{align*}
        ||S(t)||\leq Me^{\omega t},\quad \text{for all } t\geq 0
    \end{align*}
\end{thrm}

\begin{define}
    If $M=1$ and $\omega = 0$, so that $||S(t)||\leq 1$ for all $t\geq 0$, we say that $\{S(t)\}$ is a \textbf{contraction semigroup}.
\end{define}

\begin{define}
    Let $\{S(t)\}_{t\geq 0}$ be a $C_0$ semigroup on $X$. The \textbf{infinitesimal generator} of the semigroup is a linear operator $A$ given by
    \begin{align*}
        D(A)&=\bigg \{u\in X\:|\: \lim_{t\downarrow0} \frac{S(t)u-u}{t} \:\text{exists}\bigg \}\\
        Au&=\lim_{t\downarrow0} \frac{S(t)u-u}{t},\: u\in D(A)
    \end{align*}
\end{define}

\begin{thrm}
    Let $\{S(t)\}_{t\geq 0}$ be a $C_0$ semigroup and let $A$ be its infinitesimal generator. Let $u\in D(A)$. Then
    \begin{align*}
        S(t)u\in C^1([0,\infty);X)\cap C([0,\infty);X)
    \end{align*}
    and
    \begin{align*}
        \frac{d}{dt}(S(t)u)=AS(t)u=S(t)Au
    \end{align*}
\end{thrm}

\begin{note}
    If $A$ is the infinitesimal generator of a $C_0$ semigroup $\{S(t)\}$ then we know by the above theorem that
    \begin{align*}
        u(t)=S(t)u_0
    \end{align*}
    defines the unique solution of the initial value problem
    \begin{align*}
        \begin{rcases}
        \frac{du(t)}{dt}&=Au(t),\:t\geq 0\\
        u(0)&=u_0
        \end{rcases}
    \end{align*}
\end{note}

\begin{thrm}[Hille Yosida]
    A linear unbounded operator $A$ on a Banach space $X$ is the infinitesimal generator of a contraction semigroup if and only if
    \begin{itemize}
        \item $A$ is closed\\
        \item $A$ is densely defined\\
        \item For every $\lambda>0,\: (\lambda I-A)^{-1}$ is a bounded linear operator and
        \begin{align*}
            ||(\lambda I-A)^{-1}||\leq \frac{1}{\lambda}
        \end{align*}
    \end{itemize}
\end{thrm}

\begin{thrm}[Lax-Milgram]
    Let $V$ be a Hilbert space and $a(\cdot,\cdot)$ a continuous $V$-elliptic bilinear form. Then given $f\in V$, there exists a unique $u\in V$ such that
    \begin{align*}
        a(u,v)=(f,v),\quad \text{for every }v\in V.
    \end{align*}
    If $a(\cdot,\cdot)$ is also symmetric then the functional $J:V\to \R$ defined by
    \begin{align*}
        J(v)=\frac{1}{2}a(v,v)-(f,v)
    \end{align*}
    attains its minimum at $u$.\label{Lax}
\end{thrm}

\subsection{Pazy\cite{pazy}}
\begin{define}
    Let $X$ be a Banach space with dual space $X'$. Denote $x'\in X'$ at $x\in X$ by $\langle x',x\rangle$ or $\langle x,x'\rangle$. Define the following set $F(x)\subseteq X'$ as
    \begin{align*}
        F(x)=\set{x'}{\langle x',x\rangle=||x||^2=||x'||^2}
    \end{align*}
    (This set is non-empty by the Hahn-Banach theorem.)
\end{define}

\begin{define}[Dissipativity]
    A linear operator $A$ is dissipative if for every $x\in D(A)$ there is a $x'\in F(x)$ such that $Re\langle Ax,x'\rangle\leq 0$
\end{define}

\begin{define}[Maximal Dissipativity]
    A linear operator $A$ is called maximally dissipative if it is dissipative and $R(I-A)=X$.
\end{define}

\begin{thrm}[Lumer-Phillips]
    \begin{itemize}
        \item If $A$ is dissipative and there is a $\lambda _0 >0$ such that $R(\lambda _0 I-A)=X$, then $A$ is the infinitesimal generator of a $C_0$ semigroup of contractions on $X$.
        \item If $A$ is the infinitesimal generator of a $C_0$ semigroup of contractions on $X$ then $R(\lambda I-A)=X$ for all $\lambda >0$ and A is dissipative.
    \end{itemize}\label{LP}
\end{thrm}

\subsection{Kreyszig\cite{krey}}
\begin{define}
    \begin{itemize}
        \item A linear operator $A:D(A)\subseteq X\to Y$ is said to be \textbf{bounded} if there exists a $C>0$ such that
        \begin{align*}
            ||Au||_Y\leq C||u||_X,\quad \text{for every } u\in D(A)
        \end{align*}
        Otherwise it is said to be \textbf{unbounded}.
        \item A linear operator $A:D(A)\subseteq X\to Y$ is said to be \textbf{densely defined} if $\overline{D(A)}=X$
        \item A linear operator $A:D(A)\subseteq X\to Y$ is said to be \textbf{closed} if the \textbf{graph}
        \begin{align*}
            G(A)=\set{(u,Au)}{u\in D(A)}\subseteq X\times Y
        \end{align*}
        is closed as a subspace of $X\times Y$
    \end{itemize}
\end{define}

\begin{define}
    Let $X\neq \{0\}$ be a complex normed space and $T:\D(T)\subseteq X\to X$ be a linear operator. With $T$ we associate the operator
    \begin{align*}
        T_\lambda =T-\lambda I
    \end{align*}
    where $\lambda$ is a complex number and $I$ is the identity operator on $\D(T)$. If $T_\lambda$ has an inverse, we denote it by $R_\lambda (T)$ and call it the \textit{resolvent operator} of $T$ or, simply, the \textbf{resolvent} of $T$. If it is clear which operator we are discussing, we will write $R_\lambda$.
\end{define}

\begin{define}[Regular value, resolvent set, spectrum]
    Let $X\neq \{0\}$ be a complex normed space and $T:\D(T)\subseteq X\to X$ be a linear operator. A \textit{regular value} $\lambda$ of $T$ is a complex number such that
    \begin{itemize}
        \item $R_\lambda (T)$ exists,\\
        \item $R_\lambda (T)$ is bounded,\\
        \item $R_\lambda (T)$ is densely defined.
    \end{itemize}
    The \textit{resolvent set} $\rho(T)$ of $T$ is the set of all regular values $\lambda$ of $T$. Its complement $\sigma (T)=\mathbb{C}\backslash\rho(T)$ in the complex plane $\mathbb{C}$ is called the \textit{spectrum} of $T$, and a $\lambda \in \sigma (T)$ is called a \textit{spectral value} of $T$. Furthermore, the spectrum $\sigma(T)$ is partitioned into three disjoint sets as follows.\\
    The \textbf{point spectrum} or \textit{discrete spectrum} $\sigma_p(T)$ is the set such that $R_\lambda(T)$ does not exist. A $\lambda\in\sigma_p(T)$ is called an \textit{eigenvalue} of $T$.\\
    The \textbf{continuous spectrum} $\sigma_c(T)$ is the set such that $R_\lambda (T)$ exists and is densely defined, but it unbounded.\\
    The \textbf{residual spectrum} $\sigma_r(T)$ is the set such that $R_\lambda(T)$ exists, but is not densely defined (may or may not be bounded).
\end{define}