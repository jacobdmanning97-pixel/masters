\documentclass{article}

\usepackage{amsmath,amssymb,bigints,amsthm,mathrsfs,mathtools,biblatex}

\addbibresource{Bibliography.bib}

\newtheorem{thrm}{Theorem}
\newtheorem{cor}{Corollary}
\newtheorem{lemma}{Lemma}
\newtheorem{define}{Definition}
\newtheorem{prop}{Proposition}
\newtheorem{note}{Remark}

\newcommand{\R}{\mathbb{R}}
\newcommand{\D}{\mathscr{D}}
\newcommand{\E}{\mathscr{E}}
\newcommand{\Sch}{\mathcal{S}}
\newcommand{\Lin}{\mathscr{L}}
\newcommand{\Om}{\Omega}
\newcommand{\W}[2]{W^{#1}(#2)}
\newcommand{\WO}[2]{W_0^{#1}(#2)}
\newcommand{\HM}[2]{H^{#1}(#2)}
\newcommand{\LP}[2]{L^{#1}(#2)}
\newcommand{\set}[2]{\{#1\:|\:#2\}}

\title{Sobolev Spaces}
\author{Jacob Manning}
\date{February 2025}

\begin{document}
\tableofcontents
\newpage

\section{Notation}
This might need some material added
\newpage

\section{1.3 Some Operations with Distributions}
    \begin{define}\ \\
        The space of $\textbf{test-functions}$ denoted as $\D(\Om)$, where $\Om$ is any open set in $\R^n$, is a subset of $C^\infty(\Om)$ that have compact support contained within $\Om$.
    \end{define}
    
    \begin{define}[1.2.1]\ \\
        A sequence of functions $\{\phi_m\}$ in $\D(\Om)$ is said to converge to $0$ if there exists a \textit{fixed} compact set $K\subset \Om$ such that supp$(\phi_m)\subset K$ for all $m$ and $\phi_m$ and all its derivatives converge \textit{uniformly} to zero on $K$.
    \end{define}
    
    \begin{define}[1.2.2]\ \\
        A linear functional $T$ on $\D(\Om)$ is said to be a \textbf{distribution} on $\Om$ if whenever $\phi_m\to 0$ in $\D(\Om)$, we have $T(\phi_m)\to 0$.\\
        The space of distributions, which is the dual of the space of test-functions, is denoted by $\D'(\Om)$. In the case $\Om=\R^n$, the symbol $\D'$ will also be used.
    \end{define}
    
    \begin{define}\ \\
        Let $x\in \R^n$ with coordinates $(x_1,...,x_n)$. A \textit{multi-index} is a $n$-tuple
        \begin{align*}
            \alpha=(\alpha_1,...,\alpha_n), \quad \alpha_i\geq 0,\quad \alpha_i \text{ integers.}
        \end{align*}
        Associated to a multi-index $\alpha$, we have the following symbols
        \begin{align*}
            |\alpha|&=\alpha_1+...+\alpha_n\\
            \alpha !&=\alpha_1!...\alpha_n!\\
            x^\alpha&=x_1^{\alpha_1}...x_n^{\alpha_n},\quad x\in\R^n
        \end{align*}
        We say that two multi-indices $\alpha$ and $\beta$ are related by $\alpha\leq\beta$ if $\alpha_i\leq\beta_i$ for all $1\leq i\leq n$. Finally we set
        \begin{align*}
            D^\alpha=\frac{\partial^{|\alpha|}}{\partial x_1^{\alpha_1}...\partial x_n^{\alpha_n}}
        \end{align*}
    \end{define}
    
    \begin{define}\ \\
        Let $T\in \D'(\R)$. If $T=T_f$, $f$ a $C^1$ function, then $f'$ is locally integrable. Thus for $\phi\in\D(\R)$
        \begin{align*}
            T_{f'}(\phi)=\int_\R f'\phi=-\int_\R f\phi '=-T_f(\phi ')
        \end{align*}
        Generalizing this, we define for any $T\in \D'(\R)$,
        \begin{align*}
            T'(\phi)=-T(\phi '),\quad \phi\in\D(\R)
        \end{align*}
        Thus $T'\in \D'(\R)$. For if $\phi_n\to 0$ in $\D(\R)$ then $\{\phi_n\}$ is also a sequence in $\D(\R)$ converging to zero. Hence $T'(\phi_n)=-T(\phi'_n)$ converges to zero. We can iterate this:
        \begin{align*}
            T''(\phi)=-T'(\phi')=T(\phi'')
        \end{align*}
        and more generally
        \begin{align*}
            T^{(k)}(\phi)=(-1)^kT(\phi^{(k)})
        \end{align*}
        In general if $T\in \D'(\Om)$, $\Om\subset \R^n$ an open set, then we define, for any multi-index $\alpha$, the distribution $D^\alpha T$ by 
        \begin{align*}
            (D^\alpha T)(\phi)=(-1)^{|\alpha|}T(D^\alpha \phi),\quad \phi\in\D(\Om)
        \end{align*}
    \end{define}
\newpage

\section{2.1 Definition and Basic Properties}
    \begin{note}\ \\
        Important properties of a class of function spaces which will provide proper functional setting for study of PDEs
    \end{note}
    
    \begin{define}[2.1.1]\ \\
        Let $m>0$ be an integer and let $1\leq p\leq \infty$. The \textbf{Sobolev space} $\W{m,p}{\Om}$ is defined by\\
        \begin{align*}
            \W{m,p}{\Om}=\set{u \in \LP{p}{\Om}}{D^\alpha u\in \LP{p}{\Om} \: \forall\: |\alpha|\leq m}
        \end{align*}
        We provide the norm
        \begin{align*}
            ||u||_{m,p,\Om}=\sum_{|\alpha|\leq m}||D^\alpha u||_{\LP{p}{\Om}}
        \end{align*}
        Equivalently $1<p<\infty$ (Note not equal equivalent)
        \begin{align*}
            ||u||_{m,p,\Om}=\bigg (\sum _{|\alpha|\leq m} \int _\Om |D^\alpha u|^p \bigg )^{1/p} =\bigg(\sum_{|\alpha|\leq m}||D^\alpha u||_{\LP{p}{\Om}}^p\bigg)^{1/p}
        \end{align*}
    \end{define}
    
    \begin{note}\ \\
        As new notation is added the logical extension of previous notation will be assumed.
        \begin{itemize}
            \item 
                In the case $p=2$, the spaces are Hilbert spaces with notation as follows,
                \begin{align*}
                    \HM{m}{\Om}&=\W{m,2}{\Om}\qquad \text{(scribbled in margin $\subset \W{m,p}{\Om}$)}\\
                    ||u||_{m,\Om}&=||u||_{m,2,\Om}
                \end{align*}
            \item 
                A \textit{semi-norm} consists of the $L^p$-norms of the highest order derivatives with notation as follows,
                \begin{align*}
                    |u|_{m,p,\Om}=\sum_{|\alpha|=m}||D^\alpha u||_{\LP{p}{\Om}}
                \end{align*}
            \item
                The space $\LP{p}{\Om}$ can be seen as a special case of the Sobolev class. We denote the $\LP{p}{\Om}$ norm as the semi-norm by $|\cdot|_{0,p,\Om}$ (as they are the same).
            \item 
                In $\HM{m}{\Om}$ there is a natural inner-product defined as
                \begin{align*}
                    (u,v)_{m,\Om}=\sum_{|\alpha|\leq m}\int_\Om D^\alpha uD^\alpha v, \quad \text{for } u,v\in \HM{m}{\Om}
                \end{align*}
            \item 
                When $\Om=\R^n$ the space $\HM{m}{\R^n}$ can be defined via the Fourier transform. Let $u\in \HM{m}{\R^n}$. By definition $D^\alpha u \in \LP{2}{\R^n}\quad \forall |\alpha|\leq m$. Hence the Fourier transform of $D^\alpha u$ is well defined as follows
                \begin{align*}
                    \widehat{D^\alpha u}=(2\pi i)^{|\alpha|} \xi ^\alpha \hat{u}.
                \end{align*}
                Thus $\xi ^\alpha \hat{u}\in \LP{2}{\R^n}$ for all $|\alpha|\leq m$. Conversely if $u\in \LP{2}{\R^n}\ni\xi ^\alpha \hat{u}\in \LP{2}{\R^n}$ for all $|\alpha|\leq m$, we have $D^\alpha u\in \LP{2}{\R^n}$ for all $|\alpha|\leq m$ and so $u\in \HM{m}{\R^n}$
        \end{itemize}
    \end{note}
    
    \begin{lemma}[2.1.1]\ \\
        There exists positive constants $M_1$ and $M_2$ depending only on $m$ and $n$ such that
        \begin{align*}
            M_1(1+|\xi|^2)^m\leq \sum_{|\alpha|\leq m}|\xi ^\alpha |^2\leq M_2(1+|\xi|^2)^m
        \end{align*}
        for all $\xi \in \R^n$
    \end{lemma}
    
    \begin{proof}\ \\
        Recall $|\xi|^2=\sum|\xi_i|^2$ and $|\xi^\alpha|=\sum |\xi_i^{\alpha_i}|$. We can see that the same powers of $\xi$ occur in $(1+|\xi|^2)^m$ and $\sum_{|\alpha|\leq m}|\xi^\alpha|^2$ with different coefficients that depend on $n$ and $m$. Since there are only finite number of terms, the inequalities follow.
    \end{proof}
    
    \begin{note}\ \\
        \begin{itemize}
            \item 
                From this lemma we see
                \begin{align*}
                    \HM{m}{\R^n}=\set{u\in \LP{2}{\R^n}}{(1+|\xi|^2)^{m/2}\hat{u}(\xi)\in \LP{2}{\R^n}}
                \end{align*}
                By the following theorem, it follows the norm $||\cdot||_{m,\R^n}$ on $\HM{m}{\R^n}$ is equivalent to the norm
                \begin{align*}
                    ||u||_{\HM{m}{\R^n}}=\bigg (\int_{\R^n}(1+|\xi|^2)^m|\hat{u}(\xi)|^2d\xi\bigg)^{1/2}
                \end{align*}
            \item
                We can generalize the previous norm to all $s\geq 0$. Define
                \begin{align*}
                    H^s(\R^n)&=\set{u\in \LP{2}{\R^n}}{(1+|\xi|^2)^m|\hat{u}(\xi)|^2\in \LP{2}{\R^n}}\\
                    ||u||_{H^s(\R^n)}&=\bigg (\int_{\R^n}(1+|\xi|^2)^s|\hat{u}(\xi)|^2d\xi\bigg)^{1/2}
                \end{align*}
            \item 
                The map
                \begin{align*}
                    u\in \W{1,p}{\Om}\to \bigg (u,\frac{\partial u}{\partial x_1},...,\frac{\partial u}{\partial x_n}\bigg )\in (\LP{p}{\Om})^{n+1}
                \end{align*}
                is an isometry of $\W{1,p}{\Om}$ into $(\LP{p}{\Om})^{n+1}$ provided the norm
                \begin{align*}
                    ||u||=\sum_{i=1}^{n+1}|u_i|_{0,p,\Om} \text{\quad or\quad } ||u||=\bigg (\sum_{i=1}^{n+1}|u_i|_{0,p,\Om}\bigg )^{1/p}
                \end{align*}
                for $u=(u_i)\in (\LP{p}{\Om})^{n+1}$
        \end{itemize}
    \end{note}
    
    \begin{thrm}[1.10.2](Plancherel)\\
        There exists a unique isometry
        \begin{align*}
            \mathcal{P}:\LP{2}{\R}\to \LP{2}{\R^n}
        \end{align*}
        which is onto and such that
        \begin{align*}
            \mathcal{P}(f)=\hat{f},\quad \text{for every }f\in \Sch
        \end{align*}
    \end{thrm}
    
    \begin{proof}\ \\
        The Fourier transform $\mathcal{F}:\Sch\to \Sch$ is an $\LP{2}{\R^n}$-isometry by corollary of Strong Parseval Relation ($\int_{\R^n}f\bar{g}=\int_{\R^n}\hat{f}\bar{\hat{g}}$). But $\Sch$ is dense in $\LP{2}{\R^n}$ and so $\mathcal{F}$ extends uniquely to an operator of $\LP{2}{\R^n}$ into itself. If $f\in \LP{2}{\R^n}$, then $f_k\in\Sch\ni f_k\to f$ in $\LP{2}{\R^n}$. Then
        \begin{align*}
            ||\mathcal{P}(f)||_{\LP{2}{\R^n}}&=\lim||\mathcal{P}(f_k)||_{\LP{2}{\R^n}}\\
            &=\lim ||\widehat{f_k}||_{\LP{2}{\R^n}} \text{\quad Copy of book hard to read}\\
            &=\lim ||f_k||_{\LP{2}{\R^n}}\\
            &=||f||_{\LP{2}{\R^n}}\\
        \end{align*}
        Thus $\mathcal{P}$ is an isometry and hence has a closed range. But as $\mathcal{F}$ is onto, $R(\mathcal{P})\supset \Sch$ which is dense in $\LP{2}{\R^n}$. Thus $\mathcal{P}$ is onto as well.
    \end{proof}
    
    \begin{thrm}[1.2.1]
        For every $1\leq p\leq \infty$, the space $\W{1,p}{\Om}$ is a Banach space. If $1<p<\infty$, it is reflexive and if $1\leq p< \infty$, it is separable. In particular $\HM{1}{\Om}$ is a separable Hilbert space.
    \end{thrm}
    
    \begin{proof}\ \\
        Let $\{u_m\}$ be a Cauchy sequence in $\W{1,p}{\Om}$. By definition of the norm, it follows that $\{u_m\}$ and $\bigg \{\frac{\partial u_m}{\partial x_i}\bigg \}, \quad 1\leq i\leq n$ are all Cauchy sequences in $\LP{p}{\Om}$. Let $u_m\to u$ and $\frac{\partial u_m}{\partial x_i}\to v_i,\quad 1\leq i\leq n$, in $\LP{p}{\Om}$. The completeness of $\W{1,p}{\Om}$ will be proved if we show that $\frac{\partial u}{\partial x_i}= v_i$ in the sense of distributions so that, on one hand, $u\in \W{1,p}{\Om}$ and on the other $u_m\to u$ in $\W{1,p}{\Om}$.\\
        Let $\phi \in \D(\Om)$. We need to show that
        \begin{align*}
            \int_\Om u\frac{\partial \phi}{\partial x_i}=-\int_\Om v_i\phi
        \end{align*}
        Now, since $u_m\in \W{1,p}{\Om}$, we know that
        \begin{align*}
            \int_\Om u_m\frac{\partial \phi}{\partial x_i}=-\int_\Om \frac{\partial u_m}{\partial x_i}\phi
        \end{align*}
        Since $\phi \in \D(\Om)$, $\phi \in \LP{q}{\Om}$ for all $1\leq q\leq \infty$ and so we pass the limit on both sides of the last set of integrals as $m\to \infty$ to obtain the first set of integrals. Thus $\W{1,p}{\Om}$ is complete and if $p=2$, we get that $\HM{1}{\Om}$ is a Hilbert space.\\
        Now $(\LP{p}{\Om})^{n+1}$ is reflexive for $1<p<\infty$ and separable for $1\leq p<\infty$. Since $\W{1,p}{\Om}$ is complete, its image under the aforementioned isometry is a closed subspace of $(\LP{p}{\Om})^{n+1}$ which inherits the corresponding properties.
    \end{proof}
    
    \begin{note}\ \\
        \begin{itemize}
            \item (2.1.1)\\
                The results of this theorem can be proved by the same way for any integer $m\geq 2$. In the future, unless absolutely necessary, we will establish theorems only for the spaces $\W{1,p}{\Om}$. The extensions to higher order spaces will often be obvious.
            \item (2.1.2)\\
                In the course of the proof of the preceding theorem we have proved the following fact: "if $u_m\to u$ in $\LP{p}{\Om}$ and $\frac{\partial u_m}{\partial x_i}\to v_i$ in $\LP{p}{\Om}$ for each $1\leq i\leq n$, then $u\in \W{1,p}{\Om}$ and $\frac{\partial u}{\partial x_i}=v_i$". Indeed, we can weaken the hypotheses even further. What was needed to pass to the limit in the second set of integrals and obtain the first set of integrals was only the weak convergence of $\bigg \{\frac{\partial u_m}{\partial x_i}\bigg \}$ (weak $\ast$ if $p=\infty$). Since bounded sequences have weakly convergent (weak $\ast$ convergent when $p=\infty$) subsequences, it is enough to know that $u_m\to u$ in $\LP{p}{\Om}$ and $\bigg \{\frac{\partial u_m}{\partial x_i}\bigg \}$ are bounded $(1<p\leq \infty)$ to deduce $u\in \W{1,p}{\Om}$.
            \item
                $\LP{p}{\Om}$ is made up of equivalence classes of functions. Thus by saying "$u$ is a continuous function" in $\LP{p}{\Om}$ we mean that the corresponding equivalence class has a continuous representative function. 
        \end{itemize}
    \end{note}
    
    \begin{thrm}[2.1.2]\ \\
        Let $I\subset \R$ be an open interval and let $u\in W^{1,p}(I)$. Then $u$ is absolutely continuous.
    \end{thrm}
    
    \begin{proof}\ \\
        Let $x_0\in I$ and define,
        \begin{align*}
            \bar{u}(x)=\int_{x_0}^x u'(t)dt
        \end{align*}
        which, by definition, is absolutely continuous. Hence its classical derivative exists a.e. and is equal a.e. to $u'$; this is also its distribution derivative. Hence, in the sense of distributions,
        \begin{align*}
            (u-\bar{u})'=0
        \end{align*}
        and so $u-\bar{u}=c$, a constant a.e.. Thus $u=\bar{u}+c$ a.e. and the latter function is absolutely continuous.
    \end{proof}
    
    \begin{note}\ \\
        \begin{itemize}
            \item 
                We can deduce an important property of $W^{1,p}(I)$ from the previous theorem when $I$ is a bounded interval. Consider $I=(0,1)$ and $u\in W^{1,p}(I)$. We can write
                \begin{align*}
                    u(x)=u(0)+\int_0^x u'(t)dt.
                \end{align*}
                By Hölder's inequality, if $q$ is conjugate to $p$, we have
                \begin{align*}
                    |u(0)|\leq |u(x)|+|u'|_{0,p,I}|x|^{1/q}
                \end{align*}
                It follows (on integration) that
                \begin{align*}
                    |u(0)|\leq C(|u|_{0,p,I}+|u'|_{0,p,I})=C||u||_{1,p,I}
                \end{align*}
                Where $C>0$ is a constant not depending on $u$.\\
                Using the integral equation we can deduce that for any $x\in I$
                \begin{align*}
                    |u(x)|\leq C||u||_{1,p,I}
                \end{align*}
                Let $B$ be the unit ball in $W^{1,p}(I)$ ($B=\set{u\in W^{1,p}(I)}{||u||_{1,p,I}\leq 1}$). It follows that if $i:W^{1,p}(I)\to C(I)$ is the inclusion map established in theorem 2.1.2 and continuous by the previous inequality then $B=i(B)$ is a uniformly bounded set in $C(I)$. Again if $x,y \in I$ by the integral equation, we have
                \begin{align*}
                    |u(x)-u(y)|\leq |u'|_{0,p,I}|x-y|^{1/q}\leq ||u||_{1,p,I}|x-y|^{1/q}
                \end{align*}
                which implies that $B$ is equicontinuous in $C(I)$. By the Ascoli-Arzela Theorem it follows that $B$ is relatively compact in $C(I)$; in other words, the map $i:W^{1,p}(I)\to C(I)$ is a compact operator. This will be studied later.
            \item
                If $1\leq p<\infty$, we know that $\D(\Om)$ is dense in $\LP{p}{\Om}$. Also, if $\phi\in\D(\Om)$ then every derivative of $\phi$ is also in $\D(\Om)\subset \W{m,p}{\Om}$ for any $m$ and $p$. We define $\WO{m,p}{\Om}$ as the closure of $\D$ in $\W{m,p}{\Om}$. Thus $\WO{m,p}{\Om}$ is a closed subspace and its elements can be approximated in the $\W{m,p}{\Om}$ norm by $C^\infty$ functions with compact support. Generally this is a strict subspace unless $\Om =\R^n$.
        \end{itemize}
    \end{note}
    
    \begin{thrm}[2.1.3]\ \\
        Let $1\leq p<\infty$. Then for any integer $m\geq 0$,
        \begin{align*}
            \W{m,p}{\R^n}=\WO{m,p}{\R^n}.
        \end{align*}
    \end{thrm}
    
    \begin{proof}\ \\
        This proof will be done for the $m=1$ case. We want to show that if $u\in W^{1,p}(\R^n)$, then there exists $\{\phi_k\}\subset \D(\R^n)\ni \phi_k\to u$ in $W^{1,p}(\R^n)$.\\
        Step 1\\
        Let $(\rho_\varepsilon)$ be the family of mollifiers. Then if $u\in \LP{p}{\R^n}$, we have $\rho_\varepsilon \ast u\to u$ in $\LP{p}{\R^n}$. First of all since $\rho_\varepsilon\in \LP{1}{\R^n}$, the convolution is well-defined and is in $\LP{p}{\R^n}$, by theorem 1.5.2 (below). Let $\phi$ be a continuous function with compact support such that $|\phi-u|_{0,p,\R^n}< \delta/3$ where $\delta>0$ is a pre-assigned number. Now choose $\varepsilon >0$ small enough such that $|\phi\ast \rho_\varepsilon-\phi|_{0,p,\R^n}<\delta/3$ (which is possible by theorem 1.5.6). Hence
        \begin{align*}
            |u-u\ast \rho_\varepsilon|_{0,p,\R^n}\leq |u-\phi|_{0,p,\R^n}+|\phi-\rho_\varepsilon\ast\phi|_{0,p,\R^n}+|\phi\ast \rho_\varepsilon-u\ast \rho_\varepsilon|_{0,p,\R^n}<\delta
        \end{align*}
        since, by theorem 1.5.2,
        \begin{align*}
            |(\phi-u)\ast \rho_\varepsilon|_{0,p,\R^n}\leq |\phi-u|_{0,p,\R^n}+| \rho_\varepsilon|_{0,1,\R^n}<\delta/3
        \end{align*}
        This proves the claim made in the beginning of this step.\\
        Step 2\\
        Now, if $u\in \W{1,p}{\R^n}$, then $u\ast \rho_\varepsilon$ is a $C^\infty$ function and $D^\alpha(u\ast \rho_\varepsilon)=D^\alpha u\ast \rho_\varepsilon=u\ast D^\alpha \rho_\varepsilon$ for any multi-index $\alpha$. By Step 1, $u\ast \rho_\varepsilon \to u$ and $D^\alpha (u\ast \rho_\varepsilon)=\rho_\varepsilon\ast D^\alpha u$ in $\LP{p}{\R^n}$. Hence $u\ast \rho_\varepsilon\to u$ in $\W{1,p}{\R^n}$.\\
        Step 3\\
        Let $\zeta$ be a function in $\D(\R^n)$ such that $0\leq \zeta \leq 1$, $\zeta \equiv 1$ on $B(0;1)$ and supp$(\zeta)\subset B(0;2)$. We consider the sequence $\{\zeta_k\}$ in $\D (\R^n)$ defined by
        \begin{align*}
            \zeta_k(x)=\zeta(x/k)
        \end{align*}
        Let $\varepsilon_k \downarrow 0$. Set $u_k=\rho_{\varepsilon_k}\ast u$. Then $u_k\in \E(\R^n)$ and $u_k\to u$ in $\W{1,p}{\R^n}$ by Step 2. Now define $\phi_k\in\D(\R^n)$ by
        \begin{align*}
            \phi_k (x)=\zeta_k(x)u_k(x).
        \end{align*}
        We now show that $\phi_k\to u$ in $\W{1,p}{\R^n}$, which will complete the proof. Since $\zeta_k\equiv 1$ on $B(0;k)$ we have $u_k=\phi_k$ on $B(0;k)$. Hence
        \begin{align*}
            |u_k-\phi_k|_{0,p,\R^n}=\bigg (\int_{|x|>k}|u_k(x)-\phi_k(x)|^pdx\bigg )^{1/p}\leq 2\bigg (\int_{|x|>k}|u_k|^p\bigg )^{1/p}
        \end{align*}
        since $|\phi_k|\leq |u_k|$. Now the latter integral tends to zero as $k\to\infty$ since $u_k\to u$ in $\LP{p}{\R^n}$. Thus $\phi_k\to u$ in $\LP{p}{\R^n}$. Similarly, since
        \begin{align*}
            \frac{\partial \phi_k}{\partial x_i}=\frac{\partial u_k}{\partial x_i}
        \end{align*}
        on $b(0;k)$, we get $\frac{\partial \phi_k}{\partial x_i}\to\frac{\partial u}{\partial x_i}$ in $\LP{p}{\R^n}$ by an analogous argument. Thus $\phi_k\to u$ in $\W{1,p}{\R^n}$.
    \end{proof}
    
    \begin{thrm}[1.5.2]\ \\
        Let $1<p<\infty$ and $f\in\LP{1}{\R^n},g\in\LP{p}{\R^n}$. The $f\ast g$ is well-defined and further, $f\ast g\in\LP{p}{\R^n}$ with
        \begin{align*}
            ||f\ast g||_{\LP{p}{\R^n}}\leq ||f||_{\LP{1}{\R^n}}||g||_{\LP{p}{\R^n}}
        \end{align*}
    \end{thrm}
    
    \begin{proof}\ \\
        Let $q$ be the conjugate exponent of $p$. Let $h\in \LP{q}{\R^n}$. Then $(x,y)\mapsto f(x-y)g(y)h(x)$ is measurable and
        \begin{align*}
            \int_{\R_x^n}\int_{\R_y^n}|f(x-y)g(y)h(x)|dydx&=\int_{\R_x^n}|h(x)|\int_{\R_y^n}|f(x-y)g(y)|dydx\\
            &=\int_{\R_x^n}|h(x)|\int_{\R_t^n}|f(t)g(x-t)|dtdx\\
            &=\int_{\R_t^n}|f(t)|dt\int_{\R_x^n}|h(t)||g(x-t)|dx\\
            &\leq ||h||_{\LP{q}{\R^n}}||g||_{\LP{p}{\R^n}}||f||_{\LP{1}{\R^n}}\\
            &<+\infty
        \end{align*}
        (Holder's and translation invariance).\\
        Thus by Fubini's theorem
        \begin{align*}
            \int_{\R_y^n}|f(x-y)g(y)h(x)|dy
        \end{align*}
        exists for almost all $x$ and since we can choose $h(x)\neq 0$ for all $x$ we deduce that $f\ast g$ is well defined. Also
        \begin{align*}
            h\to \int (f\ast g)h
        \end{align*}
        is a continuous linear functional on $\LP{q}{\R^n}$ with norm bounded by $||g||_{\LP{p}{\R^n}}||f||_{\LP{1}{\R^n}}$ which shows, by the Riesz Representation Theorem, that $f\ast g\in\LP{p}{\R^n}$ and $||f\ast g||_{\LP{p}{\R^n}}\leq ||f||_{\LP{1}{\R^n}}||g||_{\LP{p}{\R^n}}$.
    \end{proof}
    
    \begin{thrm}\ \\
        Let $1\leq p< \infty$. Then $\D(\R^n)$ is dense in $\LP{p}{\R^n}$.
    \end{thrm}
    
    \begin{proof}\ \\
        Step 1\\
        Let $S$ be the class of measurable simple functions $\phi$ such that
        \begin{align*}
            \mu\{\phi (x)\neq 0\}<+\infty,
        \end{align*}
        $\mu$ being the Lebesgue measure. If $\phi \in S$ then $\phi$ is zero outside a set of finite measure and hence $\phi\in\LP{p}{\R^n}$. Let $f\in\LP{p}{\R^n}$ and $f\geq 0$. Then there exist simple functions $\phi_m$ such that $0\leq \phi_m\leq f,\quad \phi_m\uparrow f$. Since $f\in\LP{p}{\R^n}$, it follows that $\phi_m\in S$. Also $|\phi_m-f|^p\leq 2^p|f|^p$ and since $|f|^p$ is integrable, by the Dominated Convergence Theorem it follow that $\phi_m\to f$ in $\LP{p}{\R^n}$. If $f$ is any element in $\LP{p}{\R^n}$, write $f=f^+-f^-$ with $f^+\geq 0, f^-\geq 0$ and so there exist $\phi_m,\psi_m\in S$ with $\phi_m\to f^+,\psi_m\to f^-$ in $\LP{p}{\R^n}$ and so $\phi_m-\psi_m\to f$ in $\LP{p}{\R^n}$. Thus $S$ is dense in $\LP{p}{\R^n}$.\\
        Step 2\\
        Let $\phi \in S$. Given $\varepsilon >0$, there exists a continuous function $g$ with compact support such that $g=\phi$ except possibly on a set of measure $\varepsilon$ and such that 
        \begin{align*}
            |g|\leq ||\phi||_{\LP{\infty}{\R^n}}
        \end{align*}
        (This is a consequence of Lusin's Theorem.) Hence
        \begin{align*}
            ||g-\phi||_{\LP{p}{\R^n}}\leq 2\varepsilon^{1/p}||\phi ||_{\LP{\infty}{\R^n}}
        \end{align*}
        and it follows from Step 1 above that continuous functions with compact support are dense in $\LP{p}{\R^n}$.\\
        Step 3\\
        Let $f$ be a continuous function with compact support an let $\phi_m \in\D(\R^n)$ with supp$(f)\subset K$, $K$ a fixed compact set, such that $\phi_m\to f$ uniformly on $K$. Then
        \begin{align*}
            \int_{\R^n}|f-\phi_m|^p=\int_K|f-\phi_m|^p\leq \sup_k|f-\phi_m|\mu(K)\to 0
        \end{align*}
        as $m\to\infty$. Thus $\phi_m$ approximates $f$ in the $L^p$-norm as well and the result now follows on combining this fact with the conclusion of Step 2.
    \end{proof}
\newpage

\section{2.3 Extension Theorems}
    \begin{define}\ \\
        Let $x\in\R^n,\quad x=(x_1,...,x_n).$ We set $x'=(x_1,...,x_{n-1
        })$ and write $x=(x',x_n)$.\\
        Define the sets
        \begin{align*}
            Q_+&=\set{x\in\R^n}{|x'|<1,0<x_n<1}\\
            Q&=\set{x\in\R^n}{|x'|<1,|x_n|<1}
        \end{align*}
        Where $|x'|$ is the Euclidean norm of $|x'|$ in $\R^{n-1}$
    \end{define}
    
    \begin{define}[2.3.1]\ \\
        We say that an open set $\Om$ is of class $C^k$ ($k$ an integer $\geq$ 1) if for every $x\in\partial\Om$, there exists a neighborhood $U$ of $x$ in $\R^n$ and a map $T:Q\to U$ such that
        \begin{itemize}
            \item $T$ is a bijection
            \item $T\in C^k(\bar{Q}),\quad T^{-1}\in C^k(\bar{U})$
            \item $T(Q_+)=U\cap\Om,\quad T(Q_0)=U\cap\partial\Om$
        \end{itemize}
        Where $Q_+,Q$ are defined above and
        \begin{align*}
            Q_0=\set{x\in Q}{ x_n=0}
        \end{align*}
        We say that $\Om$ is of class $C^\infty$ if it is of class $C^k$ for every integer $k\geq 1$.
    \end{define}
    
    \begin{cor}\ \\
        If $\Om$ is of class $C^1$ and has $\partial\Om$ bounded, then $C^\infty(\bar{\Om})$ is dense in $\W{1,p}{\Om}$, $1\leq p<\infty$.
    \end{cor}
    
    \begin{thrm}[2.3.4 (\textbf{Poincaré's Inequality}).]\ \\
        Let $\Om$ be a \textit{bounded} open set in $\R^n$. Then there exists a positive constant $C=C(\Om,p)$ such that
        \begin{align*}
            |u|_{0,p,\Om}\leq C|u|_{1,p,\Om}\quad \text{for every } u\in \WO{1,p}{\Om}.
        \end{align*}
        In particular, $u\to |u|_{1,p,\Om}$ defines a \textit{norm} on $\WO{1,p}{\Om}$, which is equivalent to the norm $||\cdot||_{1,p,\Om}$. On $H^1_0(\Om)$, the bilinear form
        \begin{align*}
            (u,v)\mapsto \int_\Om \sum_{i=1}^n\frac{\partial u}{\partial x_l}\frac{\partial v}{\partial x_i},
        \end{align*}
        defines an inner-product giving rise to the norm $|\cdot|_{1,\Om}$, equivalent to the norm $||\cdot||_{l,\Om}$.
    \end{thrm}
    
    \begin{proof}\ \\
        Let $\Om=(-a,a)^n,\: a>0$. Let $u\in\D(\Om)$. Then
        \begin{align*}
            u(x)=\int_{-a}^{x_n}\frac{\partial u}{\partial x_n}(x',t)dt,\:x=(x',x_n), \text{ since } u(x',-a)=0
        \end{align*}
        Hence
        \begin{align*}
            |u(x)|\leq \bigg (\int_{-a}^{x_n}\bigg |\frac{\partial u}{\partial x_n}(x',t)\bigg |^p dt\bigg )^{1/p}|x_n+a|^{1/q}
        \end{align*}
        Where $p$ and $q$ are Hölder conjugates. Or,
        \begin{align*}
            |u(x)|^p\leq |x_n+a|^{p/q}\int_{-a}^{x_n}\bigg |\frac{\partial u}{\partial x_n}(x',t)\bigg |^p dt
        \end{align*}
        Integrating over $x'$,
        \begin{align*}
            \int|u(x',x_n)|^p dx'\leq (2a)^{p/q}\int_\Om \bigg |\frac{\partial u}{\partial x_n}\bigg |^p,
        \end{align*}
        and so
        \begin{align*}
            \int_\Om |u(x)|^p dx\leq (2a)^{p/q+1}\int_\Om \bigg |\frac{\partial u}{\partial x_n}\bigg |^p
        \end{align*}
        which proves the inequality in the theorem for $u\in\D(\Om)$. But $\D(\Om)$ is dense in $\WO{1,p}{\Om}$ and the inequality follows for all $u\in \WO{1,p}{\Om}$. If $\Om$ is not a 'box' let $\Om\subset \tilde{\Om}$, box of the form $(-a,a)^n$ and extend $u\in \WO{1,p}{\Om}$ by zero to get $\tilde{u}\in \WO{1,p}{\tilde{\Om}}$ and apply the result which is available for $\tilde{\Om}$ and the inequality follows.
    \end{proof}
\newpage

\section{2.4 Embedding Theorems}
    \begin{thrm}[2.4.5]\ \\
        Let $m\geq 1$ be an integer and $1\leq p<\infty$. Then
        \begin{itemize}
            \item if $\frac{1}{p}-\frac{m}{n}>0,\: \W{m,p}{\Om}\subset \LP{q}{\Om},\:\frac{1}{q}=\frac{1}{p}-\frac{m}{n},$
            \item if $\frac{1}{p}-\frac{m}{n}=0,\: \W{m,p}{\Om}\subset \LP{q}{\Om},\text{ for } q\in[p,\infty)$,
            \item if $\frac{1}{p}-\frac{m}{n}<0,\: \W{m,p}{\Om}\subset L^\infty(\Om),$
        \end{itemize}
        and in the latter case, i.e. when $m>(n/p)$ if we set
        \begin{align*}
            k=\bigg [m-\frac{n}{p}\bigg],\:\theta=\bigg (m-\frac{n}{p}\bigg )-k,
        \end{align*}
        What?\\
        $[.]$, denoting the integral part of a real number, we have
        \begin{align*}
            |D^\alpha u|_{0,\infty,\R^n}\leq C||u||_{m,p,\R^n}\quad \text{for } |\alpha|\leq k
        \end{align*}
        and
        \begin{align*}
            |D^\alpha u(x)-D^\alpha (y)|\leq C|x-y|^\theta||u||_{m,p,\R^n}\text{ a.e. } (x,y),
        \end{align*}
        for $|\alpha|=k$. (If $|\alpha|<k$, the previous inequality holds with $\theta=1$ by virtue of the previous inequality. In particular we have the continuous inclusion
        \begin{align*}
            \W{m,p}{\Om}\to C^k(\Om),\:m>(n/p).
        \end{align*}
        No proof given
    \end{thrm}
\newpage


\section{2.5 Compactness Theorems}

    \begin{define}\ \\
        Let $1\leq p< n.$ Then we define the exponent $p^*$ by
        \begin{align*}
            \frac{1}{p^*}=\frac{1}{p}-\frac{1}{n}\text{ or } p^*=\frac{np}{n-p}
        \end{align*}
    \end{define}
    
    \begin{thrm}[2.5.3 (\textbf{Rellich-Kondrasov})]\ \\
        Let $\Om\subset\R^n$ be a bounded open set of class $C^1$. Then the following inclusions are compact.
        \begin{itemize}
            \item if $p<n,\:\W{1,p}{\Om}\to\LP{q}{\Om},\quad 1\leq q <p^*$,
            \item if $p=n,\:\W{1,n}{\Om}\to\LP{q}{\Om},\quad 1\leq q <\infty$,
            \item if $p>n,\:\W{1,p}{\Om}\to C(\bar{\Om})$.
        \end{itemize}
    \end{thrm}
    
    \begin{proof}\ \\
        When $p>n$, as observed earlier the functions of $\W{1,p}{\Om}$ are Holder continuous. If $B$ is the unit ball in $\W{1,p}{\Om}$, it follows from theorem 2.4.4 and the analogue of the inequality (2.4.17) that the functions in $B$ are uniformly bounded and equicontinuous in $C(\bar{\Om})$. Thus $B$ is relatively compact in $C(\bar{\Om})$ by the Ascoli-Arzela Theorem.\\
        Assume for the moment that the result is true for $p<n$. Notice that as $p\to n$, $p^*\to\infty$. Hence, since $\Om$ is bounded $\W{1,n}{\Om}\subset \W{1,n-q}{\Om}$ for every $\varepsilon>0$ and given any $q<\infty$ we can find $\varepsilon>0$ such that $1\leq q<(n-\varepsilon)^*$. Hence by using the case for $p=n-\varepsilon<n$, we deduce that $\W{1,n}{\Om}$ is compactly embedded in $\LP{q}{\Om}$ for any $1\leq q<\infty$.\\
        Thus the theorem will be proved if we prove it for the case $p<n$. Let $B$ be the unit ball in $\W{1,p}{\Om}$. We now verify the first and second conditions of  the previous theorem (2.5.2. not typed out). Let $1\leq q<p^*$. Then choose $\alpha$ such that $0<\alpha\leq 1$ and
        \begin{align*}
            \frac{1}{q}=\frac{\alpha}{1}+\frac{1-\alpha}{p^*}.
        \end{align*}
        Then (as in the corollary to Theorem 2.4.1 (not typed out)) if $u\in B, \Om'\subset\subset\Om$ and $h\in\R^n$ such that $|h|<\text{dist}(\Om',\R^n \backslash \Om),$
        \begin{align*}
            |\tau_{-h}u-u|_{0,p,\Om'}&\leq |\tau_{-h}u-u|_{0,p,\Om'}^\alpha |\tau_{-h}u-u|_{0,p,\Om'}^{1-\alpha}\\
            &\leq (|h|^\alpha |u|_{1,1,\Om}^\alpha)(2|u|_{0,p^*,\Om})^{1-\alpha}\\
            &\leq C|h|^\alpha
        \end{align*}
        using Lemma 2.5.1 (not typed). We choose $h$ small enough such that $C|h|^\alpha<\varepsilon$. This will verify 2.5.5 ($|\tau_{-h}f-f|_{0,p,\Om'}<\varepsilon$).\\
        Now if $u\in B$ and $\Om'\subset \subset \Om$, it follows by Holder's inequality that
        \begin{align*}
            |u|_{0,q,\Om\backslash\bar{\Om}'}&\leq |u|_{0,p^*,\Om\backslash\bar{\Om}'}\:\text{meas}(\Om\backslash\bar{\Om}')^{1-(q/p^*)}\\
            &\leq C\:\text{meas}(\Om\backslash\bar{\Om}')^{1-(q/p^*)}
        \end{align*}
        which can be made to be less than any given $\varepsilon>0$ by choosing $\Om'\subset\subset\Om$ to be 'as closely filling $\Om$' as needed. This verifies (2.5.6 ( $|f|_{0,p,\Om\backslash\bar{\Om}'}<\varepsilon\quad\text{for every }f\in\mathscr{F}$ bounded set in $\LP{p}{\Om}$)). Thus $B$ is relatively compact in $\LP{q}{\Om}$ for $1\leq q<p^*$ and the theorem is proved.
    \end{proof}
    
\newpage

\section{2.6 Dual Spaces, Fractional Order Spaces and Trace Spaces}

    \begin{define}[2.6.1]\ \\
        Let $1\leq p<\infty$. Let $p'$ be the conjugate exponent of $p$. The dual of the space $\WO{m,p}{\Om}$ where $m\geq 1$ is an integer, is denoted by $\W{-m,p'}{\Om}$. If $p=2, \HM{-m}{\Om}$ is the dual of the space $H_0^m(\Om)$. 
    \end{define}
    
    \begin{thrm}[2.6.1]\ \\
        Let $F\in\W{-1,p'}{\Om}$. Then there exist functions $f_0,f_1,...,f_n\in \LP{p'}{\Om}$ such that
        \begin{align*}
            F(v)=\int_\Om f_0v+\sum_{i=1}^n\int_\Om f_i \frac{\partial v}{\partial x_i},\:v\in \WO{1,p}{\Om}
        \end{align*}
        and
        \begin{align*}
            ||F||=\max_{0\leq i\leq n}|f_i|_{0,p',\Om}
        \end{align*}
        Further, if $\Om$ is bounded, we assume $f_0=0$.
    \end{thrm}
    
    \begin{define}[1.9.1]\ \\
        The \textbf{Schwartz Space}, or the space of rapidly decreasing functions, $\Sch$, is given by
        \begin{align*}
            \Sch=\set{f\in\E(\R^n)}{\lim_{|x|\to\infty}|x^\beta D^\alpha f(x)|=0\quad \text{for all multi-indices $\alpha$ and $\beta$}}.
        \end{align*}
    \end{define}
    
    \begin{thrm}[2.6.2]\ \\
        (This was switched from $\R^n$ to $\Om$) Let $s>0$ be a real number. Then
        \begin{align*}
            \HM{-s}{\Om}=\set{u\in \Sch'(\Om)}{(1+|\xi|^2)^{-s/2}\hat{u}(\xi)\in\LP{2}{\Om}}
        \end{align*}
    \end{thrm}
    
    \begin{thrm}[2.6.3]\ \\
        (This was switched from $\R^n$ to $\Om$) Let $s_1<s_2$ and $s=\theta s_1+(1-\theta)s_2,\quad \theta\in(0,1)$. If $u\in\HM{s_2}{\Om}$, then
        \begin{align*}
            ||u||_{\HM{s}{\Om}}\leq ||u||^\theta_{\HM{s_1}{\Om}}||u||^{1-\theta}_{\HM{s_2}{\Om}}
        \end{align*}
    \end{thrm}
    
    \begin{note}[2.6.2]\ \\
        (This was switched from $\R^n$ to $\Om$) Let $s_1<s_2$ and $t_1<t_2$. Let $\theta\in(0,1)$ and set $s=\theta s_1+(1-\theta)s_2,\:t=\theta t_1+(1-\theta)t_2$. Assume that $T$ is a linear operator such that
        \begin{align*}
            T\in\Lin(\HM{s_1}{\Om},\HM{t_1}{\Om})\cap \Lin(\HM{s_2}{\Om},\HM{t_2}{\Om})
        \end{align*}
        Then it is true that $T\in \Lin(\HM{s}{\Om},\HM{t}{\Om})$ and we also have
        \begin{align*}
            ||T||_{\Lin(\HM{s}{\Om},\HM{t}{\Om})}\leq ||T||_{\Lin(\HM{s_1}{\Om},\HM{t_1}{\Om})}^\theta ||T||_{\Lin(\HM{s_2}{\Om},\HM{t_2}{\Om})}^{1-\theta}
        \end{align*}
    \end{note}
    
    \begin{define}
        Let $V=(\HM{1}{\Om})^3$, where $\Om\subset \R^3$ is a bounded open set of class $C^1$. If $v\in V$, let $v=(v_1,v_2,v_3)$ be its components. For $1\leq i,j\leq 3$ we define
        \begin{align*}
            \epsilon_{ij}(v)=\frac{1}{2}\bigg (\frac{\partial v_i}{\partial x_j}+\frac{\partial v_j}{\partial x_i}\bigg ).
        \end{align*}
        We denote by $||\cdot||_V$ the usual product norm on $V$.
    \end{define}
    
    \begin{thrm}[2.6.5 (\textbf{Korn's Inequality})]\ \\
        Let $\Om$ be a bounded open subset of $\R^3$, of class $C^1$. Then there exists a constant $C>0$, depending only on $\Om$, such that
        \begin{align*}
            \int_\Om \sum_{i,j=1}^3|\epsilon_{ij}(v)|^2+\int_\Om \sum_{i=1}^3|v_i|^2\geq C||v||_V^2
        \end{align*}
        for every $v\in V$.
    \end{thrm}

\newpage

\section{2.7 Trace Theory}

    \begin{thrm}[2.7.1]\ \\
        (This was changed from $\R$ case to $\Om$) Let $\Om\subset\R^n$ be bounded. Then there exists a continuous linear map $\gamma_0:\HM{1}{\Om}\to \LP{2}{\R^{n-1}}$ which is such that if $v$ is continuous on $\Om$ then
        \begin{align*}
            \gamma_0(v)=v|_{\R^{n-1}}
        \end{align*}
    \end{thrm}
    
    \begin{thrm}[2.7.2]\ \\
        The range of the map $\gamma_0$ is the space $\HM{1/2}{\R^{n-1}}$.
    \end{thrm}
    
    \begin{note}[2.7.1]\ \\
        \begin{itemize}
            \item 
                (This was changed from $\R$ case to $\Om$) Similarly we can prove that $\gamma_0$ maps $\HM{m}{\Om}$ onto $\HM{m-1/2}{\R^{n-1}}$. Also if $u\in\HM{2}{\Om}$ we can show that $\frac{\partial u}{\partial x_n}(x',0)$ is in $\LP{2}{\R^{n-1}}$ and again $\frac{\partial u}{\partial x_n}(x',0)\in\HM{1/2}{\R^{n-1}}$. We can then extend $-\frac{\partial u}{\partial x_n}(x',0)$ to a continuous linear map $\gamma_1:\HM{2}{\Om}\to\LP{2}{\R^{n-1}}$ whose range is $\HM{1/2}{\R^{n-1}}$. More generally we have a series of continuous linear maps $\{\gamma_i\}$ into $\LP{2}{\R^{n-1}}$. such that the map $\gamma=(\gamma_0,\gamma_1,...,\gamma_{m-1})$ maps $\HM{m}{\Om}$ into $(\LP{2}{\R^{n-1}})^m$ and the range in the space
                \begin{align*}
                    \prod_{j=0}^{m-1}\HM{m-j-1/2}{\R^{n-1}}
                \end{align*}
            \item 
                Looking at the kernel of the map $\gamma_0$. If $u$ is continuous on $\bar{\Om}$ and $u$ vanishes on $\Gamma$, then $u\in H_0^1(\Om)$. A few steps not included here lead to $H_0^1(\Om)=\ker (\gamma_0)$
        \end{itemize}
    \end{note}
    
    \begin{thrm}[2.7.4 (\textbf{Trace Theorem})]\ \\
        Let $\Om\subset \R^n$ be a bounded open set of class $C^{m+1}$ with boundary $\Gamma$. Then there exists a trace map $\gamma=(\gamma_0,\gamma_1,...,\gamma_{m-1})$ from $\HM{m}{\Om}$ into $(\LP{2}{\Om})^m$ such that
        \begin{itemize}
            \item 
                If $v\in C^\infty (\bar{\Om})$, then $\gamma_0(v)=v|_\Gamma,\:\gamma_1(v)=\frac{\partial v}{\partial \nu}|_\Gamma$,..., and $\gamma_{m-1}(v)=\frac{\partial^{m-1}}{\partial \nu^{m-1}}(v)|_\Gamma$, where $\nu$ is the unit exterior normal to the boundary $\Gamma$.
            \item 
                The range of $\gamma$ is the space
                \begin{align*}
                    \prod_{j=0}^{m-1}\HM{m-j-1/2}{\Gamma}
                \end{align*}
            \item 
                The kernel of $\gamma$ is $H_0^m(\Om)$.
        \end{itemize}
    \end{thrm}
    
    \begin{thrm}[2.7.5 (\textbf{Green's Theorem}, or, \textbf{Green's Formula})]\ \\
        Let $\Om$ be a bounded open set of $\R^n$ set of class $C^1$ lying on the same side of its boundary $\Gamma$. Let $u,v\in\HM{1}{\Om}$. Then for $1\leq i\leq n$,
        \begin{align*}
            \int_\Om u\frac{\partial v}{\partial x_i}=-\int_\Om \frac{\partial u}{\partial x_i}v+\int_\Gamma (\gamma_0u)(\gamma_0v)\nu_i.
        \end{align*}
    \end{thrm}
    
\newpage

\section{Bibliography}
\printbibliography

\end{document}
