\documentclass{article}

\usepackage{amsmath,amssymb,bigints,amsthm,mathrsfs,mathtools,biblatex}

\addbibresource{Bibliography.bib}

\newtheorem{thrm}{Theorem}
\newtheorem{cor}{Corollary}
\newtheorem{lemma}{Lemma}
\newtheorem{define}{Definition}
\newtheorem{prop}{Proposition}
\newtheorem{note}{Remark}

\newcommand{\R}{\mathbb{R}}
\newcommand{\D}{\mathscr{D}}
\newcommand{\E}{\mathscr{E}}
\newcommand{\Sch}{\mathcal{S}}
\newcommand{\Lin}{\mathscr{L}}
\newcommand{\Om}{\Omega}
\newcommand{\W}[2]{W^{#1}(#2)}
\newcommand{\WO}[2]{W_0^{#1}(#2)}
\newcommand{\HM}[2]{H^{#1}(#2)}
\newcommand{\LP}[2]{L^{#1}(#2)}
\newcommand{\set}[2]{\{#1\:|\:#2\}}

\title{C_0 Semigroups}
\author{Jacob Manning}
\date{February 2025}

\begin{document}
\tableofcontents
\newpage

\section{Notation}

    \begin{define}[4.1.1 Kreyszig]\ \\
        \begin{itemize}
            \item A linear operator $A:D(A)\subseteq X\to Y$ is said to be \textbf{bounded} if there exists a $C>0$ such that
            \begin{align*}
                ||Au||_Y\leq C||u||_X,\quad \text{for every } u\in D(A)
            \end{align*}
            Otherwise it is said to be \textbf{unbounded}.
            \item A linear operator $A:D(A)\subseteq X\to Y$ is said to be \textbf{densely defined} if $\overline{D(A)}=X$
            \item A linear operator $A:D(A)\subseteq X\to Y$ is said to be \textbf{closed} if the \textbf{graph}
            \begin{align*}
                G(A)=\set{(u,Au)}{u\in D(A)}\subseteq X\times Y
            \end{align*}
            is closed as a subspace of $X\times Y$
        \end{itemize}
    \end{define}

    \begin{define}[7.2 Kreyszig]\ \\
        Let $X\neq \{0\}$ be a complex normed space and $T:\D(T)\subseteq X\to X$ be a linear operator. With $T$ we associate the operator
        \begin{align*}
            T_\lambda =T-\lambda I
        \end{align*}
        where $\lambda$ is a complex number and $I$ is the identity operator on $\D(T)$. If $T_\lambda$ has an inverse, we denote it by $R_\lambda (T)$ and call it the \textit{resolvent operator} of $T$ or, simply, the \textbf{resolvent} of $T$. If it is clear which operator we are discussing, we will write $R_\lambda$.
    \end{define}

    \begin{define}[7.2-1 (Regular value, resolvent set, spectrum)]\ \\
        Let $X\neq \{0\}$ be a complex normed space and $T:\D(T)\subseteq X\to X$ be a linear operator. A \textit{regular value} $\lambda$ of $T$ is a complex number such that
        \begin{itemize}
            \item $R_\lambda (T)$ exists,\\
            \item $R_\lambda (T)$ is bounded,\\
            \item $R_\lambda (T)$ is densely defined.
        \end{itemize}
        The \textit{resolvent set} $\rho(T)$ of $T$ is the set of all regular values $\lambda$ of $T$. Its complement $\sigma (T)=\mathbb{C}\backslash\rho(T)$ in the complex plane $\mathbb{C}$ is called the \textit{spectrum} of $T$, and a $\lambda \in \sigma (T)$ is called a \textit{spectral value} of $T$. Furthermore, the spectrum $\sigma(T)$ is partitioned into three disjoint sets as follows.\\
        The \textbf{point spectrum} or \textit{discrete spectrum} $\sigma_p(T)$ is the set such that $R_\lambda(T)$ does not exist. A $\lambda\in\sigma_p(T)$ is called an \textit{eigenvalue} of $T$.\\
        The \textbf{continuous spectrum} $\sigma_c(T)$ is the set such that $R_\lambda (T)$ exists and is densely defined, but it unbounded.\\
        The \textbf{residual spectrum} $\sigma_r(T)$ is the set such that $R_\lambda(T)$ exists, but is not densely defined (may or may not be bounded).
    \end{define}

\newpage

\section{3.1 Weak Solutions of Elliptic Boundary Value Problems}
    \begin{thrm}[3.1.4 (Lax-Milgram)]\ \\
        Let $V$ be a Hilbert space and $a(\cdot,\cdot)$ a continuous $V$-elliptic bilinear form. Then given $f\in V$, there exists a unique $u\in V$ such that
        \begin{align*}
            a(u,v)=(f,v),\quad \text{for every }v\in V.
        \end{align*}
        If $a(\cdot,\cdot)$ is also symmetric then the functional $J:V\to \R$ defined by
        \begin{align*}
            J(v)=\frac{1}{2}a(v,v)-(f,v)
        \end{align*}
        attains its minimum at $u$.
    \end{thrm}

    \begin{thrm}[3.1.5 (Babuska-Brezzi)]\ \\
        Let $\Sigma,\:V$ be Hilbert spaces and $a:\Sigma\times\Sigma\to\R,\:b:\Sigma\times V\to\R$, bilinear forms which are continuous. Let 
        \begin{align*}
            Z=\set{\sigma\in\Sigma}{b(\sigma,v)=0,\quad\text{for every }v\in V}.
        \end{align*}
        Assume that $a(\cdot,\cdot)$ is $Z$-elliptic, i.e. there exists a constant $\alpha>0$ such that
        \begin{align*}
            a(\sigma,\sigma)\geq \alpha||\sigma||_\Sigma^2,\quad\text{for every }\sigma\in Z.
        \end{align*}
        Assume further that there exists a constant $\beta>0$ such that
        \begin{align*}
            \sup_{\tau\in\Sigma}\frac{b(\tau,v)}{||\tau||_\Sigma}\geq \beta||v||_V.
        \end{align*}
        Then if $\kappa\in\Sigma$ and $l\in V$, there exists a unique pair $(\sigma,u)\in\Sigma\times V$ such that
        \begin{align*}
            a(\sigma,\tau)+b(\tau,u)=(\kappa,\tau),\quad\text{for every }\tau\in\Sigma\\
            b(\sigma,v)=(l,v),\quad\text{for every }v\in V.
        \end{align*}
    \end{thrm}
\newpage

\section{4.3 $C_0$ Semigroups}

    \begin{define}[4.3.1]\ \\
            Let $X$ be a Banach space and $\{S(t)\}_{t\geq 0}$ be a family of bounded linear operators on $X$. It is said to be a $C_0$ semigroup if the following are true:
            \begin{itemize}
                \item $S(0)=I$, the identity of $X$\\
                \item $S(t+s)=S(t)S(s)$, for all $t,s\geq 0$\\
                \item For every $u\in X$
                    \begin{align*}
                        S(t)u\to u \quad \text{as } t\downarrow 0
                    \end{align*}
            \end{itemize}
        \end{define}
    
    \begin{thrm}[4.3.1]\ \\
        Let $\{S(t)\}_{t\geq 0}$ be a $C_0$-semigroup on $X$. Then there exists $M\geq 1$ and $\omega$ such that
        \begin{align*}
            ||S(t)||\leq Me^{\omega t},\quad \text{for all } t\geq 0
        \end{align*}
    \end{thrm}
    
    \begin{define}[4.3.2]\ \\
        If $M=1$ and $\omega = 0$, so that $||S(t)||\leq 1$ for all $t\geq 0$, we say that $\{S(t)\}$ is a \textbf{contraction semigroup}.
    \end{define}
    
    \begin{define}[4.3.3]\ \\
        Let $\{S(t)\}_{t\geq 0}$ be a $C_0$ semigroup on $X$. The \textbf{infinitesimal generator} of the semigroup is a linear operator $A$ given by
        \begin{align*}
            D(A)&=\bigg \{u\in X\:|\: \lim_{t\downarrow0} \frac{S(t)u-u}{t} \:\text{exists}\bigg \}\\
            Au&=\lim_{t\downarrow0} \frac{S(t)u-u}{t},\: u\in D(A)
        \end{align*}
    \end{define}
    
    \begin{thrm}[4.3.2]\ \\
        Let $\{S(t)\}_{t\geq 0}$ be a $C_0$ semigroup and let $A$ be its infinitesimal generator. Let $u\in D(A)$. Then
        \begin{align*}
            S(t)u\in C^1([0,\infty);X)\cap C([0,\infty);X)
        \end{align*}
        and
        \begin{align*}
            \frac{d}{dt}(S(t)u)=AS(t)u=S(t)Au
        \end{align*}
    \end{thrm}

    \begin{proof}\ \\
        Let $u\in D(A)$. Then
        \begin{align*}
            \bigg (\frac{S(h)-I}{h}\bigg )S(t)u=S(t)\bigg (\frac{S(h)-I}{h}\bigg )\to S(t)Au
        \end{align*}
        as $h\downarrow 0$, by the definition of $A$. Thus $S(t)u\in D(A)$ and 
        \begin{align*}
            AS(t)u=S(t)Au=D^+S(t)u.
        \end{align*}
        Next consider
        \begin{align*}
            \frac{S(t)u-S(t-h)u}{h}=S(t-h)\frac{S(h)u-u}{h}
        \end{align*}
        Hence
        \begin{align*}
            \frac{S(t)u-S(t-h)u}{h}-S(t)Au=S(t-h)\bigg (\frac{S(h)u-u}{h}-Au\bigg )+(S(t-h)-S(t))Au.
        \end{align*}
        But
        \begin{align*}
            \bigg |\bigg |S(t-h)\bigg (\frac{S(h)u-u}{h}-Au\bigg )\bigg |\bigg |&\leq Me^{\omega t}\bigg |\bigg |\frac{S(h)u-u}{h}-Au\bigg |\bigg |\\
            &\to 0 \quad \text{as}\:h\downarrow 0
        \end{align*}
        and
        \begin{align*}
            ||(S(t-h)-S(t))Au||\to 0\quad \text{as}\:h\downarrow 0
        \end{align*}
        by the boundedness of $S(t)$.\\
        Thus,
        \begin{align*}
            D^-S(t)u=S(t)Au=D^+S(t)u
        \end{align*}
        and thus
        \begin{align*}
            \frac{d}{dt}(S(t)u)=AS(t)u=S(t)Au
        \end{align*}
        Similarly, by the boundedness of $S(t)$, the map $t\mapsto S(t)Au$ is continuous so $S(t)u\in C^1([0,\infty);X)$
    \end{proof}
    
    \begin{note}[4.3.3]
        If $A$ is the infinitesimal generator of a $C_0$ semigroup $\{S(t)\}$ then we know by the above theorem that
        \begin{align*}
            u(t)=S(t)u_0
        \end{align*}
        defines the unique solution of the initial value problem
        \begin{align*}
            \begin{rcases}
            \frac{du(t)}{dt}&=Au(t),\:t\geq 0\\
            u(0)&=u_0
            \end{rcases}
        \end{align*}
    \end{note}
    
\newpage

\section{4.4 The Hille-Yosida Theorem}

    \begin{thrm}[4.4.3 (Hille Yosida)]\ \\
        A linear unbounded operator $A$ on a Banach space $X$ is the infinitesimal generator of a contraction semigroup if and only if
        \begin{itemize}
            \item $A$ is closed\\
            \item $A$ is densely defined\\
            \item For every $\lambda>0,\: (\lambda I-A)^{-1}$ is a bounded linear operator and
            \begin{align*}
                ||(\lambda I-A)^{-1}||\leq \frac{1}{\lambda}
            \end{align*}
        \end{itemize}
    \end{thrm}


\newpage

\section{Pazy Stuff}
    \begin{define}\ \\
        Let $X$ be a Banach space with dual space $X'$. Denote $x'\in X'$ at $x\in X$ by $\langle x',x\rangle$ or $\langle x,x'\rangle$. Define the following set $F(x)\subseteq X'$ as
        \begin{align*}
            F(x)=\set{x'}{\langle x',x\rangle=||x||^2=||x'||^2}
        \end{align*}
        (This set is non-empty by the Hahn-Banach theorem.)
    \end{define}
    
    \begin{define}[Dissipativity]\ \\
        A linear operator $A$ is dissipative if for every $x\in D(A)$ there is a $x'\in F(x)$ such that $Re\langle Ax,x'\rangle\leq 0$
    \end{define}
    
    \begin{define}[Maximal Dissipativity]\ \\
        A linear operator $A$ is called maximally dissipative if it is dissipative and $R(I-A)=X$.
    \end{define}

    \begin{thrm}[1.4.3 Lumer-Phillips]\ \\
        \begin{itemize}
            \item If $A$ is dissipative and there is a $\lambda _0 >0$ such that $R(\lambda _0 I-A)=X$, then $A$ is the infinitesimal generator of a $C_0$ semigroup of contractions on $X$.
            \item If $A$ is the infinitesimal generator of a $C_0$ semigroup of contractions on $X$ then $R(\lambda I-A)=X$ for all $\lambda >0$ and A is dissipative.
        \end{itemize}
    \end{thrm}

    \begin{proof}\ \\
        Let $\lambda >0$, the dissipativeness of $A$ implies that $||\lambda x-Ax||\geq \lambda ||x||$ (proof skipped) for every $\lambda >0$ and $x\in D(A)$. Since $R(\lambda_0I-A)=X$, it follows when $\lambda=\lambda_0$ that $(\lambda_0I-A)^{-1}$ is a bounded linear operator and thus closed. This implies $\lambda_0I-A$ is closed and therefore $A$ is also closed. If $R(\lambda I-A)=X$ for every $\lambda>0$ then $\rho(A)\subseteq \R$ and $||R_\lambda (A)||\leq \lambda^{-1}$. It follows by the Hille-Yosida theorem that $A$ is the infinitesimal generator of a $C_0$ semigroup of contractions on $X$.\\
        Consider the set $\Lambda=\set{\lambda}{0<\lambda<\infty,\:R(\lambda I-A)=X}$. Let $\lambda \in \Lambda$. By previous inequality, $\lambda\in\rho(A)$. Since $\rho(A)$ is open, the intersection of $B_r(\lambda)\cap\R\subseteq \Lambda$ and thus $\Lambda$ is open.\\
        On the other hand, let $\{\lambda_n\}\subseteq\Lambda$ and $\lambda_n\to\lambda>0$. For every $y\in X$ there exists an $x_n\in D(A)$ such that
        \begin{align*}
            \lambda_n x_n -Ax_n=y
        \end{align*}
        From the inequality it follows that $||x_n||\leq \lambda_n^{-1}||y||\leq C$ for some $C>0$. Now,
        \begin{align*}
            \lambda_m||x_n-x_m||&\leq ||\lambda_m(x_n-x_m)-A(x_n-x_m)||\\
            &=|\lambda_n-\lambda_m|\:||x_n||\\
            &\leq C|\lambda_n-\lambda_m|\to 0
        \end{align*}
        Therefore $\{x_n\}$ is Cauchy. Let $x_n\to x$. It follows $Ax_n\to \lambda x-y$. Since $A$ is closed, $x\in D(A)$ and $\lambda x-Ax=y$. Therefore $R(\lambda I-A)=X$ and $\lambda\in\Lambda$. Thus $\Lambda$ is closed. By assumption, $\Lambda\neq \emptyset$, therefore $\Lambda=(0,\infty)$\\
        If $A$ is the infinitesimal generator of a $C_0$ semigroup of contractions, $S(t)$, on $X$, then by the Hille-Yosida theorem $\rho(A)\supseteq (0,\infty)$ and therefore $R(\lambda I-A)=X$ for all $\lambda>0$. Furthermore if $x\in D(A), x^*\in F(x)$ then
        \begin{align*}
            |\langle S(t)x,x^*\rangle|\leq ||S(t)x||\:||x^*||\leq ||x||^2
        \end{align*}
        and therefore,
        \begin{align*}
            Re\langle S(t)x-x,x^*\rangle=Re\langle S(t)x,x^*\rangle-||x||^2\leq 0.
        \end{align*}
        By dividing the previous line by $t>0$ and letting $t\downarrow 0$ yields
        \begin{align*}
            Re\langle Ax,x^*\rangle\leq 0.
        \end{align*}
    \end{proof}
    
    \begin{thrm}[Lumer-Phillips]\ \\
        A densely defined operator $A$ is the infinitesimal generator of a $C_0$ semigroup of contractions if and only if it is maximal dissipative.
    \end{thrm}

\newpage

\section{Bibliography}
\printbibliography

\end{document}