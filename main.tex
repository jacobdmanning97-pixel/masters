\documentclass{beamer}
%%%%%%%%%%%%%%%%%%%%%%%%%%%%%
% Do not change the next few lines. 
\usepackage{amsmath,amssymb,bigints,amsthm,mathrsfs,mathtools,biblatex,xcolor,multicol}

\addbibresource{bib.bib}

\newcommand{\singleslidebig}[1]{\begin{frame}
\begin{center}
\textcolor{red}{\Huge{#1}}
\end{center}
\end{frame}}

%%%%%%%%%%%%%%%%%%%%%%%%%%%%%

\usetheme{Warsaw}

\newtheorem{thrm}{Theorem}
\newtheorem{cor}[thrm]{Corollary}
%\newtheorem{lemma}[thrm]{Lemma}
\newtheorem{define}[thrm]{Definition}
\newtheorem{prop}[thrm]{Proposition}
%\newtheorem{note}[thrm]{Remark}
\newtheorem{sump}[thrm]{Assumption}

\newcommand{\R}{\mathbb{R}}
\newcommand{\D}{\mathscr{D}}
\newcommand{\E}{\mathscr{E}}
\newcommand{\G}{\Gamma}
\newcommand{\A}{\textbf{A}}
\newcommand{\BH}{\textbf{H}}
\newcommand{\BW}{\textbf{W}}
\newcommand{\Real}{\text{Re}}
\newcommand{\Imag}{\text{Im}}
\newcommand{\Sch}{\mathcal{S}}
\newcommand{\Lin}{\mathscr{L}}
\newcommand{\Om}{\Omega}
\newcommand{\W}[2]{W^{#1}(#2)}
\newcommand{\WO}[2]{W_0^{#1}(#2)}
\newcommand{\HM}[2]{H^{#1}(#2)}
\newcommand{\LP}[2]{L^{#1}(#2)}
\newcommand{\set}[2]{\left\{#1\:|\:#2\right\}}
\newcommand{\diver}{\text{div }}
\newcommand{\IP}[2]{\left \langle #1\right \rangle_{#2}}
\newcommand{\norm}[2]{\left |\left |#1\right |\right |_{#2}}

\setbeamertemplate{enumerate items}[default]
\definecolor{myor}{RGB}{245,102,0}
\definecolor{mypur}{RGB}{82,45,128}

\setbeamercolor{structure}{fg=myor!90!mypur}
%\setbeamercolor*{block title example}{fg=white, bg= violet!90}
%\setbeamercolor*{block body example}{bg= violet!20}

\newtheorem*{model}{\footnotesize{Hodgkin-Huxley Model}}

%%%%%%%%%%%%%%%%%%%%%%%%%%%%%
% Now Start entering your information.
\title[Properties of Multilayered Structure Coupled PDE Systems]{MASTER PROJECT: Qualitative Properties of the Multilayered Structure - Fluid Interactions Coupled PDE Systems}
\author[Manning]{\textbf{Jacob Manning}\\ 
                Advisor: Dr. Pelin Guven Geredeli}
\institute[Clemson University]{Clemson University}
\date{October 16, 2025}

\begin{document}

\maketitle

\begin{frame}{Motivation}
    \includegraphics[width=5cm,height=3.5cm]{Screenshot 2025-09-11 122320.png}
    \includegraphics[width=5cm,height=3.5cm]{Screenshot 2025-09-11 122431.png}\\
    \centering
    \includegraphics[width=5cm,height=3.5cm]{Screenshot 2025-09-11 122413.png}
\end{frame}

\begin{frame}{Introduction of Model}
    \Large\underline{Geometry}
    \center\begin{itemize}
        \item $\Om_f\subseteq \R^3$, Lipschitz bounded domain
        \item$\Om_s\subseteq \R^3$, $\Om_s\subset\subset \Om_f$
        \item$\partial\Om_s=\G_s$
        \item$\partial\Om_f=\G_s\cup\G_f$
    \end{itemize}
    \includegraphics[width=5cm,height=3.5cm]{Screenshot 2025-05-01 075848.png}
\end{frame}

\begin{frame}{Introduction of First Model}
    \scriptsize
    \begin{align}
        &\begin{cases}
            u_t-\Delta u=0\text{ in } (0,T)\times \Om_f\\
            u|_{\G_f}=0\text{ on } (0,T)\times\G_f\label{bmeq:2};
        \end{cases}\\
        &\begin{cases}
            \frac{\partial^2}{\partial t^2}h_j-\Delta h_j+h_j=\frac{\partial w}{\partial \nu}|_{\G_j}-\frac{\partial u}{\partial \nu}|_{\G_j}\text{ on } (0,T)\times \G_j\\
            h_j|_{\partial{\G_j}\cap\partial{\G_l}}=h_l|_{\partial{\G_j}\cap\partial{\G_l}}\text{ on } (0,T)\times (\partial\G_j\cap\partial\G_l),\text{ for all } 1\leq l\leq K\\
            \text{ such that } \partial\G_j\cap\partial\G_l\neq\emptyset\\
            \frac{\partial h_j}{\partial n_j}\bigg |_{\partial\G_j\cap\partial\G_l}=-\frac{\partial h_l}{\partial n_l}\bigg |_{\partial\G_j\cap\partial\G_l}\text{ on }(0,T)\times(\partial\G_j\cap\partial\G_l)\text{, for all }1\leq l\leq K\\
            \text{ such that }\partial\G_j\cap\partial\G_l\neq \emptyset\label{bmeq:3}
        \end{cases}\\
        &\begin{cases}
            w_{tt}-\Delta w=0\text{ on } (0,T)\times\Om_s\\
            w_t|_{\G_j}=\frac{\partial}{\partial t}h_j=u|_{\G_j}\text{ on } (0,T)\times\G_j\text{, for }j=1,\dots,K\label{bmeq:4}
        \end{cases}
    \end{align}
    \begin{align}
        \nonumber[u(0),h_1(0),\frac{\partial}{\partial t}h_1(0),\dots,h_K(0),\frac{\partial}{\partial t}&h_K(0),w(0),w_t(0)]\\
        &=[u_0,h_{01},h_{11},\dots,h_{0K},h_{1K},w_0,w_1]\label{bmeq:5}
    \end{align}
    where $\G_s=\cup_{j=1}^K \overline{\G_j}$ and $\G_i\cap\G_j=\emptyset$, for $i\neq j$.
\end{frame}

\begin{frame}{Earlier Works}
    Generally, single layered or moving boundary FSI Systems
    \begin{itemize}
        \item George Avalos
        \item Lorena Bociu
        \item Scott Hansen
        \item Igor Kukavica
        \item Irena Lasiecka
        \item Hyesuk Lee
        \item Roberto Triggiani
        \item Amjad Tuffaha
        \item Justin Webster
        \item Enrique Zuazua
        \item etc.
    \end{itemize}
\end{frame}

\begin{frame}{Pioneer Paper Buka\u c, B. Muha, S. \u Cani\'c}
    \begin{columns}[t]
        \column{0.5\textwidth}
        \centering
        \includegraphics[width=5cm,height=3.5cm]{Screenshot 2025-09-11 122220.png}
    \end{columns}
    \begin{itemize}
        \item 2014, Fluid-Multi-layered-structure Interaction, Existence of weak solutions\\
        \item 2015, Regularity and Regularization effects of the same problem\\
        \item 2015, Numerical Results: unconditional stability for the same problem
    \end{itemize}
\end{frame}

\begin{frame}{Functional Settings}
    \scriptsize
    \begin{align}
        \BH=&\{[u_0,h_{01},h_{11},\dots,h_{0K},h_{1K},w_0,w_1]\in \LP{2}{\Om_f}\times\HM{1}{\G_1}\times\LP{2}{\G_1}\times\dots\nonumber\\
        &\times\HM{1}{\G_K}\times\LP{2}{\G_K}\times\HM{1}{\Om_s}\times\LP{2}{\Om_s}\text{, such that for each } 1\leq j\leq K:\nonumber\\
        &(i) w_0|_{\G_j}=h_{0j};\nonumber\\
        &\nonumber(ii)h_{0j}|_{\partial\G_j\cap\partial\G_l}=h_{0l}|_{\partial\G_j\cap\partial\G_l}\text{ on } \partial\G_j\cap\partial\G_l\text{, for all } 1\leq l\leq K\text{ such that }\\ 
        &\qquad \partial\G_j\cap\partial\G_l\neq \emptyset\}\label{bmeq:6}
    \end{align}
    With the accompanying inner product
    \begin{align}
        \IP{\Phi_0,\widetilde{\Phi}_0}{\BH}=\IP{u_0,\widetilde{u}_0}{\Om_f}&+\sum_{j=1}^K \IP{\nabla h_{0j},\nabla \widetilde{h}_{0j}}{\G_j}+\sum_{j=1}^K\IP{h_{0j},\widetilde{h}_{0j}}{\G_j}\nonumber\\
        &+\sum_{j=1}^K \IP{h_{1j},\widetilde{h}_{1j}}{\G_j}+\IP{\nabla w_0,\nabla\widetilde{w}_0}{\Om_s}+\IP{w_1,\widetilde{w}_1}{\Om_s}\label{bmeq:7}
    \end{align}
    where \\
    $\Phi_0=[u_0,h_{01},h_{11},\dots,h_{0K},h_{1K},w_0,w_1]\in\BH$ and $ \widetilde{\Phi}_0=[\widetilde{u}_0,\widetilde{h}_{01},\widetilde{h}_{11},\dots,\widetilde{h}_{0K},\widetilde{h}_{1K},\widetilde{w}_0,\widetilde{w}_1]\in\BH.$
\end{frame}

\begin{frame}{Functional Settings}
    For the PDE system given in ($\ref{bmeq:2})-(\ref{bmeq:5}$) if $\Phi(t)=[u,h_1,\frac{\partial}{\partial t}h_1,\dots,h_K,\frac{\partial}{\partial t}h_K,w,w_1]\in C([0,T];\BH)$ then for $\A:D(\A)\subseteq \BH\to\BH$,
    \begin{align}
        \frac{d}{dt}\Phi(t)=\A\Phi(t); \quad \Phi(0)=\Phi_0\label{bmeq:9}.
    \end{align}
    where $\A$ is given as
    \begin{align}
        \A=\begin{bmatrix}
            \Delta&0&0&\dots&0&0&0&0\\
            0&0&I&\dots&0&0&0&0\\
            -\frac{\partial}{\partial \nu}|_{\G_1}&(\Delta-I)&0&\dots&0&0&\frac{\partial}{\partial \nu}|_{\G_1}&0\\
            \vdots&\vdots&\vdots&\ddots&\vdots&\vdots&\vdots&\vdots\\
            0&0&0&\dots&0&I&0&0\\
            -\frac{\partial}{\partial \nu}|_{\G_K}&0&0&\dots&(\Delta-I)&0&\frac{\partial}{\partial \nu}|_{\G_K}&0\\
            0&0&0&\dots&0&0&0&I\\
            0&0&0&\dots&0&0&\Delta&0
        \end{bmatrix};\label{bmeq:10}
    \end{align}
\end{frame}

\begin{frame}{Introduction of First Model}
    Where the Domain of $\A$ is given as follows
    \scriptsize
    \begin{align}
        D(&\A)=\{[u_0,h_{01},h_{11},\dots,h_{0K},h_{1K},w_0,w_1]\in\BH:\nonumber\\
        (&\textbf{A.i})\:u_0\in\HM{1}{\Om_f},\:h_{1j}\in\HM{1}{\G_j}\text{ for }1\leq j\leq K,\:w_1\in\HM{1}{\Om_s};\nonumber\\
        (&\textbf{A.ii})(a)\:\Delta u_0\in\LP{2}{\Om_f},\:\Delta w_0\in\LP{2}{\Om_s},\:(b)\:\Delta h_{0j}-\frac{\partial u_0}{\partial \nu}\bigg|_{\G_j}+\frac{\partial w_0}{\partial\nu}\bigg|_{\G_j}\in\LP{2}{\G_j}\nonumber\\
        &\text{ for } 1\leq j\leq K; \text{(c) }\frac{\partial h_{0j}}{\partial n_j}\bigg |_{\partial\G_j}\in \HM{-1/2}{\partial\G_j},\text{ for }1\leq j\leq K;\nonumber\\
        (&\textbf{A.iii})\:u_0|_{\G_f}=0,u_0|_{\G_j}=h_{1j}=w_1|_{\G_j},\text{ for }1\leq j\leq K;\nonumber\\
        (&\textbf{A.iv}) \text{ For }1\leq j\leq K:\nonumber\\
        &\nonumber(a)\:h_{1j}|_{\partial \G_j\cap\partial\G_l}=h_{1l}|_{\partial\G_j\cap\partial\G_l}\text{ on }\partial\G_j\cap\partial\G_l,\text{ for all }1\leq l\leq K\text{ such that }\\
        &\qquad \partial\G_j\cap\partial\G_l\neq \emptyset;\nonumber\\
        &\nonumber (b)\:\frac{\partial h_{0j}}{\partial n_j}\bigg |_{\partial\G_j\cap\partial\G_l}=-\frac{\partial h_{0l}}{\partial n_l}\bigg |_{\partial\G_j\cap\partial\G_l}\text{ on }\partial\G_j\cap\partial\G_l,\text{ for all }1\leq l\leq K\\
        &\qquad \text{ such that }\partial\G_j\cap\partial\G_l\neq\emptyset\}.\label{bmeq:11}
    \end{align}
\end{frame}

\begin{frame}{Wellposedness}
    \center\Huge Part I
\end{frame}

\begin{frame}{Wellposedness}
    \begin{thrm}[{(Wellposedness) G. Avalos, PGG, B. Muha [JDE, 2020]}]
        The operator $\A:D(\A)\subseteq\BH\to\BH$ defined above generates a $C_0$-semigroup of contractions. Consequently, the solution $\Phi(t)=[u,h_1,\frac{\partial}{\partial t}h_1,\dots,h_K,\frac{\partial}{\partial t}h_K,w,w_1]$ of the PDE model is given by
        \begin{align*}
            \Phi(t)=e^{\A t}\Phi_0\in C([0,T];\BH),
        \end{align*}
        where $\Phi_0=[u_0,h_{01},h_{11},\dots,h_{0K},h_{1k},w_0,w_1]\in D(\A)$\label{bmthrm:1}
    \end{thrm}
    \scriptsize
    G. Avalos, P. G. Geredeli, B. Muha; “Wellposedness, Spectral Analysis and Asymptotic Stability of a Multilayered Heat-Wave-Wave System ”, Journal of Differential Equations 269 (2020), pp. 7129-7156.
\end{frame}

\begin{frame}{Sketch of Proof}
    Wellposedness was shown using the Lumer-Phillips Theorem. To this end, they showed
    \begin{itemize}
        \item $\A$ is dissipative; i.e. $\IP{\A \Phi,\Phi}{\BH}\leq 0$ for all $\Phi\in D(\A)$\\
        \item Use Lax-Milgram to solve to the static resolvent equation $(\lambda I-\A)\Phi=\Phi^*$ in $D(\A)$; i.e. $\A$ is maximally dissipative\\
        \item Recover of the other structure solution variables given initial data; i.e.
        \begin{align*}
            h_{0j}&=\frac{1}{\lambda}h_{1j}+\frac{1}{\lambda}h_{0j}^*,\text{ for }1\leq j\leq K,\\
            w_0&=\frac{1}{\lambda}w_1+\frac{1}{\lambda}w_0^*.
        \end{align*}
    \end{itemize}
\end{frame}

\begin{frame}{Dissipativity of $\A$}
    Given data $\Phi_0$ in $D(\A)$,
    \begin{align}
        \IP{\A\Phi_0,\Phi_0}{\BH}&= -\norm{\nabla u_0}{\Om_f}^2+2i\sum_{j=1}^K \Imag \IP{\nabla h_{1j},\nabla h_{0j}}{\G_j}\nonumber\\
        &+2i\sum_{j=1}^K \Imag \IP{h_{1j},h_{0j}}{\G_j}+2i\Imag \IP{\nabla w_1,\nabla w_0}{\Om_s}\label{bmeq:14},
    \end{align}
    which gives $\Real \IP{\A\Phi_0,\Phi_0}{\BH}\leq 0.$ Since $\Phi_0$ was arbitrary, it follows that $\A$ is dissipative.
\end{frame}

\begin{frame}{Maximality of $\A$}
    The author solves the following static problem with the Lax-Milgram Theorem. Given parameter $\lambda>0$, suppose $\Phi=[u_0,h_{01},h_{11},\dots,h_{0K},h_{1K},w_0,w_1]\in D(\A)$ is a solution of the equation
        \begin{align}
            (\lambda I-\A)\Phi=\Phi^*,\label{bmeq:15}
        \end{align}
        where $\Phi^*=[u_0^*,h_{01}^*,h_{11}^*,\dots,h_{0K}^*,h_{1K}^*,w_0^*,w_1^*]\in\BH$. 
\end{frame}

\begin{frame}{Maximality of $\A$}
    \scriptsize
    In PDE terms, the abstract equation above becomes
    \begin{align}
        \begin{cases}
            \lambda u_0-\Delta u_0=u_0^*\text{ in }\Om_f\\
            u_0|_{\G_f}=0\text{ on } \G_f\label{bmeq:16};
        \end{cases}
    \end{align}
    and for $1\leq j\leq K$,
    \begin{align}
        \begin{cases}
            \lambda h_{0j}-h_{1j}=h_{0j}^*\text{ in }\G_j\\
            \lambda h_{1j}-\Delta h_{0j}+h_{0j}-\frac{\partial w_0}{\partial\nu}+\frac{\partial u_0}{\partial\nu}=h_{1j}^*\text{ in } \G_j\\
            u_0|_{\G_j}=h_{1j}=w_1|_{\G_j}\text{ in }\G_j\\
            h_{0j}|_{\partial\G_j\cap\partial\G_l}=h_{0l}|_{\partial\G_j\cap\partial\G_l}\text{ on }\partial\G_j\cap\partial\G_l,\text{ for all }1\leq l\leq K\text{ such that }\\
            \qquad \partial\G_j\cap\partial\G_l\neq \emptyset\\
            \frac{\partial h_{0j}}{\partial n_j}\bigg |_{\partial\G_j\cap\partial\G_l}=-\frac{\partial h_{0l}}{\partial n_l}\bigg |_{\partial\G_j\cap\partial\G_l}\text{ on }\partial\G_j\cap\partial\G_l,\text{ for all }1\leq l\leq K\text{ such that }\\
            \qquad \partial\G_j\cap\partial\G_l\neq \emptyset;\label{bmeq:17}
        \end{cases}
    \end{align}
    and also
    \begin{align}
        \begin{cases}
            \lambda w_0-w_1=w_0^*\text{ in }\Om_s\\
            \lambda w_1-\Delta w_0=w_1^*\text{ in }\Om_s\label{bmeq:18}
        \end{cases}
    \end{align}
\end{frame}

\begin{frame}{Maximality of $\A$}
    \scriptsize Define the sets
    \begin{align}
        \mathcal{V}=&\{[\psi_1,\dots,\psi_K]\in\HM{1}{\G_1}\times\dots\times\HM{1}{\G_k}|\text{ For all } 1\leq j\leq K,\nonumber\\
        &\psi_j|_{\partial\G_j\cap\partial\G_l}=\psi_l|_{\partial\G_j\cap\partial\G_l},\text{ for all }1\leq l\leq K\text{ such that }\partial\G_j\cap\partial\G_l\neq\emptyset\}\label{bmeq:20}
    \end{align}
    and
    \begin{align}
        \BW\equiv \set{[\varphi,\psi_1,\dots,\psi_K,\xi]\in H_{\G_f}^1(\Om_f)\times\mathcal{V}\times\HM{1}{\Om_s}}{\varphi|_{\G_j}=\psi_j=\xi|_{\G_j},\text{ for }1\leq j\leq K};\label{bmeq:23}
    \end{align}
    If $\Phi=[u_0,h_{01},h_{11},\dots,h_{0K},h_{1K},w_0,w_1]\in D(\A)$ solves (\ref{bmeq:15}), then necessarily its solution components $[u_0,h_{11},\dots,h_{1K},w_1]\in \BW$ satisfy for $[\varphi,\psi,\xi]\in\BH$,
    \begin{align}
        \IP{\textbf{B}
        \begin{bmatrix}
        \varphi\\\psi_1\\\vdots\\\psi_K\\\xi
        \end{bmatrix},\begin{bmatrix}
        u_0\\h_{11}\\\vdots\\h_{1K}\\w_1
        \end{bmatrix}
        }{\BW^*\times\BW}=\textbf{F}_\lambda\left (\begin{bmatrix}\varphi\\ \psi\\ \xi\end{bmatrix}\right );\label{bmeq:24}
    \end{align}
\end{frame}

\begin{frame}{Maximality of $\A$}
    \scriptsize where
    \begin{align}
        \textbf{F}_\lambda\left (\begin{bmatrix}\varphi\\ \psi\\ \xi\end{bmatrix}\right )&=\IP{u_0^*,\varphi}{\Om_f}+\sum_{j=1}^K\bigg [\IP{h_{1j}^*,\psi_j}{\G_j}-\frac{1}{\lambda}\IP{\nabla h_{0j}^*,\nabla \psi_j}{\G_j}\nonumber\\
        &-\frac{1}{\lambda}\IP{h_{0j}^*, \psi_j}{\G_j}\bigg ]+\IP{w_1^*,\xi}{\Om_s}-\frac{1}{\lambda}\IP{\nabla w_0^*,\nabla\xi}{\Om_s}\label{bmeq:25}
    \end{align}
    and 
    \begin{align*}
        \IP{\textbf{B}
        \begin{bmatrix}
        \varphi\\\psi_1\\\vdots\\\psi_K\\\xi
        \end{bmatrix},\begin{bmatrix}
        \widetilde{\varphi}\\\widetilde{\psi}_1\\\vdots\\\widetilde{\psi}_K\\ \widetilde{\xi}
        \end{bmatrix}
        }{\BW^*\times\BW}&=\lambda\IP{\varphi,\widetilde{\varphi}}{\Om_f}+\IP{\nabla \varphi,\nabla \widetilde{\varphi}}{\Om_f}+\lambda\IP{\xi,\widetilde{\xi}}{\Om_s}+\IP{\nabla \xi,\nabla\widetilde{\xi}}{\Om_s}\\
        &+\sum_{j=1}^K\left [\lambda \IP{\psi_j,\widetilde{\psi}_j}{\G_j}+\frac{1}{\lambda}\IP{\nabla\psi_j,\nabla\widetilde{\psi}_j}{\G_j}+\frac{1}{\lambda}\IP{\psi_j,\widetilde{\psi}_j}{\G_j}\right ]
    \end{align*}
\end{frame}

\begin{frame}{Lax-Milgram}
    \begin{thrm}[Lax-Milgram]
        Let $X$ be a Hilbert space and $a(\cdot,\cdot)$ a continuous $X$-elliptic bilinear form. Then given $f\in X$, there exists a unique $u\in X$ such that
        \begin{align*}
            a(u,v)=\IP{f,v}{X},\quad \text{for every }v\in X.
        \end{align*}\label{Lax}
    \end{thrm}
    Since it is clear that $\textbf{B}\in\Lin(\BW,\BW^*)$ is $\BW$-elliptic; by the Lax-Milgram Theorem, the equation $(\ref{bmeq:24})$ has a unique solution
    \begin{align}
        [u_0,h_{11},\dots,h_{1K},w_1]\in\BW.\label{bmeq:26}
    \end{align}
    It was then showed this solution is in $D(\A)$.
\end{frame}

\begin{frame}{Intoduction of Second Model}
    The model was improved from a heat-wave-wave model to a Stokes-wave-Lam\'e model. The author has shown similar results of wellposedness for this model.
\end{frame}

\begin{frame}{Introduction of Second Model}
    \scriptsize
    \begin{align}
        &\begin{cases}
            u_t-\diver (\nabla u+\nabla^T u)+\nabla p=0 &\text{in } (0,T)\times \Om_f\\
            \diver(u)=0 &\text{in } (0,T)\times \Om_f\\
            u|_{\G_f}=0 &\text{on } (0,T)\times\G_f;
        \end{cases}\label{sweq:1}\\
        &\begin{cases}
        h_{tt}-\Delta_{\G_s}h=[\nu\cdot\sigma(w)]|_{\G_s}-[\nu\cdot(\nabla u+\nabla ^T u)]|_{\G_s}+p\nu\quad\text{on }(0,T)\times\G_s,
        \end{cases}\label{sweq:2}\\
        &\begin{cases}
            w_{tt}-\diver \sigma(w)+w=0&\text{ on } (0,T)\times\Om_s\\
            w_t|_{\G_s}=h_t=u|_{\G_s}&\text{ on } (0,T)\times\G_s
        \end{cases}\label{sweq:3}\\
        &[u(0),h(0),h_t(0),w(0),w_1(0)]=[u_0,h_0,h_1,w_0,w_1]\in \BH\label{sweq:4}
    \end{align}

    Here, $\Delta_{\G_s}(\cdot)$ is the Laplace Beltrami operator, and the stress tensor $\sigma(\cdot)$ constitutes the Lam\'e system of elasticity on the ``thick'' layer. Namely, for function $v$ in $\Om_s$,

    \begin{align*}
        \sigma(v)=2\mu\epsilon(v)+\lambda[I_3\cdot\epsilon(v)]I_3,
    \end{align*}

    where strain tensor $\epsilon(\cdot)$ is given by 

    \begin{align*}
        \epsilon_{ij}=\frac{1}{2}\bigg (\frac{\partial v_j}{\partial x_i}+\frac{\partial v_i}{\partial x_j}\bigg ),\quad 1\leq i,\:j\leq 3
    \end{align*}
\end{frame}

\begin{frame}{Functional Settings}
    \scriptsize
    With the following Hilbert space
    \begin{align}
        \nonumber \BH=\big \{[u_0,h_0,h_1,w_0,&w_1]\in[\LP{2}{\Om_f}]^3\times[\HM{1}{\G_s}]^2\times[\LP{2}{\G_s}]^2\times[\HM{1}{\Om_s}]^3\times\\
        & [\LP{2}{\Om_s}]^3\:\big |\:\diver (u_0)=0,\:u_0\cdot\nu|_{\G_f}=0,\text{ and }w_0|_{\G_s}=h_0 \big \}\label{sweq:5}
    \end{align}

    with the inner product

    \begin{align}
        \IP{\Phi_0,\widetilde{\Phi}_0}{\BH}=\IP{u_0,\widetilde{u}_0}{\Om_f}&+\IP{\nabla _{\G_s}(h_0),\nabla _{\G_s}(\widetilde{h}_0)}{\G_s}+\IP{h_1,\widetilde{h}_1}{\G_s}\\
        &+\IP{\sigma(w_0),\epsilon(\widetilde{w}_0)}{\Om_s}+\IP{w_0,\widetilde{w}_0}{\Om_s}+\IP{w_1,\widetilde{w}_1}{\Om_s}\label{sweq:6}
    \end{align}

    where

    \begin{align}
        \Phi_0=[u_0,h_0,h_1,w_0,w_1]\in\BH;\widetilde{\Phi}_0=[\widetilde{u}_0,\widetilde{h}_0,\widetilde{h}_1,\widetilde{w}_0,\widetilde{w}_1]\in\BH\label{sweq:7}
    \end{align}
\end{frame}

\begin{frame}{Elimination of Pressure}
    \scriptsize
    Elimination of the pressure will be important to formulate the PDE in ($\ref{sweq:1})-(\ref{sweq:4}$) as an ODE. To this end, by using the matching velocity condition and $u$ being divergence free it follows $p$ is harmonic, or
    \begin{align}
        \Delta p(t)=0\text{ in }\Om_f.\label{sweq:8}
    \end{align}
    
    Subsequently, by multiplying by $\nu|_{\G_s}$ and using the matching velocity condition the author obtains the following boundary condition for the pressure variable $p$:

    \begin{align}
        p+\frac{\partial p}{\partial \nu}=\diver(\nabla(u)+\nabla^T(u))\cdot\nu|_{\G_s}+[(\nabla u+\nabla^T u)\cdot\nu-\Delta_{\G_s}(h)-\nu\cdot\sigma(w)|_{\G_s}]\cdot\nu|_{\G_s}\label{sweq:9}
    \end{align}

    Since $u$ is divergence free,

    \begin{align*}
        \frac{\partial p}{\partial\nu}=[\diver(\nabla u+\nabla^T u)]\cdot\nu\text{ on }\G_f.
    \end{align*}
    
    Thus the pressure variable $p(t)$, can formally be written pointwise in time as 

    \begin{align*}
        p(t)=\mathcal{P}_1(u(t))+\mathcal{P}_2(h(t))+\mathcal{P}_3(w(t))
    \end{align*}

    where each $\mathcal{P}_i(\cdot)$ will be given on the next slide.
\end{frame}

\begin{frame}{Elimination of Pressure}
    \scriptsize
    $p(t)=\mathcal{P}_1(u(t))+\mathcal{P}_2(h(t))+\mathcal{P}_3(w(t))$
    \begin{align}
        \begin{cases}
            \Delta\mathcal{P}_1(u)=0 &\text{in }\Om_f,\\
            \mathcal{P}_1(u)=\diver(\nabla(u)+\nabla^T(u))\cdot\nu|_{\G_s}+([(\nabla u+\nabla^T u)]\cdot \nu)\cdot \nu|_{\G_s} &\text{on }\G_s,\\
            \frac{\partial\mathcal{P}_1(u)}{\partial\nu}=\diver(\nabla(u)+\nabla^T(u))\cdot\nu|_{\G_f} &\text{on }\G_f,
        \end{cases}\label{sweq:10}
    \end{align}

    \begin{align}
       \begin{cases}
            \Delta\mathcal{P}_2(h)=0&\text{in }\Om_f,\\
            \mathcal{P}_2(h)=-\Delta_{\G_s}(h)\cdot\nu|_{\G_s}&\text{on }\G_s,\\
            \frac{\partial\mathcal{P}_2(h)}{\partial\nu}=0&\text{on }\G_f,
       \end{cases}\label{sweq:11}
    \end{align}

    and

    \begin{align}
       \begin{cases}
            \Delta\mathcal{P}_3(w)=0&\text{in }\Om_f,\\
            \mathcal{P}_3(w)=-[\nu\cdot\sigma(w)|_{\G_s}]\cdot\nu|_{\G_s}&\text{on }\G_s,\\
            \frac{\partial\mathcal{P}_3(w)}{\partial\nu}=0&\text{on }\G_f.
       \end{cases}\label{sweq:12} 
    \end{align}

    Also note

    \begin{align}
        p_0=\mathcal{P}_1(u_0)+\mathcal{P}_2(h_0)+\mathcal{P}_3(w_0).\label{sweq:15}
    \end{align}
\end{frame}

\begin{frame}{Functional Settings}
    \scriptsize
    It follows, the PDE system given in $(\ref{sweq:1})-(\ref{sweq:4})$ may be associated with an abstract ODE in the Hilbert space $\BH$; namely,
    \begin{align}
        \begin{cases}
            \frac{d}{dt}\Phi (t)=\A\Phi(t)\\
            \Phi(0)=\Phi_0\label{sweq:13}
        \end{cases}
    \end{align}
    where $\Phi(t)=[u(t),h(t),h_t(t),w(t),w_1(t)],\text{ and }\Phi_0=[u_0,h_0,h_1,w_0,w_1]$. Here, the operator $\A:D(\A)\subseteq \BH\to\BH$ is defined by
    \begin{align}
        \A=&\begin{bmatrix}
            \diver(\nabla (\cdot)+\nabla^T(\cdot))&0&0&0&0\\
            0&0&I&0&0\\
            -[\nu\cdot(\nabla(\cdot)+\nabla^T(\cdot))]|_{\G_s}&\Delta_{\G_s}(\cdot)&0&\nu\cdot\sigma(\cdot)|_{\G_s}&0\\
            0&0&0&0&I\\
            0&0&0&\diver\:\sigma(\cdot)-I&0
        \end{bmatrix}+\nonumber\\
        &\begin{bmatrix}
            -\nabla \mathcal{P}_1(\cdot)&-\nabla \mathcal{P}_2(\cdot)&0&-\nabla \mathcal{P}_3(\cdot)&0\\
            0&0&0&0&0\\
            \mathcal{P}_1(\cdot)\nu&\mathcal{P}_2(\cdot)\nu&0&\mathcal{P}_3(\cdot)\nu&0\\
            0&0&0&0&0\\
            0&0&0&0&0\\
        \end{bmatrix}\label{sweq:14}
    \end{align}
    Here, the ``pressure'' operators $\mathcal{P}_i$ are as defined above.
\end{frame}

\begin{frame}{Functional Settings}
    The domain $D(\A)$ of the generator $\A$ is characterized as follows $[u_0,h_0,h_1,w_0,w_1]\in D(\A) \Leftrightarrow$\\

    (\textbf{A.i}) $u_0\in[\HM{1}{\Om_f}]^3,\:h_{1}\in[\HM{1}{\G_s}]^2\:w_1\in[\HM{1}{\Om_s}]^3;$\\
    (\textbf{A.ii}) There exists an associated $\LP{2}{\Om_f}$-function $p_0=p_0(u_0,h_0,w_0)$ such that
    \begin{align*}
        [\diver(\nabla u_0+\nabla^T u_0)-\nabla p_0]\in\LP{2}{\Om_f}
    \end{align*}

    Consequently, $p_0$ is harmonic and so one has the boundary traces\\
    \textbf{(a)} $\left [p_0|_{\G_f},\frac{\partial p_0}{\partial\nu}\big |_{\G_f}\right ]\in\HM{-1/2}{\G_f}\times\HM{-3/2}{\G_f};$\\
    \textbf{(b)} $(\nabla u_0+\nabla^T u_0)\cdot\nu\in\HM{-3/2}{\G_f}$,\\
    (\textbf{A.iii}) $\diver\sigma(w_0)\in\LP{2}{\Om_s}$; consequently, $\nu\cdot\sigma\in\HM{-1/2}{\G_s}$,\\
    (\textbf{A.iv}) $\Delta_{\G_s}(h_0)+[\nu\cdot\sigma(w_0)]_{\G_s}-[(\nabla u_0+\nabla^T u_0)\cdot\nu]|_{\G_s}$\\
    \hspace{5cm} $+[p_0\nu]|_{\G_s}\in\LP{2}{\G_s}$,\\
    (\textbf{A.v}) $u_0|_{\G_f}=0,\:u_0|_{\G_s}=h_1=w_1|_{\G_s}$\\
\end{frame}

\begin{frame}{Wellposedness}
    \begin{thrm}[{(Wellposedness) PGG [JEE, 2024]}]
        With reference to the problem $(\ref{sweq:1})-(\ref{sweq:4})$, the operator $\A:D(\A)\subseteq\BH\to\BH$, defined in $(\ref{sweq:14})$, generates a $C_0$-semigroup of contractions on $\BH$. Consequently, the solution $\Phi(t)=[u(t),h(t),h_t(t),w(t),w_t(t)]$ of $(\ref{sweq:1})-(\ref{sweq:4})$, or equivalently $(\ref{sweq:13})$, is given by
        \begin{align*}
            \Phi(t)=e^{\A t}\Phi_0\in C([0,T];\BH),
        \end{align*}
        where $\Phi_0=[u_0,h_0,h_1,w_0,w_1]\in D(\A)$.\label{swthrm:2}
    \end{thrm}
    \scriptsize
    Geredeli, P.G. An inf-sup approach to $C_0$-semigroup generation for an interactive composite structure-Stokes PDE dynamics. J. Evol. Equ. 24, 50 (2024). https://doi.org/10.1007/s00028-024-00978-3
\end{frame}

\begin{frame}{Sketch of Proof}
    Similarly, wellposedness was shown using the Lumer-Phillips Theorem as the problem is still linear. They showed
    \begin{itemize}
        \item $\A$ is dissipative\\
        \item Show that $\A$ is maximally dissipative using Babuska-Brezzi to solve to the static resolvent equation $(\lambda I-\A)\Phi=\Phi^*$ for $\Phi\in D(\A)$\\
    \end{itemize}
\end{frame}

\begin{frame}{Dissipativity of $\A$}
    Similar to the previous model, the authors shows $\A$ is maximally dissipative. It follows, given data $\Phi$ in $D(\A)$,
    \begin{align*}
        \IP{ \A\Phi,\Phi}\BH=&-\frac{1}{2}\big |\big |\nabla (u_0)+\nabla^T (u_0)\big |\big |^2+2i\Imag\big[\IP{ \nabla_{\G_s}(h_1),\nabla_{\G_s}(h_0)}{\G_s}\\
        &+\IP{\sigma(w_1),\epsilon(w_0)}{\Om_s}\\
        &+\IP{ w_1,w_0}{\Om_s}+\IP{\sigma(w_1),\epsilon(w_0)}{\Om_s}+\IP{ w_1,w_0}{\Om_s}\big ]
    \end{align*}
    which gives $\Real \IP{\A\Phi,\Phi}{\BH}\leq 0.$ Since $\Phi$ was arbitrary, it follows that $\A$ is dissipative.
\end{frame}

\begin{frame}{Maximality of $\A$}
    The author solves the following static problem to show $\A$ is maximal with the Babuska-Brezzi Theorem. Suppose $\Phi\in D(\A)$ is a solution of the equation
    \begin{align}
        (\lambda I-\A)\Phi=\Phi^*,\label{sweq:18}
    \end{align}
    where $\lambda>0,\:\Phi=[u_0,h_0,h_1,w_0,w_1]$, and $\Phi ^*=[u_0^*,h_0^*,h_1^*,w_0^*,w_1^*]\in\BH$. 
\end{frame}

\begin{frame}{Babuska-Brezzi}
    \scriptsize
    \begin{thrm}[Babuska-Brezzi]\ \\
        Let $\Sigma,\:V$ be Hilbert spaces and $a:\Sigma\times\Sigma\to\R,\:b:\Sigma\times V\to\R$, bilinear forms which are continuous. Let 
        \begin{align*}
            Z=\set{\sigma\in\Sigma}{b(\sigma,v)=0,\quad\text{for every }v\in V}.
        \end{align*}
        Assume that $a(\cdot,\cdot)$ is $Z$-elliptic, i.e. there exists a constant $\alpha>0$ such that
        \begin{align*}
            a(\sigma,\sigma)\geq \alpha||\sigma||_\Sigma^2,\quad\text{for every }\sigma\in Z.
        \end{align*}
        Assume further that there exists a constant $\beta>0$ such that
        \begin{align*}
            \sup_{\tau\in\Sigma}\frac{b(\tau,v)}{||\tau||_\Sigma}\geq \beta||v||_V.
        \end{align*}
        Then if $\kappa\in\Sigma$ and $l\in V$, there exists a unique pair $(\sigma,u)\in\Sigma\times V$ such that
        \begin{align}
            a(\sigma,\tau)+b(\tau,u)=\IP{\kappa,\tau}{\Sigma},\quad\text{for every }\tau\in\Sigma\nonumber\\
            b(\sigma,v)=\IP{l,v}{V},\quad\text{for every }v\in V.\label{sweq:16}
        \end{align}
    \end{thrm}
\end{frame}


\begin{frame}{Maximailty of $\A$}
    \scriptsize
    In PDE terms, the  resolvent equation will generate the following relations, where $p_0$ is given via $(\ref{sweq:15})$:
    \begin{align}
        &\begin{cases}
            \lambda u_0-\diver (\nabla u_0+\nabla^T u_0)+\nabla p_0=u_0^*\text{ in }\Om_f\\
            \diver (u_0)=0\text{ in }\Om_f\\
            u_0|_{\G_f}=0\text{ on }\G_f;
        \end{cases}\label{sweq:19}\\
        &\begin{cases}
            \lambda h_0-h_1=h_0^*\text{ in }\G_s\\
            \lambda h_1+[\nu\cdot(\nabla u_0+\nabla^T u_0)]|_{\G_s}-\Delta_{\G_s}(h_0)-[\nu\cdot\sigma (w_0)]|_{\G_s}-p_0\nu=h_1^*\text{ in }\G_s
        \end{cases}\label{sweq:20}\\
        &\begin{cases}
            \lambda w_0-w_1=w_0^*\text{ in }\Om_s\\
            \lambda w_1-\diver \sigma (w_0)+w_0=w_1^*\text{ in }\Om_s\\
            w_1|_{\G_s}=h_1=u_0|_{\G_s}\text{ on }\G_s
        \end{cases}\label{sweq:21}
    \end{align}
\end{frame}

\begin{frame}{Sketch of Proof}
    The author used the following outline
    \begin{itemize}
        \item Solve the decomposed stokes flow for $\textbf{u}$ and $p$\\
        \item Generate a weak formulation for the ``thin'' $(h_1)$ and ``thick'' $(w_1)$ variables\\
        \item Recover of the other structure solution variables $h_0$ and $w_0$ given data $h_0^*\in\HM{1}{\G_s}$ and $w_0^*\in\HM{1}{\Om_s}$; i.e.
        \begin{align}
            h_0=\frac{1}{\lambda}h_1+\frac{1}{\lambda}h_0^*\label{sweq:22}\\
            w_0=\frac{1}{\lambda}w_1+\frac{1}{\lambda}w_0^*\label{sweq:23}
        \end{align}
    \end{itemize}
\end{frame}

\begin{frame}{Decomposition of Stokes Flow}
    \scriptsize
    Decomposition of the Stokes flow into two parts\\
    Zero force and Dirichlet boundary data $g\in\HM{1/2}{\G_s}$: solution is $[u_1(g),p_1(g)]$\\
    \begin{align}
        \begin{cases}
            \lambda u_1-\diver (\nabla u_1+\nabla^T u_1)+\nabla p_1=0\text{ in }\Om_f\\
            \diver (u_1)=\frac{\int_{\G_s}(g\cdot\nu)d\G_s}{\text{meas}(\Om_f)}\text{ in }\Om_f\\
            u_1|_{\G_s}=g\text{ on }\G_s\\
            u_1|_{\G_f}=0\text{ on }\G_f,
        \end{cases}\label{sweq:24}
    \end{align}
    With force term $u_0^*$, zero Dirichlet data and zero divergence: solution is $[u_2(u_0^*),p_2(u_0^*)]$
    \begin{align}
        \begin{cases}
            \lambda u_2-\diver (\nabla u_2+\nabla^T u_2)+\nabla p_2=u_0^*\text{ in }\Om_f\\
            \diver(u_2)=0\text{ in }\Om_f\\
            u_2|_{\G_f}\text{ on }\G_f
        \end{cases}\label{sweq:25}
    \end{align}
    The unique $\{u_0,p_0\}$ of $(\ref{sweq:19})$ may then be expressed as
    \begin{align}
        u_0=u_1(g|_{\G_s})+u_2(u_0^*);\qquad p_0=p_1(g|_{\G_s})+p_2(u_0^*)+c_0,\label{sweq:26}
    \end{align}
    where $c_0$ is the (presently) unknown constant component of the pressure $p_0$ of $(\ref{sweq:19})$.\\
\end{frame}

\begin{frame}{Maximailty of $\A$}
    Define the space
        \begin{align*}
            \textbf{S}=\set{(\varphi,\psi)\in[\HM{1}{\G_s}]^2\times[\HM{1}{\Om_s}]^3}{\varphi=\psi|_{\G_s}}.
        \end{align*}
    The last relation now gives us the following mixed variational formulation in terms of the ``thin'' and ``thick'' structure variables $h_1$ and $w_1$: Namely,
    \begin{align}
        \textbf{a}([h_1,w_1],[\varphi,\psi])+&\textbf{b}([\varphi,\psi],c_0)=\textbf{F}([\varphi,\psi]),\text{ for all }[\varphi,\psi]\in\textbf{S}\nonumber\\
        &\textbf{b}([h_1,w_1],r)=0,\text{ for all }r\in\R.\label{sweq:33}
    \end{align}
    Where $\textbf{a}(\cdot,\cdot):\textbf{S}\times\textbf{S}\to\R,\:\textbf{b}(\cdot,\cdot):\textbf{S}\times\R\to\R,\text{ and the functional }\textbf{F}(\cdot)$ are defined as follows
\end{frame}

\begin{frame}{Maximailty of $\A$}
    \scriptsize
    \begin{align*}
        \textbf{a}([\phi,\xi],[\widetilde{\phi},\widetilde{\xi}])=&\lambda\IP{\phi,\widetilde{\phi}}{\G_s}+\frac{1}{\lambda}\IP{\nabla_{\G_s}\phi,\nabla_{\G_s}\widetilde{\phi}}{\G_s}\\
        &+\lambda\IP{\xi,\widetilde{\xi}}{\Om_s}+\frac{1}{\lambda}\IP{\sigma(\xi),\epsilon(\widetilde{\xi})}{\Om_s}+\frac{1}{\lambda}\IP{\xi,\widetilde{\xi}}{\Om_s}\\
        &+\IP{\nabla u_1(\xi|_{\G_s})+\nabla^T u_1(\xi|_{\G_s}),\nabla\widetilde{u}(\widetilde{\phi})+\nabla^T\widetilde{u}(\widetilde{\phi})}{\Om_f}\\
        &+\lambda\IP{ u_1(\xi|_{\G_s}),\widetilde{u}(\widetilde{\phi})}{\Om_f},\\
        \textbf{b}([\widetilde{\phi},\widetilde{\xi}],r)&=-r\IP{ v,\widetilde{\phi}}{\G_s},
    \end{align*}
    and
    \begin{align*}
        \textbf{F}([\widetilde{\phi},\widetilde{\xi}])=&-\IP{\nabla u_2(u_0^*)+\nabla^T u_2(u_0^*),\nabla\widetilde{u}(\widetilde{\phi})+\nabla^T \widetilde{u}(\widetilde{\phi})}{\Om_f}\\
        &-\frac{1}{\lambda}\IP{\nabla_{\G_s}h_0^*,\nabla_{\G_s}\widetilde{\phi}}{\G_s}-\frac{1}{\lambda}\IP{\sigma(w_0^*),\epsilon(\widetilde{\xi})}{\Om_s}\\
        &-\lambda\IP{ u_2(u_0^*),\widetilde{u}(\widetilde{\phi})}{\Om_f}+\IP{ u_0^*,\widetilde{u}(\widetilde{\phi})}{\Om_f}\\
        &+\IP{ h_1^*,\widetilde{\phi}}{\G_s}+\IP{ w_1^*,\widetilde{\xi}}{\Om_s}-\frac{1}{\lambda}\IP{ w_0^*,\widetilde{\xi}}{\Om_s}.
    \end{align*}
\end{frame}

\begin{frame}{$\inf\sup$ condition}
    \scriptsize
    Given $r\in\R$, let $z\in[\HM{1}{\G_s}]^2$ satisfy
    \begin{align*}
        \Delta_{\G_s}z=\text{sgn}(r)\nu\text{ on }\G_s
    \end{align*}
    It is easily seen that $||\nabla_{\G_s}z||_{\G_s}\leq C||\nu||_{\G_s}$. Now, taking into account that $\gamma:\HM{1}{\Om_s}\to\HM{1/2}{\G_s}$ is a surjective map, and so it has a continuous right inverse $\gamma^+(z)$, we have
    \begin{align*}
        \sup_{[\eta,\varsigma]}\frac{\textbf{b}([\eta,\varsigma],r)}{||[\eta,\varsigma]||_{\textbf{S}}}&\geq \frac{\textbf{b}([z,\gamma^+(z)],r)}{||z||_{[\HM{1}{\G_s}]^2}}\\
        &=\frac{-r\int_{\G_s}\nu\cdot zd\G_s}{||z||_{[\HM{1}{\G_s}]^2}}\\
        &=-r\text{ sgn}(r)\frac{\int_{\G_s}\Delta_{\G_s}z\cdot z d\G_s}{||z||_{[\HM{1}{\G_s}]^2}}\\
        &=|r|\frac{\int_{\G_s}|\nabla_{\G_s}z|^2d\G_s}{||z||_{[\HM{1}{\G_s}]^2}}\\
        &=|r|\:||z||_{[\HM{1}{\G_s}]^2}
    \end{align*}
    which yields that the inf-sup condition holds wth the constant $\beta=||z||_{[\HM{1}{\G_s}]^2}$. 
\end{frame}

\begin{frame}{Conclusion}
    By the Babuska-Brezzi Theorem, it is then shown that this unique solution is in $D(\A)$.
\end{frame}

\begin{frame}{Long-time Dynamics}
    \center\Huge Part II
\end{frame}

\begin{frame}{Stability}
    \Large\underline{Stability Types}
    \begin{itemize}
        \item Strong Stability
        \item Polynomial Stability
        \item Exponential Stability
    \end{itemize}
\end{frame}

\begin{frame}{Strong Stability}
    \begin{thrm}[{(Strong Stability) G. Avalos, PGG, B. Muha [JDE, 2020]}]
        For the modeling generator $\A:D(\A)\subseteq\BH\to\BH$ of $(\ref{bmeq:2})-(\ref{bmeq:5})$, one has $\sigma(\A)\cap i\R=\emptyset$. Consequently, the $C_0$-semigroup $\left \{e^{\A t}\right \}_{t\geq 0}$ given in Theorem $\ref{bmthrm:1}$ is strongly stable. That is, the solution $\Phi(t)$ of the PDE $(\ref{bmeq:2})-(\ref{bmeq:5})$ tends asymptotically to the zero state for all initial data $\Phi_0\in\BH$\label{bmthrm:2}
    \end{thrm}
    \scriptsize
    G. Avalos, P. G. Geredeli, B. Muha; “Wellposedness, Spectral Analysis and Asymptotic Stability of a Multilayered Heat-Wave-Wave System ”, Journal of Differential Equations 269 (2020), pp. 7129-7156.
\end{frame}

\begin{frame}{Sketch of Proof}
    \begin{thrm}[{(Strong Stability) [Arendt-Batty, 1988]}]
        Let $T(t)_{t\geq 0}$ be a bounded $C_0$-semigroup on a reflexive Banach space $X$, with generator $\A$. Assume that $\sigma_p(\A)\cap i\R=\emptyset$, where $\sigma_p(\A)$ is the point spectrum of $\A$. If $\sigma(\A)\cap i\R$ is countable then $T(t)_{t\geq 0}$ is strongly stable.\label{ssthrm:12}
    \end{thrm}
    Recall: $\sigma(\A)=\sigma_p(\A)\cup\sigma_c(\A)\cup\sigma_r(\A)$\\
    Note that  $\sigma(\A)\cap i\R=\emptyset$ is equivalent to showing $i\R\subseteq \rho(\A)$\\
    To this end the authors checked
    \begin{itemize}
        \item $0\in\rho(\A)$
        \item The continuous spectrum
        \item The eigenvalues of $\A^*$
    \end{itemize}
\end{frame}

\begin{frame}{$0\in\rho(\A)$}
    \scriptsize
    Given $\Phi^*=[u_0^*,h_{01}^*,h_{11}^*,\dots,h_{0K}^*,h_{1K}^*,w_0^*,w_1^*]\in\BH$, the problem is to find $\Phi=[u_0,h_{01},h_{11},\dots,h_{0K},h_{1K},w_0,w_1]\in D(\A)$ which solves
    \begin{align}
        \A\Phi=\Phi^*\label{bmeq:42}
    \end{align}
    It follows
    \begin{align}
        w_1&=w_0^*\in\HM{1}{\Om_s}\label{bmeq:44}\\
        h_{1j}&=h_{0j}^*\in\HM{1}{\G_j},\text{ for }1\leq j\leq K\label{bmeq:45}
    \end{align}
    and
    \begin{align}
        \begin{cases}
            \Delta u_0=u_0^*\\
            u_0|_{\G_f}=0\\
            u_0|_{\G_s}=w_0^*|_{\G_s}
        \end{cases}\label{bmeq:46}
    \end{align}
    Recall the set (to be used on the next slide)
    \begin{align}
        \mathcal{V}=&\{[\psi_1,\dots,\psi_K]\in\HM{1}{\G_1}\times\dots\times\HM{1}{\G_k}|\text{ For all } 1\leq j\leq K,\nonumber\\
        &\psi_j|_{\partial\G_j\cap\partial\G_l}=\psi_l|_{\partial\G_j\cap\partial\G_l},\text{ for all }1\leq l\leq K\text{ such that }\partial\G_j\cap\partial\G_l\neq\emptyset\}
    \end{align}
\end{frame}

\begin{frame}{$0\in\rho(\A)$}
    \scriptsize
    And define the set
    \begin{align}
            \chi\equiv\set{[\psi,\xi]\in\mathcal{V}\times\HM{1}{\Om_s}}{\psi_j=\xi|_{\G_j}\text{ for }1\leq j\leq K}.\label{bmeq:49}
    \end{align}
    Testing with these equations, it follows
    \begin{align}
        &\IP{\nabla w_0,\nabla \xi}{\Om_s}+\sum_{j=1}^K\left [\IP{\nabla h_{0j},\nabla \psi_j}{\G_j}+\IP{h_{0j},\psi_j}{\G_j}\right ]\nonumber\\
        =&-\IP{w_1^*,\xi}{\Om_s}-\sum_{j=1}^K\left [\IP{h_{1j}^*,\psi_j}{\G_j}+\IP{\frac{\partial u_0}{\partial\nu},\psi_j}{\G_j}\right ],\label{bmeq:50}
    \end{align}
    Since the bilinear form $b(\cdot,\cdot):\chi\to\R$, given by
    \begin{align}
        b([\psi,\xi],\left [\widetilde{\psi},\widetilde{\xi} \right ])=\IP{\nabla\xi,\nabla\widetilde{\xi}}{\Om_s}+\sum_{j=1}^K\left [\IP{\nabla\psi_j,\nabla\widetilde{\psi}_j}{\G_j}+\IP{\psi_j,\widetilde{\psi}_j}{\G_j}\right ]\label{bmeq:51}
    \end{align}
    for every $[\psi,\xi],\:\left [\widetilde{\psi},\widetilde{\xi} \right ]\in\chi$, is continuous and $\chi$-elliptic, then by Lax-Milgram, there exists a unique solution
    \begin{align}
        \varphi=[(h_{01},h_{02},\dots,h_{0K}),w_0]\in\chi\label{bmeq:52}
    \end{align}
    (Then shown to be in $D(\A)$)
\end{frame}

\begin{frame}{$\beta,\:i\beta\not\in\sigma_c(\A),\:\beta\neq 0$}
    Assume to the contrary. Then since $\sigma_c(\A)\subseteq\sigma_{\text{app}}(\A)$ there exists a sequence $\{\Phi_n\}=\{[u_{n},h_{1n},\xi_{1n},...,h_{Kn},\xi_{Kn},w_{0n},w_{1n}]\}\subseteq D(\A)$ which satisfy for $n\in\mathbb{N}$
    \begin{align*}
        ||\Phi_n||_{\BH}=1\text{ and }||(i\beta I-\A)\Phi_n||_{\BH}<\frac{1}{n}
    \end{align*}
\end{frame}

\begin{frame}{$\beta,\:i\beta\not\in\sigma_r(\A),\:\beta\neq 0$}
    It is known that for a closed and densely defined operator $\A$, if $\lambda\in\sigma_r(\A)$ then $\overline{\lambda}\in\sigma_p(\A^*)$. It follows\\
    $\A^*: D(\A^*)\subseteq\BH\to\BH$ is given by
        \begin{align*}
            \A^*=\begin{bmatrix}
                \Delta&0&0&\dots&0&0&0&0\\
                0&0&-I&\dots&0&0&0&0\\
                -\frac{\partial}{\partial \nu}|_{\G_1}&(I-\Delta)&0&\dots&0&0&-\frac{\partial}{\partial \nu}|_{\G_1}&0\\
                \vdots&\vdots&\vdots&\ddots&\vdots&\vdots&\vdots&\vdots\\
                0&0&0&\dots&0&-I&0&0\\
                -\frac{\partial}{\partial \nu}|_{\G_K}&0&0&\dots&(I-\Delta)&0&-\frac{\partial}{\partial \nu}|_{\G_K}&0\\
                0&0&0&\dots&0&0&0&-I\\
                0&0&0&\dots&0&0&-\Delta&0
            \end{bmatrix};
        \end{align*}
    $D(\A)=D(\A^*)$, $i\beta$ is not an eigenvalue of $\A^*$ so $i\beta\not\in\sigma_r(\A)$
\end{frame}

\begin{frame}{Conclusion}
    By the spectral analysis, the authors concluded there is strong stability for any initial data in $D(\A)$.
\end{frame}

\begin{frame}{Strong Stability}
    \begin{thrm}[{(Strong Stability) PGG [JDE, 2025]}]
        With reference to problem $(\ref{sweq:1})-(\ref{sweq:4})$, zero is an eigenvalue for the generator $\A:\:D(\A)\subseteq\BH\to\BH$. Consequently the solution $\{e^{\A t}\}|_{[Null(\A)]^\perp}$ decays to the zero state for any initial data $\Phi_0=[u_0,h_0,h_1,w_0,w_1]\in[Null(\A)]^\perp$.
    \end{thrm}
    \scriptsize
    Pelin G. Geredeli, Spectral analysis and asymptotic decay of the solutions to multilayered structure-Stokes fluid interaction PDE system, Journal of Differential Equations, Volume 427, 2025, Pages 1-25, ISSN 0022-0396, https://doi.org/10.1016/j.jde.2025.01.080.
\end{frame}

\begin{frame}{Sketch of Proof}
    As a reminder
    \begin{thrm}[{(Strong Stability) [Arendt-Batty, 1988]}]
        Let $T(t)_{t\geq 0}$ be a bounded $C_0$-semigroup on a reflexive Banach space $X$, with generator $\A$. Assume that $\sigma_p(\A)\cap i\R=\emptyset$, where $\sigma_p(\A)$ is the point spectrum of $\A$. If $\sigma(\A)\cap i\R$ is countable then $T(t)_{t\geq 0}$ is strongly stable.
    \end{thrm}
    It is then checked
    \begin{itemize}
        \item Zero is an eigenvalue
        \item The continuous spectrum
        \item The eigenvalues of $\A^*$
    \end{itemize}
\end{frame}

\begin{frame}{Zero is an Eigenvalue}
    For $\Phi=[u_0,h_0,h_1,w_0,w_1]\in D(\A)$, it follows $\A\Phi=0$ implies
    \begin{align*}
        &\diver (\nabla u_0+\nabla^T u_0)-\nabla p=0\text{ in }\Om_f\\
        &\boxed{h_1=0}\text{ in }\G_s\\
        &-\nu\cdot(\nabla u_0+\nabla^T u_0)|_{\G_s}+\Delta_{\G_s}(h_0)+\nu\cdot\sigma(w_0)|_{\G_s}+p\nu=0\text{ in }\G_s\\
        &\boxed{w_1=0}\text{ in }\Om_s\\
        &\diver\sigma(w_0)-w_0=0\text{ in }\Om_s
    \end{align*}
    Dissipativity shows
    \begin{align*}
        0=\Real\IP{\A\Phi,\Phi}{}&=\frac{1}{2}||\nabla u_0+\nabla^T u_0||^2\implies \boxed{u_0=0}\\
        &\boxed{p=c_0=\text{constant}}
    \end{align*}
\end{frame}

\begin{frame}{Zero is an Eigenvalue}
    Define $S=\set{[f,g]\in\HM{1}{\G_s}\times\HM{1}{\Om_s}}{f=g|_{\G_s}}$.\\
    Testing the previous equations with $f,g$,
    \begin{align*}
        \textbf{B}([h_0,w_0],[f,g])&=\IP{\nabla_{\G_s}(h_0),\nabla_{\G_s}(f)}{\G_s}+\IP{\sigma(w_0),\epsilon(g)}{\Om_s}+\IP{w_0,g}{\Om_s}\\
        &=\IP{c_0\nu,f}{\G_s}
    \end{align*}
    As $B(\cdot,\cdot)$ is continuous and S-elliptic, using Lax-Milgram finds $\{h_0,w_0\}\in S$. (Then in $D(\A)$)
    \begin{align*}
        \text{Null}(\A)=\text{Span}\left \{\begin{bmatrix}
        0\\h_0\\0\\w_0\\0
        \end{bmatrix} \right \}
    \end{align*}
    $\text{Null}(\A)^\perp=\set{[\tilde{u}_0,\tilde{h}_0,\tilde{h}_1,\tilde{w}_0,\tilde{w}_1]\in\BH}{\int_{\G_s}\nu\cdot\tilde{h}_0d\G_s=0}$
\end{frame}

\begin{frame}{$\beta,\:i\beta\not\in\sigma_p(\A),\:\beta\neq 0$}
    For $\Phi=[u_0,h_0,h_1,w_0,w_1]$,
    \begin{align*}
        [i\beta I-\A]\Phi=0
    \end{align*}
    Thus we see there is an overdetermined eigenvalue problem
    \begin{align*}
        -\beta^2 w_0-\diver\sigma(w_0)+w_0=0\text{ in }\Om_s\\
        w_0|_{\G_s}=0\text{ on }\G_s\\
        \nu\cdot\sigma(w_0)=-c_0\nu\text{ on }\G_s
    \end{align*}
    \textbf{Assumption} for (fixed) given $\beta$, assume that the only solution to the above overdetermined problem is $w_0=0$ (and necessarily $c_0=0$).
\end{frame}

\begin{frame}{$\beta,\:i\beta\not\in\sigma_c(\A),\:\beta\neq 0$}
    Assume to the contrary. Then since $\sigma_c(\A)\subseteq\sigma_{\text{app}}(\A)$ there exists a sequence $\{\Phi_n\}=\{[u_{0n},h_{0n},h_{1n},w_{0n},w_{1n}]\}\subseteq D(\A)$ such that
    \begin{align*}
        ||\Phi_n||=1\text{ and }||(i\beta I-\A)\Phi_n||_{\BH}<\frac{1}{n}
    \end{align*}
\end{frame}

\begin{frame}{$\beta,\:i\beta\not\in\sigma_r(\A),\:\beta\neq 0$}
    It is known that for a closed and densely defined operator $\A$, if $\lambda\in\sigma_r(\A)$ then $\overline{\lambda}\in\sigma_p(\A^*)$. It follows\\
    $\A^*: D(\A^*)\subseteq\BH\to\BH$ is given by
        \begin{align}
            \A^*=&\begin{bmatrix}
                \diver(\nabla (\cdot)+\nabla^T(\cdot))&0&0&0&0\\
                0&0&-I&0&0\\
                -[\nu\cdot(\nabla(\cdot)+\nabla^T(\cdot))]|_{\G_s}&-\Delta_{\G_s}(\cdot)&0&-\nu\cdot\sigma(\cdot)|_{\G_s}&0\\
                0&0&0&0&-I\\
                0&0&0&-\diver\sigma(\cdot)+I&0
            \end{bmatrix}+\nonumber\\
            &\begin{bmatrix}
                -\nabla \mathcal{P}_1(\cdot)&\nabla \mathcal{P}_2(\cdot)&0&\nabla \mathcal{P}_3(\cdot)&0\\
                0&0&0&0&0\\
                \mathcal{P}_1(\cdot)\nu&-\mathcal{P}_2(\cdot)\nu&0&-\mathcal{P}_3(\cdot)\nu&0\\
                0&0&0&0&0\\
                0&0&0&0&0\\
            \end{bmatrix}\label{sseq:38}
        \end{align}
    $D(\A)=D(\A^*)$, $i\beta$ is not an eigenvalue of $\A^*$ so $i\beta\not\in\sigma_r(\A)$
\end{frame}

\begin{frame}{Conclusion}
    It is important to note that $0\in\rho(\A|_{\text{Null(\A)}^\perp})$. In other words, the resolvent of $\A|_{\text{Null(\A)}^\perp}$.\\
    \vspace{1cm}
    By the spectral analysis, the author showed that there is strong stability for any initial data taken in Null($\A)^\perp$.
\end{frame}

\begin{frame}{Future Work and Open Questions}
    \Large\underline{Topics}
    \center
    \begin{itemize}
        \item Stokes Model polynomial decay
        \item Lack of exponential decay
        \item Navier-Stokes model
    \end{itemize}
\end{frame}

\begin{frame}{Conclusion}
    \center\Huge Thank You!
\end{frame}

\end{document}