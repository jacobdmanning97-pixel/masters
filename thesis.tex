%
% thesis.tex
%
% Master's Thesis/Ph.D. Dissertation Template
% Clemson University
%

%
% The document guidelines say the font can be between 10pt and 12pt.
% Specify whatever you want it to be here.
%
\documentclass[10pt]{ClemsonThesis}

%
% Use any additional packages you might need
%
\usepackage{amsmath,amssymb,bigints,amsthm,mathrsfs,mathtools}
%% \usepackage{listings}
%% \usepackage{comment}


%
% Make the document your own -- fill in these values to reflect the type of
% document you are writing.
%
\title{Qualitative Properties of the Multilayered Structure - Fluid Interactions Coupled PDE systems}
\department{Mathematical and Statistical Sciences}
\documentType{Masters Project}
\major{Mathematical Sciences}
\degree{Masters of Science}
\graduationMonth{December}
\graduationYear{2025}
\author{Jacob David Manning}
\committeeChair{Dr. Pelin Guven Geredeli}
\committeeMemberOne{Dr. Hyesuk Lee}
\committeeMemberTwo{Dr. Quyuan Lin}
\committeeMemberThree{Dr. Shitao Liu}
%% optional \committeeMemberFour{Dr. Jane Doe}
%% optional \committeeMemberFive{Dr. Mary Doe}
%% optional \committeeMemberSix{Dr. Mark Doe}

%
% PDF Setup -- most of this you do not need to touch
%
\hypersetup{
    colorlinks,
    linkcolor={black},
    citecolor={black},
    filecolor={black},
    urlcolor={black},
    pdftitle={\theTitle},
    pdfauthor={\theAuthor},
    pdfsubject={\theDocumentType},
    pdfkeywords={Clemson University, \theDepartment, \theDocumentType, \theMajor, \theDegree},
    pdfstartpage={1},
}


%
% User-specified command definitions/redefinitions
%

\newtheorem{thrm}{Theorem}
\newtheorem{cor}[thrm]{Corollary}
\newtheorem{lemma}[thrm]{Lemma}
\newtheorem{define}[thrm]{Definition}
\newtheorem{prop}[thrm]{Proposition}
\newtheorem{note}[thrm]{Remark}
\newtheorem{sump}[thrm]{Assumption}

\newcommand{\R}{\mathbb{R}}
\newcommand{\D}{\mathscr{D}}
\newcommand{\E}{\mathscr{E}}
\newcommand{\G}{\Gamma}
\newcommand{\A}{\textbf{A}}
\newcommand{\BH}{\textbf{H}}
\newcommand{\BW}{\textbf{W}}
\newcommand{\Real}{\text{Re}}
\newcommand{\Imag}{\text{Im}}
\newcommand{\Sch}{\mathcal{S}}
\newcommand{\Lin}{\mathscr{L}}
\newcommand{\Om}{\Omega}
\newcommand{\W}[2]{W^{#1}(#2)}
\newcommand{\WO}[2]{W_0^{#1}(#2)}
\newcommand{\HM}[2]{H^{#1}(#2)}
\newcommand{\BFHM}[2]{\textbf{H}^{#1}(#2)}
\newcommand{\LP}[2]{L^{#1}(#2)}
\newcommand{\BFLP}[2]{\textbf{L}^{#1}(#2)}
\newcommand{\set}[2]{\left\{#1\:\middle |\:#2\right\}}
\newcommand{\diver}{\text{div }}
\newcommand{\IP}[2]{\left \langle #1\right \rangle_{#2}}
\newcommand{\norm}[2]{\left |\left |#1\right |\right |_{#2}}

%% \newcommand{\cplusplus}{{\rm C\raise.5ex\hbox{\small ++}}}
%% \newcommand{\num}[1]{\mbox{(\textit{#1})}}
%% \renewcommand{\ttdefault}{pcr}
%% \renewcommand\lstlistlistingname{List of Listings}

\begin{document}
%  ============================================================================
    \frontmatter % Begin front matter (pages are numbered with Roman numerals)
%  ============================================================================

    \addtotoc{Title Page}{\maketitle}          % Generate the title page
    \doublespacing                             % Text should be double spaced
    \setcounter{page}{2}                       % Abstract begins on page 2
    \addtotoc{Plain Language Abstract}{\chapter*{Plain Language Abstract}
\indent Fluid structure interaction (FSI) partial differential equations (PDEs) show up all over nature and thus are a common mathematical model of study. For example, due to hemodynamic forces generated by blood moving through mammalian arteries, the vascular walls, being composed of viscoelastic materials, undergo large deformations during the blood transport process. As such, there is a coupling of respective blood flow and wall deformation dynamics. This physiological interaction between arterial walls and blood flow plays a crucial role in the physiology and pathophysiology of the human cardiovascular system and can be mathematically realized by FSI PDEs. In such FSI modeling, the blood flow is governed by the fluid flow PDE component (incompressible Stokes or Navier-Stokes); the displacements along the elastic vascular walls are described by the structural PDE component (e.g., Lam\'e systems of elasticity). “Single layered” FSI models - i.e., FSI models in which only one (three dimensional) elastic PDE appears to describe the structural dynamics - have been studied extensively in the literature. However, many biomedical devices (such as stents) are being developed with the view that vascular wall structures are constituted of composite materials and not of a single layer.\\
\indent Accordingly, in this project, we revisit two multilayered FSI systems where the coupling of the 3-D fluid (blood flow) and 3-D elastic (structural vascular wall) PDE components is realized via an additional 2-D elastic system on the boundary interface, but looking through the lens of cellular dynamics. The main purpose of this project is to review the methodologies used to show the qualitative properties of those coupled PDE dynamics such as wellposedness and longtime behavior of solutions. Both models that will be covered are a simplification of the Navier-Stokes model. The first one that will be introduced is denoted the canonical model (3D heat, 2D wave, 3D wave). Where as the second is a more realistic and complicated model (3D Stokes, 2D wave, 3D Lam\'e) which is described again via different PDE dynamics.}

    \addtotoc{Abstract}{\chapter*{Abstract}
In this work, we will conduct a survey of recent developments of two different composite structure multilayered models and investigate the qualitative properties of the wellposedness and stability. The first being a 3-D heat equation is coupled with a 3-D wave equation via a 2-D interface whose dynamics is described by a 2-D wave equation. This model is also notably set up for computational modeling as well. This is a simplification of the Navier-Stokes dynamics that we would expect of such a system and as such does not have any pressure terms that are present in the second model. The authors established wellposedness via a Lumer-Phillips approach and strong stability or asymptotic decay to the zero state for all initial data. This is done by analyzing the spectrum of the generator for the associated $C_0$-semigroup.\\
\indent The second model uses a different set of PDEs to model the interactions. This updated model is still a simplification, but is more realistic. It is a 3-D Stokes flow equation, coupled with a 3-D elastic dynamics equation with an additional 2-D interface with elastic dynamics. Similarly, as the model is still linear, Lumer-Phillips was used to show wellposedness. Using a nonstandard mixed variational formulation, the original author showed that the PDE system generates a $C_0$-semigroup. The pressure term in the 3D Stokes equation adds a great challenge to the analysis. The author used non-Leray-based elimination of the associated pressure term. The elastic solution was found via a Babuska-Brezzi approach. Long term strong stability was again shown by analyzing the spectrum of the associated operator that generates this $C_0$-semigroup. Of note, the author shows that zero is an eigenvalue of $\A$. From this, the author addresses the issue of asymptotic decay of the solution to the zero state for any initial data taken from the orthogonal complement of the zero eigenspace Null$(\A)^\perp$.\\}  % Generate the abstract

    %
    % The dedication page is optional.  Comment out this line if you do not
    % want to include this page.
    %
    \addtotoc{Dedication}{\chapter*{Dedication}
I would like to dedicate this project to my wife Rafaela. She has put in just as much work and sacrifice into this degree as I have.}

    %
    % The acknowledgment page is optional.  Comment out this line if you do
    % not want to include this page.
    %
    \addtotoc{Acknowledgments}{\chapter*{Acknowledgements}
I would like to thank Pelin Guven Geredeli who served as my advisor for my time at Clemson University as well as everyone on my committee. I would also like to recognize the National Science Foundation and acknowledge their partial funding from NSF Grant DMS-2348312 which I received during my time at Clemson.}

    \singlespacing                             % Single space the lists
    \tableofcontents \clearpage                % Generate the Table of Contents

    %
    % REMEMBER: Review your caption listings in the generated lists
    %           and make sure they include '\newline' commands as necessary.
    %           See the README for further information.
    %
    %\addtotoc{List of Tables}{\listoftables}   % Generate the List of Tables
    \addtotoc{List of Figures}{\listoffigures} % Generate the List of Figures

    %
    % Include other optional lists.  Computer science, for example, would
    % likely include a 'List of Listings' (and would \usepackage{listings}
    % and \renewcommand\lstlistlistingname{List of Listings}).
    %
    %% \addtotoc{List of Listings}{\lstlistoflistings}



%  ===========================================================================
    \mainmatter % Begin main matter (pages are numbered with Arabic numerals)
%  ===========================================================================
    \doublespacing % Text should be double spaced

    %
    % Here we have each chapter in a separate file.  Name these as you choose,
    % and include them in the order you want them to appear.  Be sure to use
    % the \inputfile command.
    %
    \inputfile{introduction.tex}
    \inputfile{notation.tex}
    \inputfile{Well Posedness.tex}
    \inputfile{Stability.tex}
    \inputfile{conclusions.tex}
    \inputfile{Appendix.tex}



    \singlespacing                             % Single space the Bibliography

    %
    % The bibliography style.  Set this to whatever matches you discipline.
    % For example, Computer Science would likely use 'plain'.  You might
    % also want to change the name from 'Bibliography' to 'References'
    % or 'Work Cited'.
    %
    % 'plain'   gets you numbered references and citations (e.g., [1] Dyson).
    %
    % 'alpha'   gets you labels formed from an abbreviation of the authors'
    %           names and the year of publication.  If there is more than
    %           one author, it will use the first letter of up to the first
    %           three authors' last names.
    %
    %           Some examples:
    %               [DED01] F.W. Dyson, A.G. Edgar, and D.B. Denny ... 2001
    %               [DE01] F.W. Dyson, A.G. Edgar ... 2001
    %               [Dys01] F.W. Dyson ... 2001
    %
    % 'apa like' gets you labels formed from the authors' names and year of
    %           publication.
    %
    %           Some examples:
    %               [Dyson et al., 2001] F.W. Dyson, A.G. Edgar, and
    %                 D.B. Denny ... 2001
    %               [Dyson and Edgar, 2001] F.W. Dyson, A.G. Edgar ... 2001
    %               [Dyson, 2001] F.W. Dyson ... 2001
    %
    \addtotoc{Bibliography}{\inputfile{References.tex}}
\end{document}
