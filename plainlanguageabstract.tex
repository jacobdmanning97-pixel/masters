\chapter*{Plain Language Abstract}
\indent Fluid structure interaction (FSI) partial differential equations (PDEs) show up all over nature and thus are a common mathematical model of study. For example, due to hemodynamic forces generated by blood moving through mammalian arteries, the vascular walls, being composed of viscoelastic materials, undergo large deformations during the blood transport process. As such, there is a coupling of respective blood flow and wall deformation dynamics. This physiological interaction between arterial walls and blood flow plays a crucial role in the physiology and pathophysiology of the human cardiovascular system and can be mathematically realized by FSI PDEs. In such FSI modeling, the blood flow is governed by the fluid flow PDE component (incompressible Stokes or Navier-Stokes); the displacements along the elastic vascular walls are described by the structural PDE component (e.g., Lam\'e systems of elasticity). “Single layered” FSI models - i.e., FSI models in which only one (three dimensional) elastic PDE appears to describe the structural dynamics - have been studied extensively in the literature. However, many biomedical devices (such as stents) are being developed with the view that vascular wall structures are constituted of composite materials and not of a single layer.\\
\indent Accordingly, in this project, we revisit two multilayered FSI systems where the coupling of the 3-D fluid (blood flow) and 3-D elastic (structural vascular wall) PDE components is realized via an additional 2-D elastic system on the boundary interface, but looking through the lens of cellular dynamics. The main purpose of this project is to review the methodologies used to show the qualitative properties of those coupled PDE dynamics such as wellposedness and longtime behavior of solutions. Both models that will be covered are a simplification of the Navier-Stokes model. The first one that will be introduced is denoted the canonical model (3D heat, 2D wave, 3D wave). Where as the second is a more realistic and complicated model (3D Stokes, 2D wave, 3D Lam\'e) which is described again via different PDE dynamics.