\chapter*{Abstract}
In this work, we will conduct a survey of recent developments of two different composite structure multilayered models and investigate the qualitative properties of the wellposedness and stability. The first being a 3-D heat equation is coupled with a 3-D wave equation via a 2-D interface whose dynamics is described by a 2-D wave equation. This model is also notably set up for computational modeling as well. This is a simplification of the Navier-Stokes dynamics that we would expect of such a system and as such does not have any pressure terms that are present in the second model. The authors established wellposedness via a Lumer-Phillips approach and strong stability or asymptotic decay to the zero state for all initial data. This is done by analyzing the spectrum of the generator for the associated $C_0$-semigroup.\\
\indent The second model uses a different set of PDEs to model the interactions. This updated model is still a simplification, but is more realistic. It is a 3-D Stokes flow equation, coupled with a 3-D elastic dynamics equation with an additional 2-D interface with elastic dynamics. Similarly, as the model is still linear, Lumer-Phillips was used to show wellposedness. Using a nonstandard mixed variational formulation, the original author showed that the PDE system generates a $C_0$-semigroup. The pressure term in the 3D Stokes equation adds a great challenge to the analysis. The author used non-Leray-based elimination of the associated pressure term. The elastic solution was found via a Babuska-Brezzi approach. Long term strong stability was again shown by analyzing the spectrum of the associated operator that generates this $C_0$-semigroup. Of note, the author shows that zero is an eigenvalue of $\A$. From this, the author addresses the issue of asymptotic decay of the solution to the zero state for any initial data taken from the orthogonal complement of the zero eigenspace Null$(\A)^\perp$.\\