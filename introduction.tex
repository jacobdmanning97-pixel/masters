\chapter{Introduction}

Multilayered fluid-structure interaction (FSI) partial differential equation (PDE) dynamics arise in the context of blood transportation in mammalian arteries and shape deformation of cells in cellular dynamics. The physiological interaction between the arterial walls and blood flowing through plays an important role in cardiovascular systems of cells \cite{ap-1,FSIforBIO,hsu,BorisSimplifiedFSI,SunBorMulti}. For example, in existing literature on FSI models, often only a single layer structural PDE component is presented. We see for example in \cite{lions1969quelques} (Section 9) and \cite{RauchZhangZuazua} that coupled heat-wave systems were examined  where the heat equation is a simplification of the fluid flow and the wave equation is a simplification of the elastic structure. Although these simplifications are instructive, there is realism lost by the changes as vascular walls are typically multilayered \cite{buk,Coutand,SunBorMulti}.\\
\indent In a pioneering paper \cite{SunBorMulti}, the authors considered a multilayered FSI PDE model where multilayered FSI is composed of 2D ``thick'' layer wave equation and 1D ``thin'' layer wave equation coupled to a 2D fluid PDE across a boundary interface. In this paper it showed that thin structure with mass at the fluid-structure interface regularizes the FSI dynamics. Being inspired, the authors in \cite{AGM} proposed a new multilayered 3D heat, 2D wave, 3D wave system in order to model eukaryotic cellular dynamics. This ``canonical'' model will be a topic of focus in this project. As mentioned before, the two 3D heat and wave equations are modeling more complicated flow and structural equations respectively. In the paper, it was shown through a semigroup approach that this system is wellposed by a combination of Lumer-Phillips and Lax-Milgram arguments as well as exhibiting strong stability by analyzing the spectrum of the generator of the semigroup. It should be noted that the authors originally intended to write a follow up paper that focused on a numerical implementation of this model. To this end, the canonical model is framed as a convex polygonal domain representing the nucleus instead of a smooth domain.\\
\indent Recently, in \cite{Ger} a more realistic model of 3D Stokes flow, 2D elastic wave, 3D Lam\'e elastic system was proposed with a smooth nuclear domain. This model will be the other topic of focus for this project. Similar to the canonical model, wellposedness was shown as well as strong stability under mild assumptions later in \cite{Ger2}. In \cite{Ger} one of the major problems the author overcame was dealing with the pressure term that is introduced with the Stokes flow. It was overcome by solving elliptic boundary value problems given in the fluid pressure variable which appeals to the application of the Lax-Milgram Theorem. As the Stokes fluid system is still linear, the main result still uses a Lumer-Phillips approach. However, as the pressure term adds more complexity, the author used the Babuska-Brezzi Theorem, in order to show the maximal dissipativity of respective generator of the associated $C_0$-semigroup. In \cite{Ger2}, the author again analyzed the spectrum of the generator of the semigroup. However, as the origin is on the imaginary axis, and zero is an eigenvalue of the associated generator, the strong stability is restricted to initial data in the orthogonal compliment of the zero eigenspace. Another issue that the author ran into in the spectral analysis is by estimating the 3D ``thick'' elastic solution variable, an overdetermined eigenvalue problem arises. This forces the geometrical imposition to guarantee that the spectrum does not intersect the imaginary axis. This assumption has been used in similar FSI problems such as \cite{Av, GR, av-trig}.\\
\indent Our main goal is to investigate the requirements to obtain the main results of each PDE model, we hope is to compare and contrast the methodologies used in each paper with respect to the given assumptions. There are major differences between the two models as the Stokes flow adds additional complexity not present in the Canonical model, such as the pressure term in the second model. We have structured this project in a way to highlight the similarities and differences of the methodologies used to show the wellposedness and long term behavior of each model in Part I and Part II respectively.